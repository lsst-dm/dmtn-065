\section{Previously Proposed Special Programs}\label{sec:prevpropsp}

{\bf This section has not been updated since 2018.}

In this section we compile information about the science goals and observational methods for Special Programs that have been previously proposed or discussed in the Science Community. We use these to infer the potential deviations from standard visit images, and to get a basic idea of the DM processing needs that would be required to enable the science. The main resources from which we have collected information about the Community's Special Program are: \citep{2008arXiv0805.2366I}; \citep{LPM-17}; the LSST Deep Drilling Field white papers from 2011\footnote{\url{https://project.lsst.org/content/whitepapers32012}}; presentations by Niel Brandt and Stephen Ridgway at the LSST Project and Community Workshop in August 2016\footnote{\url{https://project.lsst.org/meetings/lsst2016/sites/lsst.org.meetings.lsst2016/files/Brandt-DDF-MiniSurveys-01.pdf} and \url{https://project.lsst.org/meetings/lsst2016/sites/lsst.org.meetings.lsst2016/files/Ridgway-SimulationsMetrics_1.pdf}}; \citep{2013arXiv1304.3455G}; and Chapter 10 of \citep{2017arXiv170804058L}.

So far, only one aspect of the LSST Special Programs are set: the locations of the four chosen deep drilling fields\footnote{\url{https://www.lsst.org/scientists/survey-design/ddf}}. There are three mini-survey areas that have been discussed extensively by the Science Community: the North Ecliptic Spur (NES), the South Celestial Pole, and the Galactic Plane (see Figure 8 of \citep{2008arXiv0805.2366I}). In Table \ref{tab:ddfms} we list the four extragalactic deep drilling fields have already been specified, along with an \textit{incomplete} list of potential mini-surveys that have been openly discussed in the Science Community. In Section \ref{sec:SPCS}, we create detailed DM Processing Case Studies for several of these Special Programs in order to identify any potential issues with reconfiguring the DM pipelines to create specific data products for these programs.

\begin{table}[h]
\begin{center}
\begin{footnotesize}
\caption{Approved DDF and Incomplete List of Potential Special Programs.}
\label{tab:ddfms}
\begin{tabular}{lll}
\hline \hline
Name & Coordinates & Description  \\
\hline
DDF Elias S1    & 00:37:48, -44:00:00  & approved, cadence TBD \\
DDF XMM-LSS & 02:22:50, -04:45:00  & approved, cadence TBD  \\
DDF Extended Chandra Deep Field-South & 03:32:30, -28:06:00  & approved, cadence TBD  \\
DDF COSMOS  & 10:00:24, +02:10:55 & approved, cadence TBD  \\
\hline
North Ecliptic Spur      & & solar system objects (find and characterize) \\
Galactic Plane             & & more intensive stellar surveying \\
South Equatorial Cap  & & S/LMC and more Galactic science \\
Twilight                        & & short exposures (0.1s) for bright stars \\
Mini-Moons                     &  & finding mini-moons \\
Sweetspot                       & & 60 deg from Sun for NEOs on Earth-like orbits \\
Meter-Sized Impactors     & & detection a week before impact \\
GW Optical Counterparts & & search and recovery \\
Old Open Cluster M67      & dec +12 & compact survey above Galactic plane  \\
\hline
\end{tabular}
\end{footnotesize}
\end{center}
\end{table}

Here we consider a variety of scientific fields in turn, the Special Programs that have been discussed in that Science Community so far, and the implications of these Programs for the diversity of data and data products. Generally, the types of LSST Special Programs that are open for proposals include: (i) additional deep drilling fields; (ii) refined observing strategies for deep drilling fields; (iii) optimized survey areas for the NES, South Pole, and Galactic Plane; (iv) refined observing strategies for the NES, South Pole, and Galactic Plane; and (v) additional mini-surveys (areas and observing strategies).

\medskip
\noindent \textbf{A Nominal DDF Observing Strategy -- } Ivezi\'{c} et al. (2008, \citep{2008arXiv0805.2366I}; Section 3.1.2) describes a nominal DDF data set as $\sim50$ consecutive $15$ second exposures in each of four filters, repeated every two nights for four months. Each exposure would have a $5\sigma$ limit of $r\sim24$; the nightly stack would have a limit of $r\sim26.5$; and the final deep stack of all exposures would have a limit of $r\sim28$. This description does not comment on the processing mode, but, depending on the science goals the exposures could be done as either a series of 50 non-standard visits ($1\times15$ seconds) or 25 standard visits ($2\times15$ seconds). 

\medskip
\noindent \textbf{Solar System Objects (SSO) -- } Four of the mini-surveys in Table \ref{tab:ddfms} have science goals related to studies of SSO. Observations of the North Ecliptic Spur area could yield more $\geq140$ m near-earth objects (NEOs) for the final LSST sample (reference: Brandt's talk). The Mini-Moons Mini-Survey aims to find and study the temporarily captured satellites of the Earth (Section 10.2, \citep{2017arXiv170804058L}). The Sweetspot Survey would use twilight fields to find NEOs in Earth-like orbits (i.e., these objects are never in opposition fields, but overhead at sunrise/sunset; Section 10.2, \citep{2017arXiv170804058L}). The Meter-Sized Impactors program would find and track meter-sized impactors $<2$ weeks before impact (Section 10.2, \citep{2017arXiv170804058L}). {\bf Summary:} most of these science goals do not seem to require non-standard visits or exposure times, with the exception of the Sweetspot survey which occurs during twilight and thus may require shorter exposures. The cadence and patterns of these mini-surveys may differ from the WFD main survey, especially when very fast-moving objects are sought. From a processing perspective, it seems that many of these science goals will be achievable by using the products of Solar System Processing, which runs on the Prompt Pipeline's \texttt{DIASource} catalogs after they are updated each night. The exception is finding faint SSOs (e.g., Trans-Neptunian Objects Trojans, asteroids, long-period comets, dwarf planets) through shift-and-stack (SAS) processing \citedsp{Document-11013}, because SAS is not a capability being built within the DM system and cannot be done solely by reconfiguring DM pipelines. An example of user-generated pipeline for SAS is described in Section \ref{sec:SPCS}.

\medskip
\noindent \textbf{Stars in the Milky Way and Magellanic Clouds -- } As described in \citedsp{Publication-141}, mini-surveys of the Galactic Plane can better distinguish faints stars from faint red galaxies by including at least 3 filters of coverage (e.g., $izy$; similar to WFD), and could mitigate losses from proper motion and increase the detection rate of stellar flares by obtaining all the images in short time span (i.e., a more concentrated cadence than the WFD).  As described in \citedsp{Publication-145}, applying the nominal DDF observing strategy over the full area of the Large and Small Magellanic Clouds can characterize stellar variability to $M_V<6.5$ on timescales from 15 seconds to 3 days. For this, special co-adds may be required, e.g., \textit{"to reach variability levels of 0.1 to 0.005 mag will require co-adds depending on the timescale of the particular variables"} \citedsp{Publication-145}. The Twilight survey in Table \ref{tab:ddfms} proposes short exposures to enable bright stars to be put on the same photometric system as the deeper LSST WFD main survey catalog, and enable science that is based on their long monitoring baselines from historical observations. In Chapter 10.4 of \citep{2017arXiv170804058L}, a proposed short-exposure survey of M67 would use the camera's stretch goal of $0.1$ second exposures or, if that is not possible, \textit{"custom pixel masks to accurately perform photometry on stars as much as $6$ magnitudes brighter than the saturation level"}. {\bf Summary:} while some of these science goals can be accomplished with standard visits, MW \& L/SMC science goals are likely to request shorter exposure times, perhaps down to $0.1$ seconds. These science goals are also likely to propose cadence and filter distributions that are significantly different from the WFD main survey. From a processing perspective, the science goals depending on shorter exposures will only be able to be met by reconfiguring the DM pipelines if the short exposures can be shown to successfully be processed (with, e.g., instrument signature removal); the science goals can likely be met with data products in the same format as the Prompt or DR Pipeline (i.e., {\tt Source} and {\tt Object} catalogs, single visits and deep CoAdds). Although it is not mentioned in the above paragraph, the MW \& L/SMC science community is also most likely to require special processing to extract information from saturated stars, which is outside the scope of DM. See Section \ref{ssec:SPCS_GPVSEx} for more detailed DM processing case studies.

\noindent \textbf{Exoplanets -- } As described in Section 3.1.2 of \citep{2008arXiv0805.2366I}, transiting exoplanets could be detected with the nominal DDF plan, which would allow for $1\%$ variability to be detected over hour-long timescales; a DDF field at Galactic latitude $30$ degrees would yield $10^6$ stars at $r<21$ that would have $\mathrm{SNR}>100$ in each single exposure of the sequence. \citep{2013arXiv1304.3455G} describes how transits can be extract from a wider-area survey of the Galactic Plane, and how microlensing candidates can be found with $\sim22$ mag imaging over the Galactic Plane region every 3-4 days (since microlensing events are slower; these would then require follow-up with external facilities). Dealing with the more crowded fields would be mitigated by the shallower images, in this case. One of the main points of \citep{2013arXiv1304.3455G} is that the Galactic Plane can yield a lot of science despite the fact that its eventual deep co-adds would be uselessly confusion limited, and therefore should not be skipped. \textbf{Summary.} Some of these science goals appear possible with standard visit images, and some might request shorter exposures to avoid confusion in crowded fields when the science can be done with brighter stars. From a processing perspective, the science goals are likely to be achievable with reconfigured DM pipelines, but this depends heavily on performance in crowded fields. See Section \ref{ssec:SPCS_GPVSEx} for a more detailed DM processing case study for Galactic Plane regions.

\noindent \textbf{Supernovae -- } The nominal DDF plan described in \citep{2008arXiv0805.2366I}, which builds nightly stacks with a limit of $r\sim26.5$ out of standard visit images, would extend the SN sample to $z\sim1.2$ and provide more densely sampled light curves for cosmological analyses. The optimal exposure time distribution might be 6, 5, 10, 10, 9, 10 in $ugrizy$ \citedsp{Publication-144}. High-cadence observations of DDF would be the only way to detect fast transients, particularly extragalactic novae, some tidal disruption events, optical counterparts to gamma-ray bursts, and peculiar SNe \citep{2014ApJ...794...23D}. Generating the best-possible individual SN light curves for cosmological analyses requires building special, deep-as-possible, SN-free host galaxy images and using them as a template. This will also be necessary for studying SNe that appear in the template image; i.e., that last $>1000$ days. These are mostly Type IIn, probably explosions of massive stars into dense circumstellar material, which are not used for cosmology but rather to study late-stage stellar evolution and mass loss. SN-free images will also be needed to measure correlated properties for cosmology and to do host-galaxy science. The latter, specifically the "characterization of ultra-faint SN host galaxies", is also mentioned in the Galaxies DDF WP \citedsp{Publication-142}. Short-exposure observations of bright, nearby SNe may also be useful to include near-peak photometry in the LSST magnitude system, and enable full light-curve analyses. \textbf{Summary.} All of these science goals appear possible with standard visit images (with the exception of a target-of-opportunity short-exposure program to observe bright SNe). From a processing perspecitve, the science goals appear to be accessible with reconfigured DM pipelines to stack and difference the data. In particular, the DRP codes to create "transient-free CoAdds" will be suitable for generating the SN-free templates for DDF, as they will do for the Main Survey images. See also Section \ref{ssec:SPCS_SNDDF} for a DM processing case study to find SNe in a DDF.

\noindent \textbf{Galaxies -- } The additional depth of a DDF may provide access to a larger collection of low-$\mu$ objects. \citedsp{Publication-142} mentions "identification of nearby isolated low-redshift dwarf galaxies via surface-brightness fluctuations" and "characterization of low-surface-brightness extended features around both nearby and distant galaxies". The DDF stacks could also be used to characterize of high-$z$ clusters, although this ability might depend on deblending extended objects. Also, the DDF observations, when combined with the WFD, allow for AGN monitoring on a variety of timescales in well-characterized galaxies \citedsp{Publication-142,Publication-143}. \textbf{Summary.} As with the SN science goals, these use standard visit images and reconfigured DM pipelines to make deep CoAdds and extract sources. In addition, it seems likely that user-generated algorithms that are optimized to detect and characterize particular types of faint extended sources will be needed, and these are beyond the scope of DM.

\noindent \textbf{Weak Lensing -- } The deeper imaging from DDFs can help with shear systematics and the effects of magnification in the analysis of WFD data (community forum, Jim Bosch). \textbf{Summary.} As with the SN and Galaxies DDF-related science goals, these use standard visit images and reconfigured DM pipelines can be used to make deep CoAdds and extract sources, as Jim notes.
%$\bullet$ \textit{Jim Bosch -- "Will need to process at least some deep drilling fields (high-latitude ones) in the same way we process a full data release production before running the full data release production, so we can use the results to build priors and/or calibrate shear estimates on the wide survey"} (\texttt{\Large{Community}} forum) \\
%$\bullet$ \textit{Jim Bosch -- "Will need to process various wide-depth subsets of some deep drilling fields (again, high-latitude ones) using the regular DRP pipeline. We'll definitely want best-seeing, worst-seeing, and probably a couple of independent typical-seeing subsets, but there may be other ways we'd want to subdivide as well."} (\texttt{\Large{Community}} forum)  \\
%$\bullet$ \textit{MLG side note -- Photo-$z$ are very important to weak lensing \citedsp{Document-10963} and so perhaps the implemented method should be chosen with weak lensing science prioritized.} \\