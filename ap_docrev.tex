\section{Documentation Review for Requirements Related to Special Programs}\label{sec:docrev}

This review is limited to requirements related to Special Programs processing or data products,
or requirements that constrain the diversity of images (e.g., exposure time limits).

Updates to Rubin documents related to Special Programs that were motivated by past versions of this DMTN
were made via LCR-1309 and LCR-2265. 

% LCR 1309: https://project.lsst.org/groups/ccb/node/2383
% LCR 2265: https://project.lsst.org/groups/ccb/node/4036


\subsection{Science Requirements Document (SRD)}

Version 5.2.4 (revision 2018-01-30), \citedsp{LPM-17}.

Section 3.4 ``The Full Survey Specifications" states the SRD's assumption that 90\% of the total 
available survey time would be spent on the main survey, and that the remaining 10\% would be spent 
{\it ``to obtain improved coverage of parameter space ... [or to] observe special regions"}.



\subsection{LSST System Requirements (LSR)}

Version 7.1 (revision 2020-03-05), \citeds{LSE-29}.

Note that Version 5 (2018-06-26) was an update for LCR-1309, which added requirements, specifications, 
and discussions regarding the processing of Special Programs data based on earlier versions of this DMTN.

In Section 1.5.1.3, ``Processing Data from Special Programs", LSR-REQ-0122\lsrreq{0122} is a requirement 
that the LSST system {\it ``shall deliver unique and separate data products for visits from Special Programs"} 
whenever possible, and that they {\it ``shall be delivered on timescales intermediate"} to the Prompt and 
Data Release timescales {\it ``when this enables the intended science of the Special Program"}.
The discussion clarifies that {\it ``the term 'whenever possible' includes cases where the Data Management System 
can run original or reconfigured versions of existing pipelines, and excludes cases where the development of new 
algorithms, or the allocation of significant additional computational resources, are required"}.

In Section 2.4.1.1.2, ``Non-Standard Visit", LSR-REQ-0111\lsrreq{0111} requires that the LSST system 
{\it ``be capable of obtaining and processing exposures not taken in a standard visit mode including those with a 
minimum exposure time of"} 1 second ({\tt minExpTime}\reqparam{minExpTime}).
The discussion notes that {\it ``non-standard visit exposures may possibly be degraded in some aspects of performance 
(e.g. cosmic ray rejection on visits consisting of a single exposure), and might be incompatible with difference 
imaging and alert production (e.g., short exposures in which the PSF is not fully formed)"}.

The requirement in Section 1.5.1.3 is echoed in Section 2.6.1.1, ``Organization of Data Products", in which 
LSR-REQ-0032\lsrreq{0032} is a requirement that the data processing system provide the means for three 'classes' of 
data products on different timescales (Prompt, Data Release, and User-Generated), and also to provide a means for 
processing Special Programs data because the {\it ``science goals of Special Programs may require that their processed 
data products be made available in an additional fourth class, and possibly with intermediate timescales"}.

In Section 2.6.1.1.3, ``Level 3 Data Products", LSR-REQ-0041\lsrreq{0041} specifies that the LSST system 
{\it ``shall support"} User-Generated data products.
The discussion clarifies that ``{\it there will be technical limits on DM's ability to meet this requirement, such as 
cases where an intensive amount of additional computational resources is required, because only ~10\% of the total 
computational system is allocated for user processing"}.
This level of support applies also to user processing of Special Programs data.
See also the reference to LSR-REQ-0055 below.

Section 2.6.1.1.4, ``Data Products for Special Programs", LSR-REQ-0121\lsrreq{0121} specifies that the LSST system 
{\it ``shall produce unique and separate Data Products as the result of processing data from Special Programs whenever 
possible, on a timescale that enables the intended science goals of the Special Program.
The cumulative size of the online Special Programs data products shall be no more than ~10\% of the size of the DRP 
data products from the most recent data release"}.
The discussion clarifies that {\it ``the term 'whenever possible' includes cases where the Data Management System can 
run original or reconfigured versions of existing pipelines, and excludes cases where the development of new 
algorithms, or the allocation of significant additional computational resources, are required.
The cumulative size of the Special Programs data products is capped at ~10\% of the most recent DR because this 
matches the expected fractional survey area of Special Programs compared to the main survey"}.

In Section 2.7.1.6, ``Community Computing Services", LSR-REQ-0055\lsrreq{0055} requires that the LSST system 
{\it ``shall provide and maintain an amount of computing capacity equivalent to at least"} 10\% 
({\tt userComputingFraction}\reqparam{userComputingFraction}) {\it ``of the total LSST data processing capacity 
(computing and storage) for the purpose of scientific analysis of LSST data and the production of"} User-Generated 
data products. 
The discussion clarifies that the scope of this service remains to be determined.
This level of computational resources includes user processing of Special Programs data.

In Section 3.1.3.1, ``Survey Time Allocation", LSR-REQ-0075\lsrreq{0075} requires that the {\it ``survey performance 
requirements shall be met utilizing approximately 90\% of the historically available observing time, leaving the 
remaining time available for yet to be defined special programs"}.


\subsection{Observatory System Specifications (OSS)}

Version 19.3 (revision 2022-12-12), \citeds{LSE-30}.

Note that Version 13 (2018-06-26) was an update for LCR-1309, which added requirements, specifications, and discussions regarding the processing of Special Programs data based on earlier versions of this DMTN.

In Section 2.2.3.1, ``Standard Operating States", OSS-REQ-0044\ossreq{0044} specifies that {\it ``the LSST 
observatory system shall be designed and constructed to support ... manual observing - used for specific 
non-scheduler driven observing to support system verification and testing or specialized science programs"}. 
Although most Special Programs will be executed via the survey scheduler as part of {\it ``fully automated 
observing"}, manual observing might be necessary for, e.g., target-of-opportunity Special Programs.

Section 3.1.5.1.2, ``Data Products Handling for Special Programs", OSS-REQ-0392\ossreq{0392} is a flow-down of 
requirements from the LSR (0122, 0075, and 0121; LSE-29), and specifies that {\it ``the handling of data products 
from Special Programs shall be compliant with the approach defined in LSE-163"}.

In Section 3.6.1.3, ``Continuous Exposures", OSS-REQ-0319\ossreq{0319} requires that {\it ``The Observatory shall be 
capable of continuous operation throughout a night with the interval between successive visits equal to the FPA 
readout time"}.
The discussion clarifies that {\it ``this mode of observing is needed to support observations when the telescope is 
not being re-pointed. For example observing ``deep drilling" fields..."}.

In Section 3.6.1.4, ``Minimum Exposure Time", OSS-REQ-0291\ossreq{0291} specifies that {\it ``the camera shall be 
able to obtain a single exposure with an effective minimum exposure time of no more than"} 1 second 
({\tt minExpTime}\reqparam{minExpTime}) {\it ``with a goal of an effective minimum exposure time of"} 0.1 seconds 
({\tt minExpTimeGoal}\reqparam{minExpTimeGoal}). 
The discussion clarifies that {\it ``if the exposure is shortened from the 15 second nominal, the spacing between 
successive exposures should be extended to maintain the average readout rate consistent with a 15 second exposure"}, 
which may increase the overheads of Special Programs using short exposure times.
The discussion also clarifies that {\it ``if the exposure is lengthened from the 15 second nominal, the thermal 
stability may be affected, which may affect photometric accuracy"}.

In Section 3.6.1.5, ``Publish Visit Type", OSS-REQ-0384\ossreq{0384} specifies that {\it ``the OCS [Observatory 
Control System] shall configure the [Data Management System] DMS (in particular Prompt Processing) with the type 
of visits to be processed: Standard, Alternate, or a specific type of Non-Standard"}.
The discussion clarifies that this allows the Prompt processing pipeline to be reconfigured on-the-fly in order to 
incorporate non-standard visits from, e.g., Special Programs.
The time required for reconfiguration might introduce some latency or cause some images to not be processed by the 
Prompt pipeline.

In Section 3.6.2.1.2, ``Maximum time for operational filter change", OSS-REQ-0293\ossreq{0293} specifies that 
{\it ``the camera system shall provide the capability of changing the operational filter with any other internal 
filter in a time less than"} 120 seconds ({\tt tFilterChange}\reqparam{tFilterChange}).
This would impose a large overhead on, e.g., a Special Program that changes filters often without slewing.
See also OSS-REQ-0295\ossreq{0295}, Appendix~\ref{sec:hardbounds} of this document, and/or the filter change 
memorandum (\url{ls.st/spt-494}), for more information about the total lifetime number of filter changes.

In Section 3.6.3.1, ``Absolute Pointing", OSS-REQ-0298\ossreq{0298} specifies that {\it ``the LSST shall point to 
a defined set of sky coordinates with an RMS accuracy of"} 2 arcseconds ({\tt absPointErr}\reqparam{absPointErr}).

In Section 3.6.3.3, ``Rotator tracking Time, OSS-REQ-0301\ossreq{0301} specifies that {\it ``the LSST shall be able 
to maintain field rotation tracking over a period of at least"} 1 hour ({\tt rotTrackTime}\reqparam{rotTrackTime}).
The discussion clarifies that this {\it ``is driven by the need to conduct extended 'deep drilling' observations on 
a single field"}.

In Section 3.6.3.5, ``Offset Pointing", OSS-REQ-0302\ossreq{0302} specifies that {\it ``the LSST shall be capable of 
offset pointing within a single field-of-view with a precision of no more than"} 0.2 arcseconds 
({\tt offsetPointingErr}\reqparam{offsetPointingErr}).

In Section 3.6.3.6, ``Open Loop Tracking", OSS-REQ-0303\ossreq{0303} specifies that {\it ``The LSST shall be 
capable of open loop tracking without the assistance of real time optical feedback to an accuracy of"} 1.0 arcseconds 
({\tt openTrackErr}\reqparam{openTrackErr}) {\it ``over any 10 minute duration during normal night time operations"}.
Note that the open loop tracking requirement is \emph{without guiding}.

In Section 3.6.3.10, ``Non-Sidereal Tracking", OSS-REQ-0380\ossreq{0380} specifies that {\it ``the LSST system shall 
be capable of tracking in an arbitrary direction on the sky along a parametric RA(t) and DEC(t) trajectory, at 
angular rates of up to"} 220 arcseconds per second ({\tt nonsiderealAngularRateEl}\reqparam{nonsiderealAngularRateEl} 
and {\tt nonsiderealAngularRateAZ}\reqparam{nonsiderealAngularRateAZ}) {\it ``with a tracking error not to exceed"} 
0.5 arcseconds per minute ({\tt nonsiderealTrackingError}\reqparam{nonsiderealTrackingError}).
The discussion notes that {\it ``this is standard capability for modern telescopes"}, but might be relevant to some 
Special Programs.


\subsection{Data Management Subsystems Requirements (DMSR)}

Version 9 (revision 2021-02-12), \citeds{LSE-61}. 

Note that Version 8.3 (2020-05-04) was an update for LCR-2265, which updated requirements, specifications, and 
discussions regarding the processing of Special Programs data based on earlier versions of this DMTN.

In Section 1.2.3, ``Raw Science Image Metadata", DMS-REQ-0068\dmreq{0068} specifies that {\it ``for each raw 
science image, the DMS shall store image metadata"} including {\it ``Program metadata (identifier for main survey, 
deep drilling, etc.)"}.
The discussion clarifies that {\it ``the program metadata should be sufficient to associate an image with a specific 
Special Program so that DMS-REQ-0320 and DMS-REQ-0397 can be satisfied"}.

In Section 1.3.13, ``Alert Content", the discussion for DMS-REQ-0274\dmreq{0274} explains that the {\it ``program 
and/or scheduler metadata"} included in an alert packet {\it ``should be sufficient to identify whether the image is 
associated with a Special Program (such as an in-progress Deep Drilling Field)"}.

In Section 1.4.18.1, ``Produce All-Sky HiPS Map", the discussion for DMS-REQ-0379\dmreq{0379} raises the point that 
generating separate HiPS maps for Special Programs (e.g., DDFs) remains an open question.

In Section 1.4.18.5, ``Produce MOC Maps", DMS-REQ-0383\dmreq{0383} specifies that Data Release processing 
{\it ``shall include the production of Multi-Order Coverage maps for the survey data"}, and that {\it ``additional 
MOCs SHOULD be produced to represent special-programs datasets"}.
It is noted that a separate technical note would be created to define these MOCs.

The bulk of the DMS's requirements related to Special Programs are in Section 1.6 of the DMSR.

In Section 1.6.1, ``Processing of Data From Special Programs", DMS-REQ-0320\dmreq{0320} specifies that {\it ``it 
shall be possible for special programs to trigger their own data processing recipes, during the night instead of 
the nightly Alert Processing (but the recipes may still issue Alerts), or on alternative timescales"}.
The discussion clarifies that the {\it ``LSST will provide these recipes ... when possible, which includes cases 
where DM can run original or reconfigured versions of existing pipelines, and excludes cases where the development 
of new algorithms, or the allocation of significant additional computational resources, are required. An example of 
an alternative timescale is a nightly trigger to coadd all the deep-drilling field images. Decisions about which 
recipes are applied to which Special Programs will be made by the Operations team, after consideration of the 
scientific goals, computational resources, and data rights policy"}.
This requirement is derived from OSS-REQ-0392, which is essentially a flow-down of requirements from the LSR (0122, 
0075, and 0121).

In Section 1.6.2, ``Prompt/DR Processing of Data from Special Programs", DMS-REQ-0397\dmreq{0397} specifies that 
{\it ``it shall be possible for special programs data to be processed with the prompt and/or annual-release pipelines 
alongside data from the main survey"}.
The discussion further clarifies that {\it ``the data from Special Programs should only be included ... when it is 
(a) possible ... to do so without additional effort and (b) beneficial to the LSST's main science objectives. 
Decisions about which data are included ... will be made by the Operations team"}.
This requirement is also derived from OSS-REQ-0392, which is essentially a flow-down of requirements from the LSR 
(0122, 0075, and 0121).

In Section 1.6.3, ``Level 1 Processing of Special Programs Data", DMS-REQ-0321\dmreq{0321} specifies that {\it ``all 
[Prompt] processing from special programs shall be completed before data arrives from the following night's 
observations"}.
This is essentially adding a quantifier to DMS-REQ-0397, to specify that {\it ``when it is (a) possible ... to do so"} 
means when it is possible to complete the processing before the next night's observations.
This requirement is also derived from OSS-REQ-0392, which is essentially a flow-down of requirements from the LSR 
(0122, 0075, and 0121).

In Section 1.6.4, ``Constraints on Level 1 Special Program Products Generation", DMS-REQ-0344\dmreq{0344} specifies 
that {\it ``the publishing of [Prompt] data products from Special Programs shall be subject to the same performance 
requirements of"} 24 hours ({\tt L1PublicT}\reqparam{L1PublicT}) for the release of Prompt data products and 1 minute 
({\tt OTT1}\reqparam{OTT1}) for the transmission of Alert packets.
This is essentially a more detailed version of DMS-REQ-0321 which includes the Alert production timescale.
This requirement is also derived from OSS-REQ-0392, which is essentially a flow-down of requirents from the LSR 
(0122, 0075, and 0121).

In Section 1.6.5, ``Special Programs Database", DMS-REQ-0322\dmreq{0322} specifies that {\it ``data products for 
special programs shall be stored in databases that are distinct from those used to store standard [Prompt] and 
[Data Release] data products"} and that {\it ``it shall be possible for these databases to be federated ... to allow 
cross-queries and joins"}.
This requirement is also derived from OSS-REQ-0392, which is essentially a flow-down of requirements from the LSR 
(0122, 0075, and 0121).

In Section 4.1.16, ``Level 2 and Reprocessed Level 1 Catalog Access", DMS-REQ-0313\dmreq{0313} specifies that 
{\it ``the DMS shall maintain ... versions of the most recent catalogs generated from Special Programs data"}. 
As with all LSST data, {\it ``there is no requirement for older data releases to be queryable"}.


\subsection{Data Management Applications Design (DMAD)}

Version 4.3 (revision 2020-11-10), \citeds{LDM-151}.

The DMAD is not a requirements document.
Instead, it describes the scientific design of the LSST Science Pipelines: the algorithms and software that will 
be implemented to meet the requirements for processing the LSST data. 

Special Programs are only mentioned a few times, either as a potential source of single-snap visits or as a 
potential source of reference images or catalogs (e.g., training sets).

As described above (e.g., LSR-REQ-0122), the LSST system shall deliver unique and separate data products for visits 
from Special Programs whenever this (1) enables the intended science of the Special Program and (2) can be 
accomplished using the original or reconfigured versions of the LSST Science Pipelines.
For cases in which the development of new algorithms or the allocation of significant additional computational 
resources are required to produce Special Programs data products, User-Generated pipelines and processing will be 
necessary.

The DMAD can be used as the reference document to decide whether a given Special Program will require User-Generated pipelines and processing.


\subsection{Data Products Definitions Document (DPDD)}

Version 3.6 (revision 2021-12-17), \citeds{lse-163}.

Section 6 describes the data products for Special Programs.
The DPDD is not a requirements document; Section 6 summarizes the requirements presented above and does not introduce 
any new constraints or new information about Special Programs. 
