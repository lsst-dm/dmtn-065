\section{Processing for Special Programs Data Products}\label{sec:proc}

Rubin Observatory and the LSST Data Management System (DMS) are required 
to process Special Programs data to produce unique and separate data 
products whenever the original or reconfigured versions of the LSST 
Science Pipelines can be used to do so (Section~\ref{sec:req}).

The LSST Science Pipelines and the DMS are designed to process standard 
(or alternative standard) visits.

It depends on the science goals related to the Special Program, but some 
Special Programs' data can and should be processed and served along with 
(or as a part of) the WFD data products (Section~\ref{ssec:proc_wfd}).
Others will require RDM to generate unique and separate data products 
(Section~\ref{ssec:proc_rdm}). 

Cases where the Special Program obtains non-standard visits that cannot 
be processed by the DMS (Section~\ref{ssec:proc_bounds}), and/or where 
development of new algorithms or the allocation of significant additional 
computational resources are needed, will require users tp generate unique
and separate data products (Section~\ref{ssec:proc_user}).


\subsection{Boundaries on Obtaining and Processing Non-Standard Visits} \label{ssec:proc_bounds}

Most of the Special Programs that are currently under consideration will 
use standard or alternative standard visits.

However some Special Programs are likely to use non-standard visits with 
shorter exposures (5 sec), and some are likely to acquire images with a 
significantly brighter sky background (e.g., twilight images) than 
standard (or altnerative) WFD main survey visits.

The cadence and patterns of Special Programs might also differ from the 
WFD, such as long series of exposures obtained of the same field 
(e.g., DDFs), or a strategy optimized to find very fast-moving objects 
(e.g., NEOs).

None of the currently proposed Special Programs listed in the Phase 2 
SCOC recommendations \citedsp{PSTN-055} 
seem likely to approach (or violate) any harware boundaries 
(Section~\ref{ssec:proc_bounds_hardware}).

However, some Special Programs observations seem likely to generate 
images that cannot be processed with original or reconfigured versions of 
the LSST Science Pipelines (Section~\ref{ssec:proc_bounds_processing}), 
and will require user processing.


\subsubsection{Hardware Boundaries}\label{ssec:proc_bounds_hardware}

Appendix~\ref{sec:hardbounds} lists all of the hardware boundaries that 
might constrain the potential diversity of Special Programs data.

The minimum exposure time is 1 second (stretch goal: 0.1 seconds), and 
there is a potential hardware boundary that limits the readout rate to 1 
every 15 seconds.
This would affect the image acquisition rate and increase the overheads 
on short exposures.

Special Programs which required the \emph{exact same} field pointing and 
rotation for \emph{every exposure} (to sub-arcsecond levels) might run 
into hardware boundaries on pointing and tracking.
None of the currently-proposed programs seem likely to be affected by 
this potential boundary. 

Hardware imposes no other boundaries on how data can be obtained, but 
Special Programs that request a high number of filter changes and/or long 
slews could be inefficient due to large overheads.


\subsubsection{Processing Boundaries}\label{ssec:proc_bounds_processing}

Appendix~\ref{sec:procbounds} describes the boundaries on what types of visits 
can be processed and calibrated by the DMS.

Very short ($<$2 sec) exposures could be difficult to reduce due to an 
incompletely-formed PSF, and very short or very long ($>$150 sec) 
exposures could be difficult to calibrate due to having too few (or too 
few unsaturated) stars.
As mentioned above the DMS is required to be able to process exposure 
times as low as 1 second, and it is known that such short exposures might 
have degraded image quality.

It is currently unclear whether images with very bright sky backgrounds 
(twilight images) can be processed with the LSST Science Pipelines, or 
whether user generated pipelines will be needed.

The full reduction and calibration of images obtained with non-sidereal 
tracking, in which the stars are streaked, is currently beyond the scope 
of the LSST Science Pipelines, and will require a user generated pipeline.


\subsection{Inclusion in the WFD Data Products}\label{ssec:proc_wfd}

Generally, data from Special Programs should be included in the Prompt 
and/or Data Release processing and data products alongside data from the 
WFD (i.e., included in the data products described by \citeds{lse-163}) 
whenever this is (1) possible and (2) scientifically beneficial.

Decisions about when and whether to include Special Programs data in 
the WFD's processing and data products are left to the discretion of the 
Rubin Operations Data Production and System Performance teams.


\subsubsection{WFD Prompt Processing and Alert Production}

As all visits can contribute to the time-domain astronomy science goals 
of the LSST, all Special Programs data that {\it can} be processed by the 
prompt processing and alert production pipelines {\it should} be, 
alongside visits from the WFD.

This would include all standard and alternative visits in sky regions for which 
a template image exists.
This might also include some non-standard visits (shorter or longer exposures) 
as long as they can be processed by the Prompt 
pipeline\footnote{If a Special Program's science goals do require 
specialized templates and 
Prompt processing, the DMS will have the capability to load and use an 
alternative template for some regions, based on the image metadata. 
However, there would not be enough memory to hold alternative templates 
for the whole sky region.}.

Visits with very short or very long exposure times (or very bright sky 
backgrounds) might be excluded if they would 
need specialized algorithms for difference-imaging or template-generation.

Special Programs data that is processed by the Prompt pipeline would 
contribute to the Prompt data products described in Section 3 of the 
DPDD \citedsp{lse-163}. 

Difference images, {\tt DiaSources}, and alerts would have region and 
program labels (Section~\ref{ssec:sci_labels}), providing origin and
provenance for brokers and users.

\textbf{Two potential issues with Prompt processing for DDFs}, 
or any mini-survey that obtains a sequence of images without slewing, 
include:

(1) As alert packets contain the full records of all associated 
{\tt DiaSources} from the past 12 months, alerts for {\tt DiaObjects} in 
the DDFs might become very large.

(2) The association of {\it new} {\tt DiaSources} into {\tt DiaObjects} 
will be somewhat compromised for a DDF sequence.
The processing for the second image begins when the processing for the 
first image is only halfway done; when the {\tt DiaObject} catalog has 
not yet updated with the new {\tt DiaSources} detected in the first image.
Thus, the {\tt DiaSource} from images one and two for a new transient 
would not be associated with a single {\tt DiaObject}, but instead would 
each instantiate a new {\tt DiaObject}.

These two issues would not impact time-domain science.
Brokers and users would be able to use the region and program labels
in the data as context (i.e., as flags for potentially-compromised 
data products) and, for example, avoid DDF alerts should they so choose.


\subsubsection{WFD Solar System Processing}

Since Solar System Processing takes {\tt DiaSource}s as input, so any 
Special Programs images that can be run through the Alert Pipeline can 
also be incorporated into Solar System Processing.


\subsubsection{WFD Data Release Processing}

\textbf{Difference images, {\tt DiaSource} and {\tt DiaObject} catalogs -- }
As all visits can contribute to the time-domain astronomy science goals 
of the LSST, all Special Programs data that {\it can} be processed with 
the difference image analysis (DIA) pipeline alongside the WFD visits 
{\it should} be.
All of this processing should use the same template image for a given
field.
Users who want, e.g., only {\tt DiaSources} a given DDF, would be 
able to write queries using the region and program labels in the catalog.

\textbf{Processed visit images, {\tt Source} and {\tt ForcedSource} catalogs -- }
For the same reasons as above, all Special Programs data that 
{\it can} be processed with the Data Release pipieline alongside the WFD 
visits {\it should} be.

\textbf{Deep coadds and the {\tt Object} catalog -- }
Some of the core LSST science goals require a WFD main survey data 
products of nearly uniform depth.
Whether and how to include any Special Programs data in the deep image 
coadds and the corresponding {\tt Object} catalog is left entirely to the 
discretion of the RDM team in Operations.

For example, perhaps Special Programs images will only be included when 
they bring additional area to the same depth as the rest of the WFD, or 
when they suppress edge effects or low-order modes in the 
all-sky photometric solutions.

For stacked images and their associated catalogs, the region labels should be
included (but the program labels would not be needed or possible).


\subsection{RDM-Generated ``Unique and Separate" Data Products}\label{ssec:proc_rdm}

Rubin Data Management (RDM) is required to produce unique, separate, and 
joinable data products for Special Programs whenever this is possible with the 
original or reconfigured versions of the LSST Science Pipelines, and no new 
algorithmic development or significant computational resources are needed.

As described in Section~\ref{ssec:proc_wfd}, in some cases the science goals 
of a Special Program might be met by including the data in the WFD data 
products; especially the Prompt or DIA data products for standard visits.
Thus, most of the RDM-generated data products considered below are for 
static-sky Data Release data products, or data products for non-standard visits.

\textbf{A note about intermediate-depth ``custom" coadds -- }
As with the annual WFD deeply coadded images and associated catalogs, RDM would 
only generate one coadd (including all images to date) for a given special region.
Custom coadds (e.g., weekly, monthly) for very specific science goals are considered 
user-generated data products, for both WFD and Special Programs.

Decisions about when to produce separate Special Programs data products are 
left to the discretion of RDM.
Appendix~\ref{sec:spcs} provides more detailed, step-by-step data processing 
examples for some potential Special Programs as further illustration. 


\subsubsection{Deep Drilling Fields}

As the DDFs will likely be observed with standard or alternative standard 
visits, it is likely that RDM will be able to reconfigure existing pipelines 
to process the DDF data and produce ``unique and separate" DDF data products.

For example, RDM-generated ``unique and separate" DDF data products might 
include:
\begin{itemize}
\item nightly-coadded images (24 h)
\item nightly-coadded difference images (24 h)
\item {\tt DiaSource}- and {\tt DiaObject}-like catalogs for the nightly-coadds (24 h)
\item deeply-coadded images (all images to date; yearly)
\item templates for the nightly-coadded difference images (yearly)
\item {\tt Source}- and {\tt Object}-like catalogs for the nightly-coadded and deeply-coadded images (yearly)
\end{itemize}


\subsubsection{Short-Exposure Mini-Surveys}

As of the Phase 2 SCOC recommendations \citedsp{PSTN-055}, three kinds of 
short-exposure mini-surveys were under consideration: 
a static short exposure map of the sky in $ugrizy$ for calibration; 
a static to transient short exposure survey; and 
an Near-Earth Objects (NEO) twilight survey.

If either the transients or NEO short-exposure suvey is executed, and if 
RDM finds that the LSST Science Pipelines and, specifically, template images 
can be made and difference image analysis (DIA) run, then the RDM-generated 
data products may be similar to the Prompt data products (potentially 
including alerts; Section 3 of \citeds{lse-163}).

If either of the static short-exposure mini-surveys is executed, and if 
RDM finds that the LSST Science Pipelines can be reconfigured and used 
for short exposures, then the RDM-generated data products would likely be 
of a similar format to the Data Release data products 
(Section 4, \citeds{lse-163}).
For example, a separate repository of processed visit images and coadded 
images, and separate (but joinable) {\tt Source}, {\tt ForcedSource}, and 
{\tt Object} catalogs.


\subsubsection{Mini-Surveys}

The cases where ``unique and separate" data products for mini-surveys are 
needed to reach the science goals of a Special Program, and are possible for 
RDM to generate, are most likely to be of a similar format to the annual 
Data Release data products: static-sky coadded images and associated 
{\tt Source}- and {\tt Object}-like cataogs for standard or alternative visits.

In cases where the observations are all obtained with a short season 
(e.g., a week spent observing a specific region), such ``annual" 
data products might be generated and released on an intermediate time scale, 
if resources allow.


\subsubsection{Target-of-Opportunity (TOO)}

Options for RDM to process TOO, especially during the first year of Operations 
when the template coverage will be low, are discussed in more detail in 
\citeds{rtn-008}.


\subsection{User-Generated ``Unique and Separate" Data Products}\label{ssec:proc_user}

For science goals that require data products which are not possible to create with 
the original or reconfigured versions of the 
LSST Science Pipelines, and/or for which new algorithmic development or significant 
computational resources are needed, 
user-designed and user-run processing will be needed.

As described above, custom coadds (e.g., weekly, monthly) are also left to users 
to generate, as required by their specific science goals.

Users will have access to the LSST Science Pipelines and data processing 
infrastructure, as well as dedicated computational resources next-to-the-data, 
via the Rubin Science Platform; \citeds{lse-319}.
Details of the planned ``User Batch" facility for data processing are described in 
\citeds{dmtn-202}.
Very computationally intense user processing might require external resources. 

It is expected that some User-Generated pipelines and data products 
might be ``adopted" or ``federated" into the LSST Science Pipelines and the Prompt 
and Data Release data products, however, details regarding the federation of 
user-generated data products are to be provided elsewhere.

\textbf{A note prompt processing and alert production -- }
As the latency on processed visit image availability has an 80-hour
embargo, no user-generated pipeline will be able to process Special Programs 
data on a timescale similar to prompt processing and alert production 
(60 seconds to 24 hours).
No user-generated pipeline may contribute alerts to the LSST alert stream on 
any timescale.


\subsubsection{Deep Drilling Fields}

User-generated data products might include, for example, DDF images coadded on custom 
timescales (e.g., weekly, monthly), or coadded using algorithms outside of the LSST 
Science Pipelines.


\subsubsection{Short-Exposure Mini-Surveys}

Short-exposure images obtained during twilight, which will have a very bright sky 
background unlike other LSST images, might require User-Generated algorithms to 
subtract the high sky background. 

Short-exposure images obtained during the night would have fewer stars for 
astrometric and photometric calibration, and might require user-generated 
processing pipelines.


\subsubsection{Mini-Surveys}

Mini-surveys with time-domain science goals that aren't met by the Prompt pipelines, 
e.g., those that require difference imaging with coadded images on an intermediate 
timescale (e.g., a weekly stack), might require user-generated processing.

Appendix~\ref{sec:spcs} provides more detailed, step-by-step data processing examples 
for some potential Special Programs as further illustration. 


