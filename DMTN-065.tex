\documentclass[DM,lsstdoc,toc]{lsstdoc}
\usepackage{graphicx}
\usepackage{url}
\usepackage{latexsym}
\usepackage{enumitem}

\title[LSST Special Programs]{Data Management for LSST Special Programs}

\author{M.~L.~Graham, Y.~AlSayyad, K.~Bechtol, E.~C.~Bellm, J.~F.~Bosch, J.~L.~Carlin, 
G.~P.~Dubois-Felsmann, L.~P.~Guy, M.~Juri\'{c}, K.-T.~Lim, E.~S.~Rykoff, C.~T.~Slater}

\setDocRef{DMTN-065}
\date{\today}
\setDocUpstreamLocation{\url{https://github.com/lsst-dm/dmtn-065}}

\setDocAbstract{
Special Programs are additional survey areas and/or observing strategies that are driven by specific science 
goals which build on, or are beyond, the core science pillars of the Wide Fast Deep Main Survey.
In order to meet the requirements and enable science related to Special Programs, this document provides 
recommendations for Rubin Data Management regarding processing and serving data products for Special Programs.

\smallskip
Hardware and processing boundaries on the potential diversity of data from Special Programs
are discussed along with scenarios in which user-generated processing 
and data products might be needed to meet Special Programs' science goals.
}

\setDocChangeRecord{%
\addtohist{0}{2017-11-14}{Status: internal working document.}{Melissa Graham}
\addtohist{1}{2018-06-17}{Updated to finalize and issue.}{Melissa Graham}
\addtohist{2}{2021-12-01}{Updates per DM-20375.}{Melissa Graham}
\addtohist{3}{2023-10-19}{Updates per DM-32723.}{Melissa Graham}
}

\begin{document}

\maketitle

% CITATION EXAMPLES
% \verb|\citellp|: \citellp{LPM-17, LSE-30} \\
% \verb|\citell|: (SRD; \citell{LPM-17,LSE-29}) \\
% \verb|\citep[][]|: \citep[e.g.,][are interesting]{LPM-17,LSE-29} \\
% \verb|\cite|: \cite{LPM-17,LSE-29}

\section{Summary} \label{sec:mem}

The three main takeaways for Rubin Data Management regarding Special Programs are:

\begin{enumerate}

\item All visits should have a region label \textit{and} an observing mode
label, and these labels should be available to users as query constraints for
all images and tables.
(Section~\ref{ssec:sci_labels})

\item Special Processing should be done by Rubin Data Management to
produce unique and separate data products for Special Programs in
certain cases that meet specific definitions of "possible and necessary".
(Section~\ref{ssec:sci_sproc})

\item All visits that \emph{can} be processed by the Prompt pipelines and generate
alerts \emph{should} be, in support of time domain and Solar System science goals.
(Section~\ref{ssec:sci_pproc})

\end{enumerate}

\section{Terms and Definitions}\label{sec:intro}

The following is an introduction to the key terms related to Special Programs.

\subsection{Visit Types}

A visit is an observation of a single pointing at a given time, of which
there are three types:

\begin{itemize}
\item Standard Visit -- Composed of $2\times15$ second exposures (commonly referred to as ``snaps").
\item Alternative Standard Visit -- Composed of a single $30$ second exposure.
\item Non-Standard Visit -- Any other exposure time(s) or number of snaps.
\end{itemize}

Non-standard visits with shorter or longer exposure times are being
considered for some Special Programs.

\subsection{Region Label}

Applied to a visit based on its Right Acension and Declination (Section~\ref{ssec:sci_labels}).

\subsection{Observing Mode Label}

Applied to a visit based on the scheduler mode at the time of the observation,
i.e., survey cadence or strategy (Section~\ref{ssec:sci_labels}).

\subsection{Main Survey including Wide-Fast-Deep (WFD)}

The LSST Main Survey includes the low-dust Wide-Fast-Deep (WFD) area 
(which includes low-dust extragalactic and Galactic regions) 
and several special regions:
the dusty areas of the Galactic bulge and plane; the South Celestial Pole (SCP); 
the North Ecliptic Spur (NES); and the Virgo Cluster \citedsp{PSTN-055}.
The WFD component is the core component of the LSST Main Survey, designed to achieve the science 
goals defined by the Science Requirements Document (SRD; \citeds{LPM-17}).

The Main Survey will be executed with a set of observing modes, and 
is expected to use alternative standard visits in the $u$-band and 
standard visits in all other bands \citedsp{PSTN-055}.
Different cadences, filter balances, rotational dithers, etc., 
might be applied in the different regions composing Main Survey.
For example, rolling cadence is expected to be implemented in the low-dust 
WFD region only.

Data from the Main Survey will be processed with standard (non-special) processing (Section~\ref{ssec:intro_stdproc}).
This processing will produce, among other data products (LSE-163), 
a contiguous sky footprint that covers at 
least $\sim$18000 deg$^2$ (with $\gtrsim$800 visits per field in WFD regions), 
and is expected to be accomplished with 85--90\% of the observing time.
This contiguous all-sky "deep coadd" may have region-dependent processing parameters and inputs
(e.g., different calibration parameters or deblending algorithms for crowded fields)
and will be of variable depth, as most special regions recieve fewer visits.

\subsection{Special Programs} 

This is a Rubin Data Management term used to refer to sky regions within
or beyond the Main Survey footprint that have a distinct observing strategy from
the Main Survey's plans for that region (e.g., different visit type, cadence, filter distribution).

Special Programs are typically driven by specific science goals that build on or 
add to the core science pillars of the LSST.
They include LSST components such as the Deep Drilling 
Fields (DDFs) and the mini-, micro-, and nano-surveys \citedsp{PSTN-055}.
About 10--15\% of the total 10-year LSST will be spent obtaining 
observations associated with Special Programs.

Data products from Special Programs is subject to the Rubin Data Policy \citedsp{RDO-013}
in the same way as data products from the Main Survey.

Some science goals for Special Programs can be met with standard processing,
but some will require Special Processing by Rubin Data Management or user-generated processing.

\subsection{Special Processing}

This is a Rubin Data Management term to describe processing that 
uses components of the LSST Science Pipelines and is applied by Rubin
Data Management to images from Special Programs.
Special Processing creates data products that are unique and separate from those produced
by standard processing for the Main Survey.

Special Processing is likely to use different inputs or configurations for the
LSST Science Pipelines, or to run on different timescales, than standard processing - 
as appropriate for the Special Programs' data and science goals.
However, the development and application of specialized \emph{algorithms} or new software
is beyond the scope of Special Processing.

Examples of Special Processing include the processing
of images with non-standard exposure times, images obtained
during twilight, or nightly stacking and differencing
of images in deep drilling fields.
See Section~\ref{ssec:proc_special} for more detailed examples of Special Processing.

\subsection{Standard (non-special) processing}\label{ssec:intro_stdproc}

This is a term used only in this document to refer to the image processing 
described in the Data Management Science Pipelines Design \citeds{LDM-151} that produces the data products described in the Data Products Definitions Document
(\citeds{LSE-163}) that are designed for, and will be applied to, the Main Survey's observations.

In some cases, standard processing is also appropriate for Special Programs.


\subsection{User-Generated Processing}

Any processing of Rubin data done by users in order to reach specific science goals, including
processing for Special Programs data, is referred to as User-Generated Processing.

User-Generated Processing for Special Programs data would be necessary in cases where
the science goals require custom algorithms, software, or very large computational
capacities which are beyond the scope of Special Processing or the Rubin-provided
computational resources (Section~\ref{ssec:sci_comp}).

Guidelines for User-Generated Processing, and for user-generated data products
that can be federated with the Rubin-product data products (i.e., joinable tables),
is forthcoming.

\subsection{Deep Drilling Field (DDF)}

A single pointing for which many (e.g., a hundred) visits are obtained 
(usually sequentially) during a single night, and repeated every few 
nights.

As of the Phase 2 SCOC recommendations in \citeds{PSTN-055}, the five 
confirmed DDFs were:

\begin{itemize}
\item Elias S1 (00:37:48, -44:00:00)
\item XMM-LSS (02:22:50, -04:45:00)
\item Extended Chandra Deep Field-South (03:32:30, -28:06:00)
\item COSMOS (10:00:24, +02:10:55)
\item Euclid Deep Field South  (04:04:58, -48:25:23)\footnote{\url{https://www.cosmos.esa.int/web/euclid/euclid-survey}}
\end{itemize}

Processing for the DDF images is likely to be a combination of standard (Prompt) processing,
Special Processing, and User-Generated Processing.

\subsection{Mini-, Micro- and Nano-Surveys}

Specific sky areas covered by a few hundred, a hundred, or tens of visits (respectively).
This document will refer to them collectively as mini-surveys.

The sky areas of mini-surveys can be within, adjacent to, or detached from the Main Survey footprint.
Mini-surveys can have non-standard visits.
Target-of-opportunity (TOO) observations for, e.g., the discovery of optical counterparts to 
multi-messenger astrophysical phenomena, are considered a type of mini-survey in this document.
For a list of the mini-surveys under consideration,
see \citeds{PSTN-055}.

Processing for the mini-surveys is likely to be a combination of standard (Prompt) processing,
Special Processing, and User-Generated Processing.


\section{Recommendations to Enable Science from Special Programs}\label{sec:sci}

Recommended implementations for Rubin Data Management to meet the 
requirements related to Special Programs (Section~\ref{sec:req})
and enable science with data from Special Programs.

\subsection{Region and Program Labels}\label{ssec:sci_labels}

\textbf{All visits should have a region label \textit{and} a program 
label, and these labels should be propagated to processed images and catalog data.}

\textbf{Region Labels -- } based on Right Acension and Declination.
Region labels should include, e.g., WFD low-dust, Galactic Plane, Galactic 
Bulge, SCP, NES, or Virgo; DDF fieldname; mini-survey region name.
Region labels should be propagated to processed images (visit, 
difference, and deep coadds), catalog sources, and catalog static-sky objects. 

\textbf{Program Labels -- } 
Based on survey cadeance or strategy.
Program labels might include, e.g., WFD, DDF, TOO, mini-survey name, or other 
options like engineering, commissioning, and director's discretionary.
Program labels should be propagated to single-visit processed and 
difference images and catalog sources.
It would not make sense to propagate program labels to deep coadds or 
object tables, as these data products could be a mix of programs.

There are three main motivations for this recommendation.

\begin{enumerate}

\item To meet the requirement that metadata for observations associated 
with Special Programs is stored, and is sufficient for triggering 
real-time data processing recipes (Section~\ref{ssec:req_meta}).

\item To enable users to query and retrieve processed image and catalog
data associated with a specific Special Program, and meet the science goals
of that Special Program, when regular processing as been applied 
(e.g., Prompt Processing, Section~\ref{ssec:sci_pproc}).

\item To enable provenance when Special Programs data in included in regular
processing and the WFD program data products (e.g., if used to improve the
all-sky coadd).

\end{enumerate}

\subsection{Special Processing}\label{ssec:sci_sproc}

\textbf{Special Processing should be done by Rubin Data Management to 
produce unique and separate (and joinable) data products
for Special Programs when it is both possible and necessary.}

\textbf{Possible -- } When original or reconfigured versions of the LSST
Science Pipelines can be used, and no new algorithmic or software development,
or significant additional computational resources, are needed.

\textbf{Necessary -- } When the primary science goal for a Special Program 
cannot be met by including the data in regular processing (e.g., Prompt processing),
or where doing so would compromise the WFD data products (e.g., introduce non-uniformity).

There are two main motivations for this recommendation.

\begin{enumerate}

\item To meet the requirement that Rubin Observatory produce
unique, separate, and joinable data products whenever this is possible 
with the original or reconfigured versions of the LSST Science Pipelines
(Section~\ref{ssec:req_dp}).

\item To enable science with Special Programs by all users, not just those
with the time and effort to process the data, and to reduce computational
load (and potential redundancy) in User-Generated Processing.

\end{enumerate}

The definitions of possible and necessary are further illustrated with examples in Section~\ref{ssec:proc_special}.
Note that secondary science goals may be considered as ``not necessary".

The general scope of Special Processing, including situations in which cross-match and table joins 
would be possible and scientifically relevant, 
are ultimately left to the discretion of Rubin Data Management in Operations.

Further discussion on Special Processing, with examples, is provided in Section~\ref{ssec:proc_special}.

\subsection{Prompt Processing}\label{ssec:sci_pproc}

\textbf{All visits that \emph{can} be processed by the Prompt pipelines and generate 
alerts \emph{should} be, in support of time domain and Solar System science goals.}

The condition ``can be processed" is ultimately left to the discretion of
Rubin Data Management in Rubin Operations, but it is expected to include
all standard and alternative visits in sky regions for which a template image exists.

The main motivation for this recommendation is that all time-domain and 
moving-object science goals (two of the four science pillars for the LSST) are
enhanced by any and all additional observations, even if they are not
optimized for the science (as the WFD program's cadence is being optimized).

Further discussion of regular Prompt processing of Special Programs data, with
examples and a discussion of a few challenges,
is provided in Section~\ref{sssec:proc_reg_prompt}.

\subsection{Computational Resources for User-Generated Processing}\label{ssec:sci_comp}

Rubin Observatory will reserve 10\% of its total data processing capacity for users.
This component would include {\it all} user processing and re-processing of any and 
all LSST data, including Special Programs. 
This is already a requirement, as mentioned in Section~\ref{ssec:req_proc}.

As described in Section~\ref{ssec:proc_user} this processing capacity will be 
accessed via Rubin Science Platform, with a supported software environment and 
infrastructure for batch processing \citedsp{dmtn-202}.

Very computationally intense processing (e.g., shift-and-stack for faint moving 
objects) will likely require external resources\footnote{For more details about 
the boundary between what Rubin Observatory will provide (in terms of data products 
and processing resources) and what will be left to the expertise of the science community, 
see \url{https://www.lsst.org/about/dm/data-products}}.

\subsection{Rubin Science Platform Capabilities}\label{ssec:sci_rsp}

Users will need to be able to query for data that restricts by sky region and 
program label. 
This can be accomplished by including those labels in all image and catalog 
metadata as described in Section~\ref{ssec:sci_labels}, as the TAP service 
and butler already provide the mechanism for user-specified queries.

Users will need to be able to discover Special Programs data products when 
browsing data.
This could be accomplished with, e.g., a toggle to overlay region boundaries
for Special Programs when browsing all-sky coadds built from WFD program data
in the Portal Aspect.




\section{Details and Challenges of Processing Data from Special Programs}\label{sec:proc}

A discussion of the anticipated details and challenges related to 
obtaining and processing data from Special Programs.

Appendix~\ref{sec:spcs} provides detailed examples for the processing
of data from Special Programs, including scenarios in which standard,
Special, and User-Generated Processing are all involved.

\subsection{Boundaries on Non-Standard Visits} \label{ssec:proc_bounds}

Special Programs that do not use standard or alternative standard visits
might be affected by hardware or processing boundaries.

\subsubsection{Hardware Boundaries}\label{ssec:proc_bounds_hardware}

Appendix~\ref{sec:hardbounds} lists all of the hardware boundaries that 
might constrain the potential diversity of Special Programs data.

In general, the currently-proposed Special Programs in \citeds{PSTN-055}
are not anticipated to be limited by hardware boundaries.

A few potential challenges posed by hardware
boundaries are summarized below.

\begin{itemize}

\item \textbf{Short exposures --}
Special Programs that use short exposures would be limited to the
minimum exposure time of 1 second (stretch goal: 0.1 seconds\footnote{A short-exposure survey 
of the bright stars of M67, described in Chapter 10.4 of the 
Observing Strategy White Paper \citep{2017arXiv170804058L}, suggests using the stretch goal of 
0.1 second exposures or, if that is not possible, \textit{"custom pixel masks to accurately perform 
photometry on stars as much as 6 magnitudes brighter than the saturation level"}. 
This may require User-Generated Processing.}).
There is a potential hardware boundary that limits the readout rate to 1 
every 15 seconds, which would affect the image acquisition rate and 
increase the overheads on short exposures.

\item \textbf{Repeated pointing --}
Special Programs that require the \emph{exact same} field pointing and 
rotation for \emph{every exposure} (to sub-arcsecond levels) might run 
into hardware boundaries on pointing and tracking.
Atmospheric distortions also pose a major limitation on positional
repeatability across the focal plane.

\item \textbf{Twilight images --}
Special Programs that obtain twilight images will be subject to safe
limits on sky background flux, as with any astronomical camera.

\end{itemize}

Finally, as a side note, Special Programs that request a high number of 
filter changes and/or long slews could be inefficient due to large overheads,
but would not be limited by hardware boundaries.

\subsubsection{Processing Boundaries}\label{ssec:proc_bounds_processing}

Appendix~\ref{sec:procbounds} describes the boundaries on what types of visits 
can be processed and calibrated by the Data Management System and the LSST
Science Pipelines.

Most of the currently-proposed Special Programs in \citeds{PSTN-055}
are not anticipated to be limited by hardware boundaries.
However, those which use non-standard visits, especially those with
short exposure time or those obtained during twilight, might
be affected by processing boundaries.

The most likely challenges posed by processing
boundaries are summarized below.

\begin{itemize}

\item \textbf{Very short exposures --}
Special Programs that use very short ($<$2 sec) exposures 
could be difficult to reduce due to an incompletely-formed PSF 
(Section~\ref{ssec:procbounds_expt}).
The Data Management System is required to be able to process exposure 
times as low as 1 second (Section~\ref{ssec:req_proc}), 
but it is known that such short exposures might have degraded image quality.

\item \textbf{Very short or very long exposures --}
Special Programs that use very short or very long ($>$150 sec) 
exposures could be difficult to calibrate due to having too few 
(or too few unsaturated) stars.

\item \textbf{Twilight images --}
For Special Programs that obtain images with very bright sky backgrounds
(twilight images), it is currently unclear whether they can be processed
with the LSST Science Pipelines; 
User-Generated Processing might be needed (e.g., \citealt{2022AJ....164..168S}).

\item \textbf{Streaked images --}
The full reduction and calibration of data from any Special Programs that 
use non-sidereal tracking, which produce images with star streaks, is
currently beyond the scope of the LSST Science Pipelines; 
User-Generated Processing would be needed.

\end{itemize}

\subsection{Standard (non-special) processing}\label{ssec:proc_reg}

Recall that the term ``standard processing" is used only within this document,
and refers to the image processing pipelines that are designed for, 
and will be applied to, the Main Survey's observations (Section~\ref{sec:intro}).

Decisions about when to apply standard processing to Special Programs data,
or when to include it in the data products for the Main Survey
(e.g., if it improves the all-sky coadds), 
are ultimately left to the discretion of the Rubin Observatory's 
Data Management and System Performance departments in Rubin Operations.


\subsubsection{Prompt Processing and Alert Production}\label{sssec:proc_reg_prompt}

As described in Section~\ref{ssec:sci_pproc}, 
all visits that \emph{can} be processed by the Prompt pipelines and generate 
alerts \emph{should} be, in support of time domain and Solar System science goals.

\textbf{The meaning of ``can be processed".}
This is ultimately left to the discretion of
Rubin Data Management in Rubin Operations, but is expected to include
all standard and alternative visits in sky regions for which a template image exists.
This might also include some non-standard visits (shorter or longer exposures), 
as long as they can be processed by the Prompt pipeline and an appropriate template image exists.
Visits with very short or very long exposure times (or very bright sky 
backgrounds) might be excluded if they would need specialized algorithms for,
e.g., instrument signature removal, difference-imaging, template-generation 
(Section~\ref{ssec:proc_bounds}).

If, for example, Prompt processing takes longer on Special Programs data or short exposures demand
additional processing because the image acquisition rate exceeds the nominal $\sim$2 per minute,
the available compute capacity of Prompt processing might be saturated.
This could compromise the latency of alert distribution while Special Programs are being executed,
and could also impact alert distribution for subsequent Main Survey visits.
It is left to the discretion of Rubin Data Management to prioritize alert distribution
for Main Survey visits by potentially limiting Prompt processing for Special Programs in such cases.

\textbf{The use of specialized (alternative) template images.}
If a Special Program's primary science goal requires specialized templates and
Prompt processing, the Data Management System will have the capability to load
and use an alternative template for some sky regions, based on the image metadata
(i.e., the labels described in Section~\ref{ssec:sci_labels}).
Potential processes for the creation and verification of such alternative
templates, the cost-benefit of allocating disk storage space to hold
alternative templates, and decisions about when a science goal requires specialized
templates, are left to the discretion of Rubin Data Management.

\textbf{How Main Survey and Special Program data would co-exist.}
No ``unique and separate" data products for the Special Progams would be 
produced by standard Prompt processing.
Special Programs data that is processed by the Prompt pipeline would 
contribute to the Prompt data products for the Main Survey as 
described in Section 3 of the DPDD \citedsp{LSE-163}. 
These data products are the results of Difference Image Analysis (DIA),
such as the difference images, catalogs of sources detected in difference
images ({\tt DiaSources}) and associated static-sky {\tt DiaSources}
into {\tt DiaObjects}, and alert packets.
Including visits from Special Programs in standard Prompt processing alongside
vists from the Main Survey is not, in general, anticipated to affect Main Survey 
or WFD-specific science goals.
For example, analyses for a WFD-only subset could still be done using the observing mode
and region labels described in Section~\ref{ssec:sci_labels}, which would be
propogated to difference images, difference-image catalogss, and alerts.

\textbf{Potential issues with Prompt processing for untiled sequences from Special Programs.}
There are two potential issues with Prompt processing for DDFs, or any mini-survey 
that obtains a sequence of untiled images (images at the same pointing or which overlap).
For example, a DDF which obtains a hour-long series of about a hundred images at the same coordinates,
every few nights for a few months.

\begin{enumerate}

\item \textbf{DIA Object histories may become too large for the sizing model.}
Alert packets contain the full records of all associated 
{\tt DiaSources} from the past 12 months, but the alert
stream bandwidth is sized for the expected histories for
Main Survey fields. 
The Prompt pipeline resources are also sized for the
Main Survey, and it might not be possible to load up
thousands of epochs at a time.
The Data Management team will have to test the realized
capabilities of Prompt processing and alert distribtuion,
and potentially impose a mitigation strategy such as
limiting histories to the last $N$ observations instead
of the last 12 months in heavily-observed regions.
Alerts from Special Programs data are subject to the same latency
requirements as Main Survey alerts (Section~\ref{ssec:req_proc}).

\item \textbf{The association of new DIA Sources into DIA Objects may be compromised.}
For consecutive images, processing for the second image begins when the processing for the 
first image is only halfway done.
At this point, the {\tt DiaObject} catalog has 
not yet been updated with the new {\tt DiaSources} detected in the first image.
Thus, the {\tt DiaSource}s from images one and two for a new transient 
would not be associated with a single {\tt DiaObject}, but instead would 
each instantiate a new {\tt DiaObject}.

\end{enumerate}

These two potential issues pose challenges, but are not necessarily showstoppers 
in processing Special Programs data with the Prompt pipelines. 
The overall impact on time-domain science would still be positive, even 
if mitigations are needed for these issues.
For example, brokers and users would be able to use the region and observing mode 
labels in the data as context (i.e., as flags) and avoid including 
limited-history or potentially-compromised {\tt DiaSources} in their
analyses if necessary.

\subsubsection{Solar System Processing}\label{sssec:proc_reg_ss}

Since Solar System Processing takes {\tt DiaSource}s as input, any 
Special Programs images that are processed by the Prompt pipeline
could be incorporated into Solar System Processing.

\subsubsection{Data Release Processing}\label{sssec:proc_reg_dr}

\textbf{Time-domain DIA data products --}
This includes the results of the annual reprocessing of Main Survey data with 
Difference Image Analysis (DIA), and the production of Data Release
versions of the processed images (single-visit and difference images)
and associated catalogs ({\tt DiaSource}, {\tt DiaObject}, {\tt Source},
{\tt ForcedSource}, {\tt DiaForcedSource}, and so on).
These data products will be used primarily for time-domain science.
They should include Special Programs data for the same reasons as
provided in Section~\ref{ssec:sci_pproc}, and with the same 
considerations as discussed in Section~\ref{sssec:proc_reg_prompt}.
This processing should use the same template image for a given field.

\textbf{Static-sky data products --}
This includes the tessellation and coaddition of Main Survey images
and the associated multi-band {\tt Object} catalog and survey property maps.
Whether and how to include any Special Programs data in these data products
is left entirely to the discretion of the Rubin Data Management team in Operations.
As an example, perhaps Special Programs images will only be included when they 
assist with uniformity or suppress edge effects or low-order modes in the 
all-sky photometric solutions.

\subsection{Special Processing}\label{ssec:proc_special}

As described in Section~\ref{ssec:sci_sproc}, 
Special Processing should be done by Rubin Data Management to 
produce unique and separate data products
for Special Programs when it is both possible and necessary.
This is a requirement (Section~\ref{ssec:req_dp}).

In short, \emph{possible} means that original or reconfigured versions of the LSST
Science Pipelines can be used , and \emph{necessary} means the primary science goal for Special Program
could not be met without the data products produced by the Special Processing.

\textbf{Joinable tables --}
Any tables for the unique and separate data products should be joinable to the 
data products for the Main Survey, when possible.
This is a requirement (Section~\ref{ssec:req_dp}).

\textbf{Survey property maps --}
As for the tesslated all-sky coadded images made from Main Survey data, survey property maps
should be made individually as part of any Special Programs that generates
tesselated coadded images. This is \emph{not} a requirement (Section~\ref{ssec:req_dp}).

\textbf{Special Processing timescales --}
The timescales for Special Processing should be adopted that best serve the
primary science goal for the Special Program.
For example, nighly-coadd difference-image analysis for DDFs is the most
scientifically useful if done on a daily cadence, but the deeply coadded
images for DDF fields could be released annually.

\subsubsection{Scenarios to illustrate ``possible and necessary"}

The interpretations of possible and necessary, and the scope of Special Processing,
are ultimately left to the discretion of Rubin Data Management in Operations.

Specific examples are provided only to illustrate this interpretation of possible and necessary.
These examples do not place limits on what Special Processing might be done.

\begin{itemize}

\item \textbf{Possible and necessary:}
in order to detect high-redshift (faint) galaxies in the DDFs,
Rubin Data Management uses the LSST Science Pipelines to deeply
coadd images for each field, and store the results of source
detection and characterization in unique and separate tables that
are included in the annual data release.

\item \textbf{Possible but not necessary:}
a time-domain mini-survey that uses standard visits \emph{could}
have separate difference-image analysis object and source catalogs
generated, but this is not necessary as the science goals for the
mini-survey can be met by processing its data with standard Prompt
Processing, and ensuring the data is properly labeled.

\item \textbf{Possible but not necessary (secondary science goals):}
a time-domain mini-survey (or DDF) has a secondary science goal of detecting
precursor outbursts for transients, which requires coadding images
within windows of days, weeks, and months to reach various depths.
This set of custom coadds may be considered as ``not necessary" and requiring 
User-Generated Processing.
Note that for the Main Survey data, such custom coadds are also considered 
as beyond scope and in need of User-Generated Processing.

\item \textbf{Necessary but not possible:}
in order to find the most distant, faint Kuiper Belt Objects in the DDF,
a specialized, computationally intensive form of ``shift-and-stack" processing
is required for detection, but such algorithms are not used by the 
LSST Science Pipelines and so User-Generated Processing will be needed.
A second example is a twilight survey that uses non-standard visits 
which are outside the boundaries of what the LSST Science Pipelines can process.

\end{itemize}

Further examples of potential Special Processing for anticipated Special Programs are provided below.

\subsubsection{Deep Drilling Fields (DDFs)}\label{sssec:proc_special_ddf}

As the DDFs will likely be observed with standard or alternative standard 
visits, Data Management will be able to reconfigure existing pipelines for
Special Processing to produce unique and separate DDF data products.

For example, Special Processing for the DDF data products might include:
\begin{itemize}
\item nightly-coadded images
\item nightly-coadded difference images
\item {\tt DiaSource}- and {\tt DiaObject}-like catalogs for the nightly-coadds
\item deeply-coadded images
\item templates for the nightly-coadded difference images
\item {\tt Source}- and {\tt Object}-like catalogs for the nightly-coadded and deeply-coadded images
\end{itemize}

The release of nightly-coadded or difference images is subject to the 80-hour embargo.
The timescales for updating catalogs based on the nightly-coadds could be daily
or on another intermediate timescale; this is left to the discretion of Rubin Data Management.


\subsubsection{Short-Exposure Mini-Surveys}

As described in \citedsp{PSTN-055}, there are a few
short-exposure mini-surveys are under consideration.
Two examples are a short exposure map of the sky in $ugrizy$ for calibration,
and a Near-Earth Objects (NEO) twilight survey.

Special Processing for short-exposure mini-surveys remains to
be determined.
Since it falls under the remit of Data Management to perform proper calibration,
an evaluatation of whether short exposures for calibration are necessary
will be done. 
If Data Management does find that short-exposure and/or high-sky brightness images
can be processed with reconfigured versions of the LSST Science Pipelines,
then unique and separate data products could be generated with Special Processing.
For the NEO twilight survey, these data products would likely be similar to the
Prompt or DIA data products, but note the potential issues with alert
latency from short exposures mentioned in Section~\ref{sssec:proc_reg_prompt}.
For the calibration survey, these data products would like be similar to the
annual data release tessellated coadds and associated catalog.

\subsubsection{Standard Visit Mini-Surveys}

Consider a Special Program in which a special region of sky is observed with standard
visits but a a special strategy or cadence which is significantly distinct from the Main Survey
observing modes, and lasts for a limited amount of time.
For example, a short-term survey of the Magellanic Clouds, with dual primary science
goals in time-domain and static-sky science.

In cases like this, a set of unique and separate data products with the same formats as the 
time-domain DIA and static-sky data products described in Section~\ref{sssec:proc_reg_dr}
should be created with Special Processing.
They might be released with an annual data release or on an intermediate timescale, e.g.,
within six months of the conclusion of the mini-survey observations.

\subsubsection{Target-of-Opportunity (TOO) Observations}

Options for Data Management to process TOO observations, especially during the first year of Operations 
when the template coverage will be low, are discussed in more detail in 
\citeds{rtn-008}.

\subsection{User-Generated Processing}\label{ssec:proc_user}

Science goals that require data products which are not possible to create with 
the original or reconfigured versions of the LSST Science Pipelines, 
and/or for which new algorithmic development or significant computational resources 
are needed, will require user processing and user-generated data products.
As described above, custom coadds (e.g., weekly, monthly) are also left to users 
to generate, as required by their specific science goals.

\textbf{Computational resources -- }
Users will have access to the LSST Science Pipelines and data processing 
infrastructure, as well as dedicated computational resources next-to-the-data, 
via the Rubin Science Platform; \citeds{lse-319}.
Details of the planned ``User Batch" facility for data processing are described in 
\citeds{dmtn-202}.
Very computationally intense user processing might require external resources. 

\textbf{Adopting user code or data products --}
It is expected that some User-Generated pipelines and data products 
might be ``adopted" or ``federated" into the LSST Science Pipelines and the Prompt 
and Data Release data products.
Details regarding this are to be provided elsewhere.

\textbf{Alert production is restricted -- }
User-Generated Processing will not be able to release alert packets in the LSST alert stream.
As the latency on processed visit image availability has an 80-hour
embargo, no user-generated pipeline will be able to process Special Programs 
data on a timescale similar to prompt processing and alert production 
(60 seconds to 24 hours).
Thus, no User-Generated Processing may contribute alerts to the LSST alert stream on 
any timescale.

Further examples of potential User-Generated Processing for anticipated Special Programs are provided below.

\subsubsection{Deep Drilling Fields}

User-Generated Processing and data products might include, for example, DDF images coadded 
on custom timescales (e.g., weekly, monthly), or coadded using algorithms outside of the 
LSST Science Pipelines.

\subsubsection{Short-Exposure Mini-Surveys}

Short-exposure images obtained during twilight, which will have a very bright sky 
background unlike other LSST images, might require specialized algorithms
to subtract the high sky background which are not available in the LSST Science Pipelines,
and might require User-Generated Processing.

Short-exposure images obtained during the night might have too-few stars to satisfy the
astrometric and photometric calibration routines in the LSST Science Pipelines,
and might require User-Generated Processing.

\subsubsection{Mini-Surveys}

Mini-surveys with time-domain science goals that aren't met by the Prompt pipelines, 
e.g., those that require difference imaging with coadded images on an intermediate 
timescale (e.g., a weekly stack), would require User-Generated Processing.


\section{Requirements Related to Special Programs}\label{sec:req}
% \lsrreq, \ossreq, \dmreq
% \reqparam

Appendix~\ref{sec:docrev} provides a full description of the requirements 
on the Data Management System related to Special Programs, which are summarized here.

\subsection{Metadata}\label{ssec:req_meta}

\textbf{Program metadata that is sufficient to trigger Special Processing should be stored. -- }
In order to support Special Programs processing, the LSST system is 
required to store metadata that includes program information for every raw 
image, such as identifiers for images obtained as part of the Main Survey 
or a Special Program (DMS-REQ-0068).
It is required that this metadata be sufficient for Special Programs to 
trigger their own real-time data processing recipes ``whenever possible" 
(DMS-REQ-0320), and be included in alert packets (DMS-REQ-0274).

\subsection{Data Products}\label{ssec:req_dp}

\textbf{Produce unique and special (and joinable) data products when possible. --}
Rubin Observatory and the LSST system (the observatory and the data 
management systems) are required to process Special Programs data to 
produce unique and separate data products ``whenever possible" 
(LSR-REQ-0121).
It is a requirement that these Special Programs data products be distinct 
and joinable with the Prompt and/or Data Release data products (DMS-REQ-0322).

The term ``whenever possible" includes cases where the original or 
reconfigured versions of the LSST Science Pipelines can be run, and 
excludes cases where the development of new algorithms or the allocation 
of significant additional computational resources are required 
(LSR-REQ-0121).

The statement ``to produce unique and separate data products" typically 
refers to producing the same kinds of data products as will be generated 
by the Prompt and Data Release pipelines (processed visit images, coadded 
images, difference images, and catalogs of sources and objects for those 
images).

The term ``joinable" means the Special Programs data products can be federated 
or cross-matched with the relevant Prompt and Data Release data products, and that
a column of cross-matched object identifiers is provided to enable table joins.

\textbf{The size of Special Programs data products should be about 10\% of the total. --}
It is a requirement that the cumulative size of the Special Programs data 
products generated by Rubin Observatory be no more than $\sim$10\% the 
size of the Data Release data products (LSR-REQ-0121).

The spirit of this requirement on data volume is that the size be proportional to the fraction 
of survey time spent on Special Programs.

The derivation of value-added data products, such as HiPS or MOC maps, for 
Special Programs remains an open question (DMS-REQ-0379, 0383), and is not required.
However, they are suggested to be produced with Special Processing where
appropriate (Section~\ref{ssec:proc_special}).

\subsection{Processing}\label{ssec:req_proc}

\textbf{Latency requirements apply to alerts from Special Programs images. --}
It is a requirement that any Special Programs processing done with the 
Prompt pipeline is subject to the same 
timescales and latency constraints of 24 hours for the release of Prompt 
data products and 1 minute for the transmission of Alert packets 
(DMS-REQ-0344).

\textbf{Intermediate timescales should be used to enable science when needed. --}
It is also a requirement that Special Programs processing be done on 
timescales intermediate to the Prompt and Data Release processing, 
``whenever possible" and whenever necessary to enable the intended science 
goals of the Special Program (LSR-REQ-0032).

\textbf{Exposure times of 1 second should be processable. -- }
It is a requirement that the LSST system be able to process non-standard 
visits with short exposure times as low as 1 second, with a discussion 
note that such short exposures might have degraded image quality 
(LSR-REQ-0111).

It is not a requirement, but processing for Special Programs by Rubin Observatory is expected to use no 
more than $\sim$10\% of computational and storage capacity of the Rubin data processing cluster 
(i.e., proportional to the fraction of survey time spent; Section 6 of LSE-163).

The 10\% of the total data processing capacity that Rubin Observatory is 
required to reserve for \emph{all} User-Generated Processing includes that 
applied by users to Special Programs data.
There is no additional capacity {\it for users} that will be reserved only for Special 
Programs data (LSR-REQ-0041).

This is a minor detail but, there are two different "10\%" values quoted here.
The accounting order is not in the requirements but Rubin Data Management considers
the 10\% for User-Generated Processing as coming off of the total capacity first,
with 90\% allocated for Rubin processing.
Then the 10\% for Rubin Special Processing comes from that 90\%,
and so is actually 9\% of the total capacity.


% % % % % % % % % % % % % % % % % %
\bibliography{local,lsst,refs,books,refs_ads}

% % % % % % % % % % % % % % % % % %
\appendix

\section{Hypothetical Scenarios}\label{sec:spcs}

\textbf{\emph{Hypothetical}} examples of Special Programs
and the standard, Special, and User-Generated processing
to illustrate what \emph{might be done}.

The details of the data acquisition and processing mentioned below are 
\emph{just illustrative examples} of decisions that have yet to be made.

The steps used to describe the hypothetical processing for each case scenario are: \\
Step 1. Data acquisition. \\
Step 2. Standard Prompt processing and Alert Production. \\
Step 3. Special Processing with reconfigured pipelines. \\
Step 4. Standard processing for inclusion in Main Survey data products. \\
Step 5. User-Generated Processing. \\

\subsection{Outer solar system mini-survey}\label{ssec:SPCS_TNO}

This hypothetical Special Programs processing summary is based on the Becker et al. (2011) 
white paper to find outer solar system objects with shift-and stack (SAS) \citedsp{Document-11013}.

Step 1. Data acquisition. \\
In a single night, the 9 adjacent fields in a 3x3 grid are observed with 
$336$ $\times$ $15$ second $r$ or $g$-band exposures (168 standard visits). 
These observations are repeated 2-3 nights later, and then this 2-night sequence
is repeated 3 more times: 1.5 months, 3 months, and 13.5 months later. 
They are not all at the same RA, Dec, but at selected ecliptic coordinates.

Step 2. Standard Prompt processing and Alert Production. \\
Each $2\times15$ second standard visit is processed by the Prompt pipeline 
and alerts are released within 60 seconds.
Within 24 hours, the {\tt DiaSource} and {\tt DiaObject} catalogs are updated
to include the results of Prompt processing of these visits.
After 80 hours, the processed visit images and difference images become available.
All images and sources originating from this Special Program have 
region and observing mode labels, e.g., ``SP-OSSO".

The results of Prompt processing are not very relevant for this Special Program's primary science goal,
which requires a year of dispersed observations before the processing pipelines for shift-and-stack can be run.
However, including these data in Prompt processing means that
they can contribute to LSST's other time-domain and Solar System science goals.

Step 3. Special Processing with reconfigured pipelines. \\
None possible. 
Shift-and-stack processing is beyond the scope of existing algorithms in the LSST Science Pipelines.

Step 4. Standard processing for inclusion in Main Survey data products. \\
Every year, each $2\times15$ second standard visit is reprocessed by the DIA data release pipelines
and the results are included alongside Main Survey data in the relevant DIA data products
(e.g., processed visit images, difference images, associated source catalogs).
In the first year after the Special Program is executed,
Rubin Data Management finds that 10\% of the standard visits from this Special Program
had coordinates and image quality that help improve uniformity of the all-sky coadd,
and so they are included.
In later years, this fraction decreases (remember, this is \emph{hypothetical}).
In all data releases, any and all processed images and catalog sources that originate in visits from this Special Program
have the same region and observing mode labels, e.g., ``SP-OSSO".

Step 5. User-Generated Processing. \\
The User-Generated Processing pipeline running the shift-and-stack processing is be set up and submitted 
for batch processing by the user through the Science Platform or on an external system. 
The pipeline's inputs are the processed visit images (and/or difference images) from Prompt processing.
User-generated custom algorithms then shift-and-stack the images, and then the LSST Science Pipelines
tasks are used to do source detection and characterization and create catalogs.
User-generated custom code derives orbital parameters for the detections, and stores
them in a user-generated catalog with a similar format to {\tt SSObjects}.


\subsection{Deep Drilling Field}\label{ssec:SPCS_SNDDF}

Step 1. Data acquisition. \\
In the COSMOS DDF, the scheduler obtains 10 standard visits in a row in each of the $griz$ filters
with a small dither pattern between visits.
This happens every other night for a three month season for four years.

Step 2. Standard Prompt processing and Alert Production. \\
Same as above, but the observing mode label might be, e.g., ``SP-DDF-COSMOS".

Step 3. Special Processing with reconfigured pipelines. \\
First, a template image of appropriate depth for ``nightly" difference imaging is created.
At the end of each nightly sequence of observations, a pipeline based on recongfigured 
components of the LSST Science Pipelines could be automatically triggered.
This pipeline creates nightly coadds in each filter and runs DIA using the template.
Unique and separate catalogs with the same format
as {\tt DiaObject} and {\tt DiaSource} could be updated within, e.g., 24 hours (not a requirement),
and alert packets could be created and produced (with a flag set to identify them
as longer-latency alerts from nightly stacked DDF images).
Instead of running daily, this processing might run weekly instead (TBD).
At the end of each season, deeply coadded images that include all of the DDF's visits 
from all years are re-generated, along with a separate {\tt Object}-like catalog.
All images and catalogs are stored in separate butler collections and TAP tables from
the Main Survey data products.

Step 4. Standard processing for inclusion in Main Survey data products. \\
Every year, each standard visit is reprocessed by the DIA data release pipelines
and the results are included alongside Main Survey data in the relevant DIA data products
(e.g., processed visit images, difference images, associated source catalogs).
Due to their small dither and lack of rotation, not even a single DDF image 
is used to supplement the Main Survey's all-sky coadd.

Step 5. User-Generated Processing. \\
In order to achieve a secondary science goal of finding very high-$z$ faint supernovae,
a team of users reconfigure the LSST Science Pipelines to create weekly deep coadds
of the COSMOS field an appropriate-depth template image, and to run DIA at the
end of the season.
These data products are stored in separate catalogs with the same format and schema as
the {\tt DiaSource} and {\tt DiaObject} tables that are private to the team.


\subsection{Short-exposure twilight survey}\label{ssec:SPCS_Twilight}

Twilight observations obtained at, e.g., 60 degrees from the Sun, are particularly
well-suited for finding Near-Earth Objects (NEOs).

Step 1. Data acquisition. \\
At a specified time (or e.g., 6 degree twilight), the scheduler begins a dither pattern of 
$2$-second exposures. 
Coordinates and exposure times are set by the Sun distance, sky brightness, and desired saturation limits.

Step 2. Standard Prompt processing and Alert Production. \\
Pending studies of DIA and Alert Production pipeline capabilities to process 
short-exposure, high sky-background images (see Section~\ref{ssec:proc_bounds_processing}).

Step 3. Special Processing with reconfigured pipelines. \\
Pending studies of the LSST Science Pipelines capabilities to process 
short-exposure, high sky-background images (see Section~\ref{ssec:proc_bounds_processing}).

Step 4. Standard processing for inclusion in Main Survey data products. \\
These short-exposure, high sky background images would not contribute to the data products created for the Main Survey.

Step 5. User-Generated Processing. \\
If short-exposure images cannot be processed with the existing DM algorithms, 
user-generated processing would be needed to reduce the raw data, and to 
futher detect and characterize sources in the processed images.





\section{Potential Hardware Boundaries on Data Diversity}\label{sec:hardbounds}

The potential boundaries on the diversity of data products that could be imposed by limitations from the Rubin Observatory hardware -- camera, telescope, and/or site -- are considered.

\subsection{Filter Changes}
The maximum time for filter change is 120 seconds: 30 seconds for the telescope to reorient the camera to its nominal zero angle position on the rotator, and 90 seconds to the camera subsystem for executing the change (OSS-REQ-0293; \citeds{LSE-30}).
Assuming that most Special Programs would be designed to keep overheads $<$100\% and would be using standard 30 second visits, the filter change time indicates that it is likely that at least 4 exposures in a given filter would be obtained between filter changes, but this is not actually a hardware boundary. 
The filter change mechanism is designed to undergo a total of 100000 changes over its lifetime, and each filter is designed to support up to 30000 changes over its lifetime, where lifetime is 15 years.
That is an average of $\sim$27 changes per day, some of which would occur in they day during calibrations (estimate, $\sim$10) and the rest at night.
As stated in the filter change memorandum (\url{ls.st/spt-494}), {\it ``the system could support as many changes involving the 5 filters loaded in the carousel as desired, without any practical limitation"}.

\subsection{Filter Carousel Loads}
The filter carousel can hold five of the six LSST filters at a time.
The system is designed to support $3000$ loads in $15$ years (\url{ls.st/spt-494}).
Filter loads are only done in the day, and there will never be data in more than five filters in a given night.

\subsection{Exposure Times}
The minimum exposure time is $1$ second, with a stretch goal of $0.1$ seconds (OSS-REQ-0291; \citeds{LSE-30}).
The maximum exposure time is not restricted.

\subsection{Readout Time}
The readout time is $2$ seconds, and would be significant overhead on short exposures.

\subsection{Inter-Image Time}
Images with exposure times $<15$ seconds {\it might} still have to be separated by $15$ seconds for thermal tolerance; i.e., the minimum readout rate might be one image every $15$ seconds, regardless of exposure time (OSS-REQ-0291; \citeds{LSE-30}).

As discussed in Jira ticket DM-12573, the main issue is thermal and is related to the shutter, both the motors and the brakes; an elevated Camera skin temperature would affect image quality.
As of 2022, early tests suggest that a sustained (30 minutes) sequence that increases the heat load by large factors would not work, but further functional testing of the system once the Camera was fully assembled are needed for full characterization of the issue.

This potential $15$ interval between images is also a potential hardware boundary on the potential diversity of data products.


\subsection{Telescope Slew}
As described in \citeds{Document-28382}, large slews would have considerable overheads, but there are no hardware boundaries on the size of a single slew or the accrued slew distance.

\subsection{Pointing and Sidereal Tracking}
The specifications for the telescope's pointing and tracking in \citeds{LSE-30} indicate that $<$0.2 arcsecond precision in field pointing (OSS-REQ-0302) and $<$1 arcsecond in open-loop tracking (OSS-REQ-0303) would not be possible, but guiding would improve the latter (OSS-REQ-0305).
Furthermore, obtaining the \emph{exact same} alignment of the pixel grid in RA-Dec {\it ``would put demands on the camera rotator that were not planned"}\footnote{As per C. Claver's comments in ticket DM-12573.}.

\subsection{Non-Sidereal Tracking}
The requirement that the LSST system be able to perform non-sidereal tracking is set by OSS-REQ-0380 in \citeds{LSE-30}.
This capability will include angular rates of up to 220 arcseconds per second in both azimuth and elevation. 

\subsection{Camera Rotation}
The requirements on the rotator's capabilities do not set any limits on the per-night or total lifetime rotation (OSS-REQ-0301, -0300; \citeds{LSE-30}) which might put boundaries on the distance between successive visits or the ability to jump between two widely separated fields.
Currently, there are no hardware boundaries imposed by camera rotation constraints on the potential diversity of data products.


\section{Potential Processing Boundaries on Data Diversity}\label{sec:procbounds}

The capability of the LSST Science Pipelines to process diverse data is explored below.

Note that processing boundaries might ultimately be defined not by what is technically possible, but by the resulting image quality parameters (e.g., the number of stars with sufficient flux for photometric calibration).
Furthermore, the processing boundaries might not be fully constrained until the final performance of the LSST Science Pipelines, as described in the Data Management Applications Design, \citeds{LDM-151}) document, is fully characterized.

{\bf Summary of the conclusions below:}\\
Very short ($<$2 sec) exposures could be difficult to reduce due to an incompletely-formed PSF, and very short or very long ($>$150 sec) exposures could be difficult to calibrate due to having too few (or too few unsaturated) stars.
It is currently unclear whether images with very bright sky backgrounds (twilight images) can be processed with the LSST Science Pipelines, or whether user generated pipelines will be needed.
The full reduction and calibration of images obtained with non-sidereal tracking, in which the stars are streaked, is currently beyond the scope of the LSST Science Pipelines, and will require a user generated pipeline.

\subsection{Exposure Times}\label{ssec:procbounds_expt}

Images which deviate significantly from the $15$ second duration for the WFD main survey may encounter issues in the instrument signature removal routine, in the correction for differential chromatic refraction, in the difference imaging analysis pipeline, and/or in the photometric and astrometric calibrations due to a differently sampled set of standard stars per CCD.

\subsubsection{Short Exposures (Non-Standard Visits of $<$30 sec)}
The LSST System Requirements document states that {\it ``The LSST shall be capable of obtaining and processing exposures not taken in a standard visit mode including those with a minimum exposure time of} {\tt minExpTime}", which is 1 second (stretch goal 0.1 seconds; LSR-REQ-0111 in \citeds{LSE-29}).

However, for exposure times there are other considerations, as changing the exposure time also affects the photometric and astrometric calibrations.
Assuming that 1 second exposure can be reduced and calibrated, its detected point sources will span a dynamic range of $r$$\approx$ 13--21 magnitudes.
A template image built on 15 second exposures will saturate at $r$$\approx$15.8 mag, but this still leaves stars between 15.8--21.0 magnitudes to be used in the PSF-matching (and all other filters have a similarly large overlap).

In order for an image to be successfully PSF-matched to the template, the PSF must be well formed (no speckle pattern), and have a spatial variance that the pipeline is capable of modeling (be smoothly varying on some minimal scale).
As a simple demonstration, Figure \ref{fig:expt} shows that perhaps exposure times shorter than $2$ seconds do not have a well-formed PSF (using the centroid of a 2D Gaussian fit as a proxy for "well-formed").

\begin{figure}
\begin{center}
\includegraphics[width=14cm,trim={0cm 0cm 0cm 0cm}, clip]{figures/exptime.png}
\caption{At left, Arroyo atmosphere-only simulated PSF for LSST (with oversampled pixels) with exposure times of 0.5, 2, and 15 seconds (top to bottom), courtesy of Bo Xin. At right, blue and purple lines show the location of the centroid derived from a 2D Gaussian fit to the PSF as a function of exposure time, with the red dashed line showing the true center. We can see that for exposure times greater than 2 seconds, the centroid converges near its true value. \label{fig:expt}}
\end{center}
\end{figure}

% In conversation with DM-AP team members (Reiss, Findeisen, Connolly, Bo) there has not yet been a study of the safe range of exposure times that will be allowed to contribute Alert Production.
% One possibly useful study is Chang et al. (2012), "Atmospheric point spread function interpolation for weak lensing in short exposure imaging data".
% They show that a 15 second exposure contains PSF variability on short spatial scales across a 1 square degree image which, for extragalactic fields with few stars (i.e., but good for weak lensing), is hard to characterize.
% They also present a new software package to do mitigate the effects.
% Software packages \texttt{PhoSim} (Peterson et al. 2015; \citep{2015ApJS..218...14P}) or \texttt{ARROYO} \citep{2004SPIE.5497..290B} could be used to characterize the PSF stability as a function of exposure time.

\subsubsection{Long Exposures (Non-Standard Visits of $>$30 sec)}

There is no maximum exposure time specified for an LSST image.
Given that the template image will be a stack of at least a year or two of data, processing a $5$--$10$ times deeper single image through the difference imaging pipeline should be fine.
However, a $2\times150$ second exposure would saturate at $r \approx 18.3$, perhaps leaving too few stars overlapping with e.g., templates or WFD images, for astrometric and photometric calibrations.
Furthermore, cosmic-ray rejection completeness might be reduced for longer exposures (unknown), which could impact the quality of a difference image and the detected sources.
Additionally, any system qualities that vary on short (but $>30$ second) timescales could inhibit photometric calibration (e.g., tracking).

\subsection{Twilight Images with a Bright Background}

Images obtained during twilight for scientific purposes are also likely to have shorter exposure times, and so the issues described in Section \ref{ssec:procbounds_expt} also apply here.
Whether or not bright-background images can (or shall) be fully processed -- reduced, calibrated, background-subtracted, and delivered with astrometric and photometric solutions -- or whether this will require a user generated pipeline, remains to be determined (see also the example in Section \ref{ssec:SPCS_Twilight}).
This may depend on the exposure time and the number of stars available in the image.

\subsection{Images Obtained with Non-Sidereal Tracking}

Non-sidereal tracking leads to images in which stars are streaked, but the moving object appears as a point source.
Full processing -- providing reduced, calibrated, background-subtracted images that are delivered with astrometric and photometric solutions -- of these images is beyond the scope of the DM pipelines as it would require the development of new algorithms, and will need to be done as a user generated pipeline.
The first steps of such a pipeline, such as Instrument Signature Removal, will probably be possible to achieve by reconfiguring the relevant DM software tasks.


%%% MLG removed the following in Feb 2022, it's no longer a concern.
% \subsection{Number of Exposures per Visit (Long Sequences of a Single Field)}

% There is no processing constraint on the number of consecutive exposures that could be obtained of a single field.
% From a DM perspective, it would be best if these exposures were packaged into visits of no more than 2 exposures per visit, to minimize the need to reconfigure of the pipelines, and because the camera only ``clears" between visits. 

% K.-T. Lim has pointed out that an odd number of exposures is a non-standard visit; two snaps is hardwired into the code. This is baked-in to a configuration so that the pipeline can have a definition of what kind of timing delay constitutes ``late".  Moving away from 2 exposures per visit requires a configuration change to the pipelines, which incurs an overhead (up to 1 minute) -- in fact, K.-T. things that between $10$ and $120$ seconds exposure times can easily be handled by the pipeline (i.e., can be run through ISR using scaled calibration frames), so long as they come in pairs. The real problem is knowing how long the processing should take, and not killing a process that is taking longer because there were 4 snaps in the visit instead of 2. To accommodate non-standard visits requires that the scheduler pass on the information of the number of snaps in the visit (\ref{DMSR-1}). Then the processing pipeline will know to, e.g., not attempt to difference the two snaps in the case were there is an odd number of snaps in a visit. \textit{MLG -- I've heard rumors of a CR regarding alternate standard visits of $1\times30$ seconds, but do not know the status or implications of this.}

% K.-T. has also pointed out that currently, a deep drilling field would be interpreted as a single visit of 50 exposures by the scheduler. One implication of this is that since the camera only ``clears" prior to a new visit, it would not do this for the entire 50-exposure sequence. The processing pipeline would need to know how to divide this sequence up into visits. As there is no current requirement for DM to receive the information that the scheduler is about to do a 50 exposure visit, we need \ref{DMSR-1} to add the proposal ID and the number of exposures per visit to the meta data, and then it should be OK for DM to parse this visit information in the reduction pipeline.


%%% MLG removed the following in Feb 2022, it's no longer a concern for Special Programs (crowded fields are in the WFD).
% \subsection{Images in Very Crowded Fields}

% The LSST pipelines' performance in crowded fields is documented in \citeds{DMTN-077}, which finds that, e.g., in Galactic Plane regions with a source density of $500000$ sources per square degree, the completeness drops to 50\% at $20.2$ magnitudes.
% The slide deck at \citeds{Document-27962} also describes DM's plans for processing crowded fields. These may or may not be appropriate for Special Programs data, depending on the science goals.

\section{Documentation Review for Requirements Related to Special Programs}\label{sec:docrev}

Below are described mentions of Special Programs in Rubin documentations, with a focus on requirements and specifications.

Updates to Rubin documents related to Special Programs -- motivated by past versions of this DMTN -- were made via LCR-1309 and LCR-2265. 

% LCR 1309: https://project.lsst.org/groups/ccb/node/2383
% LCR 2265: https://project.lsst.org/groups/ccb/node/4036


\subsection{Science Requirements Document (SRD)}

Version 5.2.4 (revision 2018-01-30), \citedsp{LPM-17}.

Section 3.4 ``The Full Survey Specifications" states the SRD's assumption that 90\% of the total available survey time would be spent on the main survey, and that the remaining 10\% would be spent {\it ``to obtain improved coverage of parameter space ... [or to] observe special regions"}.



\subsection{LSST System Requirements (LSR)}

Version 7.1 (revision 2020-03-05), \citeds{LSE-29}.

Note that Version 5 (2018-06-26) was an update for LCR-1309, which added requirements, specifications, and discussions regarding the processing of Special Programs data based on earlier versions of this DMTN.

In Section 1.5.1.3, ``Processing Data from Special Programs", LSR-REQ-0122\lsrreq{0122} is a requirement that the LSST system {\it ``shall deliver unique and separate data products for visits from Special Programs"} whenever possible, and that they {\it ``shall be delivered on timescales intermediate"} to the Prompt and Data Release timescales {\it ``when this enables the intended science of the Special Program"}.
The discussion clarifies that {\it ``the term 'whenever possible' includes cases where the Data Management System can run original or reconfigured versions of existing pipelines, and excludes cases where the development of new algorithms, or the allocation of significant additional computational resources, are required"}.

In Section 2.4.1.1.2, ``Non-Standard Visit", LSR-REQ-0111\req{0111} requires that the LSST system {\it ``be capable of obtaining and processing exposures not taken in a standard visit mode including those with a minimum exposure time of"} 1 second (\reqparam{minExpTime}).
The discussion notes that {\it ``non-standard visit exposures may possibly be degraded in some aspects of performance (e.g. cosmic ray rejection on visits consisting of a single exposure), and might be incompatible with difference imaging and alert production (e.g., short exposures in which the PSF is
not fully formed)"}.

The above requirement indicates that the LSST system shall be able to process non-standard visits from Special Programs, but that the image quality might be degraded.
Improvements that require algorithms or processing outside of what the LSST Science Pipelines can provide would be left to the science community and require User-Generated pipelines and data products.

The requirement in Section 1.5.1.3 is echoed in Section 2.6.1.1, ``Organization of Data Products", in which LSR-REQ-0032\lsrreq{0032} is a requirement that the data processing system provide the means for three 'classes' of data products on different timescales (Prompt, Data Release, and User-Generated), and also to provide a means for processing Special Programs data because the {\it ``science goals of Special Programs may require that their processed data products be made available in an additional fourth class, and possibly with intermediate timescales"}.

In Section 2.6.1.1.3, ``Level 3 Data Products", LSR-REQ-0041\lsrreq{0041} specifies that the LSST system {\it ``shall support"} User-Generated data products.
The discussion clarifies that ``{\it there will be technical limits on DM's ability to meet this requirement, such as cases where an intensive amount of additional computational resources is required, because only ~10\% of the total computational system is allocated for user processing"}.
This level of support applies also to user processing of Special Programs data.
See also the reference to LSR-REQ-0055 below.

Section 2.6.1.1.4, ``Data Products for Special Programs", LSR-REQ-0121\lsrreq{0121} specifies that the LSST system {\it ``shall produce unique and separate Data Products as the result of processing data from Special Programs whenever possible, on a timescale that enables the intended science goals of the Special Program.
The cumulative size of the online Special Programs data products shall be no more than ~10\% of the size of the DRP data products from the most recent data
release"}.
The discussion clarifies that {\it ``the term 'whenever possible' includes cases where the Data Management System can run original or reconfigured versions of existing pipelines, and excludes cases where the development of new algorithms, or the allocation of significant additional computational resources, are required.
The cumulative size of the Special Programs data products is capped at ~10\% of the most recent DR because this matches the expected fractional survey area of Special Programs compared to the main survey"}.

In Section 2.7.1.6, ``Community Computing Services", LSR-REQ-0055\lsrreq{0055} requires that the LSST system {\it ``shall provide and maintain an amount of computing capacity equivalent to at least"} 10\% (\reqparam{userComputingFraction}) {\it ``of the total LSST data processing capacity (computing and storage) for the purpose of scientific analysis of LSST data and the production of"} User-Generated data products. 
The discussion clarifies that the scope of this service remains to be determined.
This level of computational resources includes user processing of Special Programs data.

In Section 3.1.3.1, ``Survey Time Allocation", LSR-REQ-0075\lsrreq{0075} requires that the {\it ``survey performance requirements shall be met utilizing approximately 90\% of the historically available observing time, leaving the remaining time available for yet to be defined special programs"}.

In Section 1.3.1.3, LSR-REQ-0124\lsrreq{0124}, the discussion specifies that an image quality parameter related to ellipticity applies only to main survey data.



\subsection{Observatory System Specifications (OSS)}

Version 19.1 (revision 2021-07-30), \citeds{LSE-30}.

Note that Version 13 (2018-06-26) was an update for LCR-1309, which added requirements, specifications, and discussions regarding the processing of Special Programs data based on earlier versions of this DMTN.

In Section 2.2.3.1, ``Standard Operating States", OSS-REQ-0044\ossreq{0044} specifies that {\it ``the LSST observatory system shall be designed and constructed to support ... manual observing - used for specific non-scheduler driven observing to support system verification and testing or specialized science programs"}.
Although most Special Programs will be executed via the survey scheduler as part of {\it ``fully automated observing"}, manual observing might be necessary for, e.g., target-of-opportunity Special Programs.

Section 3.1.5.1.2, ``Data Products Handling for Special Programs", OSS-REQ-0392\ossreq{0392} is essentially a flow-down of requirements from the LSR (0122, 0075, and 0121), and specifies that {\it ``the handling of data products from Special Programs shall be compliant with the approach defined in LSE-163"}.

In Section 3.2.5.3, OSS-REQ-0403\ossreq{0403} is a flow-down of LSR-REQ-0124\lsrreq{0124} related to the ellipticity correlation function distribution.

In Section 3.6.1.3, ``Continuous Exposures", OSS-REQ-0319\ossreq{0319} requires that {\it ``The Observatory shall be capable of continuous operation throughout a night with the interval between successive visits equal to the FPA readout time"}.
The discussion clarifies that {\it ``this mode of observing is needed to support observations when the telescope is not being re-pointed"}, such as deep drilling fields or other Special Programs.

In Section 3.6.1.4, ``Minimum Exposure Time", OSS-REQ-0291\ossreq{0291} specifies that {\it ``the camera shall be able to obtain a single exposure with an effective minimum exposure time of no more than"} 1 second (\reqparam{minExpTime}) {\it ``with a goal of an effective minimum exposure time of"} 0.1 seconds (\reqparam{minExpTimeGoal}). 
The discussion clarifies that {\it ``if the exposure is shortened from the 15 second nominal, the spacing between successive exposures should be extended to maintain the average readout rate consistent with a 15 second exposure"}, which may increase the overheads of Special Programs using short exposure times.
The discussion also clarifies that {\it ``if the exposure is lengthened from the 15 second nominal, the thermal stability may be affected, which may affect photometric accuracy"}.

In Section 3.6.1.5, ``Publish Visit Type", OSS-REQ-0384\ossreq{0384} specifies that {\it ``the OCS [Observatory Control System] shall configure the [Data Management System] DMS (in particular Prompt Processing) with the type of visits to be processed: Standard, Alternate, or a specific type of Non-Standard"}.
The discussion clarifies that this allows the Prompt processing pipeline to be reconfigured on-the-fly in order to incorporate non-standard visits from, e.g., Special Programs.
The time required for reconfiguration might introduce some latency or cause some images to not be processed by the Prompt pipeline.

In Section 3.6.2.1.2, ``Maximum time for operational filter change", OSS-REQ-0293\ossreq{0293} specifies that {\it ``the camera system shall provide the capability of changing the operational filter with any other internal filter in a time less than"} 120 seconds (\reqparam{tFilterChange}).
This would impose a large overhead on, e.g., a Special Program that changes filters often without slewing.
See also OSS-REQ-0295\ossreq{0295}, Appendix~\ref{sec:hardbounds} of this document, and/or the filter change memorandum (\url{ls.st/spt-494}), for more information about the total lifetime number of filter changes.

In Section 3.6.3.3, ``Rotator tracking Time, OSS-REQ-0301\ossreq{0301} specifies that {\it ``the LSST shall be able to maintain field rotation tracking over a period of at least"} 1 hour (\reqparam{rotTrackTime}).
The discussion clarifies that this {\it ``is driven by the need to conduct extended 'deep drilling' observations on a single field}.
There do not seem to be any constraints on the speed of the rotator or the minimum distance between successive visits.

In Section 3.6.3.10, ``Non-Sidereal Tracking", OSS-REQ-0380\ossreq{0380} specifies that {\it ``the LSST system shall be capable of tracking in an arbitrary direction on the sky along a parametric RA(t) and DEC(t) trajectory, at angular rates of up to"} 220 arcseconds per second (\reqparam{nonsiderealAngularRateEl} and \reqparam{nonsiderealAngularRateAZ}) {\it ``with a tracking error not to exceed"} 0.5 arcseconds per minute (\reqparam{nonsiderealTrackingError}).
The discussion notes that {\it ``this is standard capability for modern telescopes"}, but might be relevant to some Special Programs.

%%% MLG: this does not appear in OSS V13
% $\bullet$ OSS-REQ-0027 requires that the scheduling system be able to optimize over at least \texttt{nSciProp} = 6 "science proposals", where these "proposals" are observing targets/constraints such as the distribution of filters, the astronomical conditions, and relative priority (OSS 2.1.1.2, Multiple Science Programs). {\bf JIRA ticket DM-12579 confirmed that there is no maximum number, and so many Special Programs will be able to be included in the scheduler}.

%%% MLG: this does not appear in OSS V13
% $\bullet$ OSS-REQ-0381 requires that the schedule be able to handle targets of opportunity, which would be relevant for e.g., Special Programs for gravitational wave follow-up (OSS 2.1.1.7, Visit Optimization).

%%% MLG: these appear in OSS V13, but aren't as relevant to DMTN-065 anymore
% $\bullet$ OSS-REQ-0189 and OSS-REQ-0190 set the minimum number of raw exposures to be supported as \texttt{nRawExpNightWinterAvg} = 1960 per night on average (but up to \texttt{nRawExpNightMax} = 2800 per night if e.g., two hours of a short-exposure twilight mini-survey are included) and \texttt{nRawExpYear} = 5.5$\times10^5$ per year, respectively. These numbers are set by predicting the maximum number of exposures that would be acquired on the longest night of the year in WFD cadence with 2 second slews, assuming $\sim80\%$ completion, but adding a 10\% margin. These estimates appear adequate for Special Programs in general.

%%% MLG: these appear in OSS V13, but aren't as relevant to DMTN-065 anymore
% $\bullet$ OSS-REQ-0194 and OSS-REQ-0323 set the minimum number of calibration exposures to be supported as \texttt{nCalibExpDay} = 450 on average and \texttt{nCalExpYear} = 1.5$\times10^5$ per year, respectively. These are \textit{minimums}, and so if a Special Program requires additional exposures, this should be possible to accommodate.



\subsection{Data Management Subsystems Requirements (DMSR)}

Version 9 (revision 2021-02-12), \citeds{LSE-61}. 

Note that Version 8.3 (2020-05-04) was an update for LCR-2265, which updated requirements, specifications, and discussions regarding the processing of Special Programs data based on earlier versions of this DMTN.

In Section 1.2.3, ``Raw Science Image Metadata", DMS-REQ-0068\dmreq{0068} specifies that {\it ``for each raw science image, the DMS shall store image metadata"} including {\it ``Program metadata (identifier for main survey, deep drilling, etc.)"}.
The discussion clarifies that {\it ``the program metadata should be sufficient to associate an image with a specific Special Program so that DMS-REQ-0320 and DMS-REQ-0397 can be satisfied"}.

In Section 1.3.13, ``Alert Content", the discussion for DMS-REQ-0274\dmreq{0274} explains that the {\it ``program and/or scheduler metadata"} included in an alert packet {\it ``should be sufficient to identify whether the image is associated with a Special Program (such as an in-progress Deep Drilling Field)"}.

In Section 1.4.18.1, ``Produce All-Sky HiPS Map", the discussion for DMS-REQ-0379\dmreq{0379} raises the point that generating separate HiPS maps for Special Programs (e.g., DDFs) remains an open question.

In Section 1.4.18.5, ``Produce MOC Maps", DMS-REQ-0383\dmreq{0383} specifies that Data Release processing {\it ``shall include the production of Multi-Order Coverage maps for the survey data"}, and that {\it ``additional MOCs SHOULD be produced to represent special-programs datasets"}.
It is noted that a separate technical note would be created to define these MOCs.

The bulk of the DMS's requirements related to Special Programs are in Section 1.6 of the DMSR.

In Section 1.6.1, ``Processing of Data From Special Programs", DMS-REQ-0320\dmreq{0320} specifies that {\it ``it shall be possible for special programs to trigger their own data processing recipes, during the night instead of the nightly Alert Processing (but the recipes may still issue Alerts), or on alternative timescales"}.
The discussion clarifies that the {\it ``LSST will provide these recipes ... when
possible, which includes cases where DM can run original or reconfigured versions of existing pipelines, and excludes cases where the development of new algorithms, or the allocation of significant additional computational resources, are required. An example of an alternative timescale is a nightly trigger to coadd all the deep-drilling field images. Decisions about which recipes are applied to which Special Programs will be made by the Operations team, after consideration of the scientific goals, computational resources, and data rights policy"}.
This requirement is derived from OSS-REQ-0392, which is essentially a flow-down of requirements from the LSR (0122, 0075, and 0121).

In Section 1.6.2, ``Prompt/DR Processing of Data from Special Programs", DMS-REQ-0397\dmreq{0397} specifies that {\it ``it shall be possible for special programs data to be processed with the prompt and/or annual-release pipelines alongside data from the main survey"}.
The discussion further clarifies that {\it ``the data from Special Programs should only be included ... when it is (a) possible ... to do so without additional effort and (b) beneficial to the LSST's main science objectives. Decisions about which data are included ... will be made by the Operations team"}.
This requirement is also derived from OSS-REQ-0392, which is essentially a flow-down of requirements from the LSR (0122, 0075, and 0121).

In Section 1.6.3, ``Level 1 Processing of Special Programs Data", DMS-REQ-0321\dmreq{0321} specifies that {\it ``all [Prompt] processing from special programs shall be completed before data arrives from the following night's observations"}.
This is essentially adding a quantifier to DMS-REQ-0397, to specify that {\it ``when it is (a) possible ... to do so"} means when it is possible to complete the processing before the next night's observations.
This requirement is also derived from OSS-REQ-0392, which is essentially a flow-down of requirements from the LSR (0122, 0075, and 0121).

In Section 1.6.4, ``Constraints on Level 1 Special Program Products Generation", DMS-REQ-0344\dmreq{0344} specifies that {\it ``the publishing of [Prompt] data products from Special Programs shall be subject to the same performance requirements of"} 24 hours (\reqparam{L1PublicT}) for the release of Prompt data products and 1 minute (\reqparam{OTT1}) for the transmission of Alert packets.
This is essentially a more detailed version of DMS-REQ-0321 which includes the Alert production timescale.
This requirement is also derived from OSS-REQ-0392, which is essentially a flow-down of requirents from the LSR (0122, 0075, and 0121).

In Section 1.6.5, ``Special Programs Database", DMS-REQ-0322\dmreq{0322} specifies that {\it ``data products for special programs shall be stored in databases that are distinct from those used to store standard [Prompt] and [Data Release] data products"} and that {\it ``it shall be possible for these databases to be federated ... to allow cross-queries and joins"}.
This requirement is also derived from OSS-REQ-0392, which is essentially a flow-down of requirements from the LSR (0122, 0075, and 0121).

In Section 4.1.16, ``Level 2 and Reprocessed Level 1 Catalog Access", DMS-REQ-0313\dmreq{0313} specifies that {\it ``the DMS shall maintain ... versions
of the most recent catalogs generated from Special Programs data"}.
As with all LSST data, {\it ``there is no requirement for older data releases to be queryable"}.



\subsection{Data Management Applications Design (DMAD)}

Version 4.3 (revision 2020-11-10), \citeds{LDM-151}.

The DMAD is not a requirements document.
Instead, it describes the scientific design of the LSST Science Pipelines: the algorithms and software that will be implemented to meet the requirements for processing the LSST data. 

Special Programs are only mentioned a few times, either as a potential source of single-snap visits or as a potential source of reference images or catalogs (e.g., training sets).

As described above (e.g., LSR-REQ-0122), the LSST system shall deliver unique and separate data products for visits from Special Programs whenever this (1) enables the intended science of the Special Program and (2) can be accomplished using the original or reconfigured versions of the LSST Science Pipelines.
For cases in which the development of new algorithms or the allocation of significant additional computational resources are required to produce Special Programs data products, User-Generated pipelines and processing will be necessary.

The DMAD can be used as the reference document to decide whether a given Special Program will require User-Generated pipelines and processing.

%%% MLG: the following are all from <2018 and were not updated in 2022

% \subsubsection{Extreme-Depth CoAdds} The system has been sized to hold $\sim200$ exposures in memory at once, which defined by the current maximum number of visits per field in the WFD main survey in $10$ years (from a conversation with K.-T.~Lim). Note that the panchromatic CoAdds would be built from the individual filter CoAdds, so the algorithm does not need to handle $\sim800$ images. From a computational standpoint, $200$ is the maximum number of images that can be stacked with an algorithm that requires all images to be accessible in memory at once (i.e., loading all images and calculating the median for each pixel). Deeper stacks might be possible with algorithms that deal with images consequentially. It is conceivable that a Special Program which needs to stack $>200$ images is not possible to accomplish with reconfigured pipelines, and would have to be processed with external, user-contributed resources. However, the exact DM capabilities in this area are not yet well known because NCSA has not yet defined the machine capabilities. Furthermore, the planned commissioning data will go to a $\sim20$ year depth, and so it can reasonably be expected that DM will have to be able to accommodate at least a stack that deep.

% \subsubsection{Deblending} The deep deblender algorithm described in Section 5.3.3 will, out of necessity, be optimized for use in the bulk of the WFD main survey. It may or may not end up being appropriate for use in the Galactic Plane mini-survey area, depending on the science goal. Level 3 deblenders for specific Special Programs fields may require development by the user community.

% \subsubsection{Variability Characterization} The periodic and aperiodic variability characterizations described in Section 6.21 of the DMAD are placeholders, but are representative of what is likely to be implemented: algorithms that are applicable to a broad range of variability types. From DM's perspective, all that is needed is sufficient information to enable relatively useful filters, from which the downstream broker/user can do additional filtering, and these parameterizations might not be sufficient for all science goals. It is conceivable that the goals of a particular Special Program might require different algorithms; these could be provided by DM, or written as Level 3 and either made joinable to the DM reconfigured data products or perhaps incorporated directly.

% \subsubsection{Photometric Redshifts} As described in Section 5.6.5 the Level 2 DRP \texttt{Object} catalog will include a photometric redshift, but this algorithm will be produced by the science community and then adopted and run at scale by DM. It is conceivable that the photo-$z$ algorithm for a Special Programs data set, such as a deep drilling field, might be different from that used for the WFD main survey.



\subsection{Data Products Definitions Document (DPDD)}

Version 3.6 (revision 2021-12-17), \citeds{LSE-163}.

Section 6 describes the data products for Special Programs.
The DPDD is not a requirements document; Section 6 summarizes the requirements presented above and does not introduce any new constraints or new information about Special Programs. 

%%% MLG commented out the paragraphs below in 2022, they were out of date.

% $\bullet$ The database schema for \texttt{DIASource} does not appear to have an element that identifies which template image that was used. This will be needed for both Levels 1 and 3 differencing pipelines and products, for both Special Programs and WFD main survey data. DMS-REQ-0074 already does require that the identity of input exposures is stored for each difference image (LSE-61). In conversation with K.-T., we find that this isn't a problem, as it will be handled by provenance: the code configuration used and the time of processing are sufficient to identify and regenerate the template image. However, K.-T. has pointed out that the capability to regenerate the \textit{exact same} template -- the pixels that were subtracted -- is not a current deliverable. However, the stamp of the difference image will live on in the Alerts database, so we also do not foresee this as a problem.

% $\bullet$ The DMAD specifies that externally defined targets can be incorporated into the \texttt{Objects} catalog (Section 3.2.5), and this may be a particular interest to Special Programs. It is unclear how such targets will be identified or flagged as such in the database schema, and whether we need to add an element for this. Currently, the \texttt{Object} database contains an element \texttt{prv\_inputId} which is an \texttt{integer}, and is described as the \textit{"Pointer to prv\_InputType. Indicates which input was used to produce a given object."} Is that all we need? {\bf JIRA ticket DM-12580 clarified that  \texttt{Object.prv\_inputID} in the database schema is one possible way to identify whether an \texttt{Object} is an externally provided coordinate}.

% $\bullet$ The \texttt{Object} and \texttt{DIAObject} elements that have been reserved for variability characterization, as described in the DMAD and the DPDD, are as follows: \\
% \texttt{Object} and \texttt{DIAObject.lcPeriodic} = \texttt{float[6 x 32]} = Periodic features extracted from light-curves using generalized Lomb-Scargle periodogram \\
% \texttt{Object} and \texttt{DIAObject.lcNonPeriodic} = \texttt{float[6 x 32]} = Non-periodic features extracted from light-curves using generalized Lomb-Scargle periodogram \\
% Section 6.21 of the DMAD describes the nominal algorithms to define these parameters, but as we mentioned in Section \ref{ssec:docrev_dmad}, different kinds of variability might be measurable Special Programs cadences that are quite different from the WFD main survey. Are 32 floats in each of the 6 filters always going to be a large enough volume? The Transients and Variable Stars Science Collaboration currently has a Task Force assigned to address this, and {\bf it is the topic of JIRA ticket DM-12581}.



%%% MLG commented out this whole section in 2022, since LSE-180 is from 2013
%%%  LSE-180 is deprecated?

% \subsection{Photometric Calibration for the LSST Survey, \citeds{LSE-180}}

% LSE-180 is built on \texttt{OpSim} runs that do include some nominal DDF, but the photometric calibration investigated in this work does not much deal with potential issues induced by non-standard visit patterns or exposure times of Special Programs, as its scope is the WFD main survey. Potential issues with DM processing -- including calibrations -- of non-standard visit exposure times is raised in Section \ref{ssec:dmplans_NSV}. Regarding the reduction and calibration of non-standard visit images, LSE-180 makes two relevant points: \\
% $\bullet$ In LSE-180, it is assumed that all factors affecting the system transmission are stable on 15 second timescales (page 10), but not what the upper limit of that might be. \\
% $\bullet$ LSE-180 comments on the dither pattern for the WFD survey in that "dither patterns where the overlap is one quarter of the field of view or more produce results meeting the SRD requirements", but this is specific to photometric calibration of the WFD. The LSE-180 also mentions that an inappropriate dither pattern can make it hard to correct for the variation of system bandpass as a function of the focal plane position -- but so long as this is solved in the WFD, the corrections can be applied to the much smaller amount of data from the Special Programs.

% However, Lupton's recent work on calibrations has made much of LSE-180 obsolete, and it is not clear whether it is still needed as a separate document. At the time of writing, Lupton's most recent take on calibrations can be found in \url{https://github.com/lsst-dm/calibration/blob/master/calibration.pdf}. K.-T. referenced the Appendix B for the overview of calibration types, but it seems that there is not yet a final plan that can be assessed for its suitability for Special Programs. It is plausible that Special Programs could have their own calibration database, and obtaining additional calibration frames is not expected to cause a bottleneck (at the moment there is $\sim1.7$ hours per day for this, and exceeding this could eat into the engineering time). The most likely cause for trouble involving calibrations and Special Programs is the computational time needed to create the additional and/or different calibration files to be applied to non-standard Special Programs data, and/or the additional overhead required to load them in the Level 1 pipeline if the schedule interleaves SP-WFD-WFD-SP.

% {\bf JIRA ticket DM-12582 is currently open, and aims to define the potential additional calibration needs of Special Programs data.}


\end{document}
