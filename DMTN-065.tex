\documentclass[DM,lsstdoc,toc]{lsstdoc}
\usepackage{graphicx}
\usepackage{url}
\usepackage{latexsym}
\usepackage{enumitem}

\title[LSST Special Programs]{Data Management for LSST Special Programs}

\author{M.~L.~Graham, Y.~AlSayyad, K.~Bechtol, E.~C.~Bellm, J.~F.~Bosch, J.~L.~Carlin, 
G.~P.~Dubois-Felsmann, L.~P.~Guy, M.~Juri\'{c}, K.-T.~Lim, E.~S.~Rykoff, C.~T.~Slater,
and the Data Management System Science Team}

\setDocRef{DMTN-065}
\date{\today}
\setDocUpstreamLocation{\url{https://github.com/lsst-dm/dmtn-065}}

\setDocAbstract{
Special Programs are additional survey areas and/or observing strategies that are driven by specific science 
goals which build on, or are beyond, the core science pillars of the Wide Fast Deep Main Survey.
In order to meet the requirements and enable science related to Special Programs, this document provides 
recommendations for Rubin Data Management regarding processing and serving data products for Special Programs.
Hardware and processing boundaries on the potential diversity of data from Special Programs
are discussed along with scenarios in which user-generated processing 
and data products might be needed to meet Special Programs' science goals.
}

\setDocChangeRecord{%
\addtohist{0}{2017-11-14}{Status: internal working document.}{Melissa Graham}
\addtohist{1}{2018-06-17}{Updated to finalize and issue.}{Melissa Graham}
\addtohist{2}{2021-12-01}{Updates per DM-20375.}{Melissa Graham}
\addtohist{3}{2024-03-01}{Updates per DM-32723.}{Melissa Graham}
}

\begin{document}

\maketitle

% CITATION EXAMPLES
% \verb|\citellp|: \citellp{LPM-17, LSE-30} \\
% \verb|\citell|: (SRD; \citell{LPM-17,LSE-29}) \\
% \verb|\citep[][]|: \citep[e.g.,][are interesting]{LPM-17,LSE-29} \\
% \verb|\cite|: \cite{LPM-17,LSE-29}

\section{Executive Summary} \label{sec:execsum}

The three memoranda for Rubin Data Management regarding Special Programs are:

\begin{enumerate}

\item All visits should have both a region and a program label.
These labels should be propagated to all derived data products, such as
processed images, catalog data, and alerts. (Section~\ref{ssec:sci_labels})

\item All visits that \emph{can} be processed by the Prompt pipelines and generate 
alerts \emph{should} be, in support of time domain and Solar System science goals. (Section~\ref{ssec:proc_wfd})

\item Special Processing to create ``unique and separate" data products for Special Programs
should be done when it is possible and necessary, as defined in Section~\ref{ssec:proc_rdm}.

\end{enumerate}

\section{Introduction}\label{sec:intro}

The following is an introduction to the key terms related to Special Programs.

\subsection{Visit Types}

A visit is an observation of a single pointing at a given time, of which 
there are three types:

\begin{itemize}
\item Standard Visit -- Composed of $2\times15$ second exposures (commonly referred to as ``snaps").
\item Alternative Standard Visit -- Composed of a single $30$ second exposure.
\item Non-Standard Visit -- Any other exposure time(s) or number of snaps.
\end{itemize}

Non-standard visits with shorter or longer exposure times are being 
considered for some Special Programs.

\subsection{Wide Fast Deep (WFD) Program}

The WFD is the core science program of the LSST, designed to achieve the science 
goals defined by the Science Requirements Document (SRD; \citeds{LPM-17}).

The WFD program is defined from a scheduling perspective as a certain observing strategy
(i.e., a set of cadences and filter balances) applied to a low-dust region 
(extragalactic areas) and several special regions:
the Galactic bulge and plane; the South Celestial Pole (SCP); the North Ecliptic Spur (NES);
and the Virgo Cluster \citedsp{PSTN-055}.

The WFD is expected to use alternative standard visits in the $u$-band and 
standard visits in all other bands \citedsp{PSTN-055}.
Different cadences, filter balances, rotational dithers, etc., might
be applied in the different regions composing WFD.
For example, rolling cadence is expected to be implemented in the low-dust region only.

WFD program data will be processed with regular (non-special) processing.
This processing will produce, among other data products (LSE-163), 
a contiguous sky footprint that covers at 
least $\sim$18000 deg$^2$ with $\gtrsim$800 visits per field, 
and is expected to be accomplished with 85--90\% of the observing time.
This contiguous all-sky "deep coadd" may have region-dependent processing parameters and inputs
(e.g., different calibration parameters or deblending algorithms for crowded fields)
and will be of variable depth, as most special regions recieve fewer visits.

\subsection{Regular (non-special) processing}

This is a term used only in this document to refer to the image processing 
described in the Data Products Definitions Document
(LSE-163) that are designed for, and will be applied to, the WFD program's observations.

In some cases, regular processing is also appropriate for Special Programs.

\subsection{Special Programs} 

This is a Rubin Data Management term used to refer to sky regions within
or beyond the WFD footprint which have a distinct observing strategy from
the WFD program's plans for that region (e.g., different visit type, cadence, filter distribution).

Special Programs are typically driven by specific science goals that build on or 
add to the core science pillars of the LSST.

About 10--15\% of the total 10-year LSST time-span will be spent obtaining 
observations associated with Special Programs.
Special Programs includes non-WFD LSST components such as the Deep Drilling 
Fields (DDFs) and the mini-, micro-, and nano-surveys \citedsp{PSTN-055}.

Data products from Special Programs is subject to the Rubin Data Policy \citedsp{RDO-013}
in the same way as data products from the WFD program.

Some science goals for Special Programs can be met with regular processing,
but some will require Special Processing by Rubin Data Management or user-generated processing.

\subsection{Special Processing}

This is a Rubin Data Management term to describe processing that 
uses components of the LSST Science Pipelines and is applied by Rubin
Data Management to images from Special Programs.
Special Processing creates data products that are unique and separate from those produced
by regular processing for the WFD program.

Special Processing is likely to use different inputs or configurations for the
LSST Science Pipelines, or to run on different timescales, than regular processing - 
as appropriate for the Special Programs' data and science goals.
However, the development and application of specialized \emph{algorithms} or new software
is beyond the scope of Special Processing.

\subsection{User-Generated Processing}

Any processing of Rubin data done by users in order to reach specific science goals, including
processing for Special Programs data, is referred to as User-Generated Processing.

User-Generated Processing for Special Programs data would be necessary in cases where
the science goals require custom algorithms, software, or very large computational
capacities which are beyond the scope of Special Processing or the Rubin-provided
computational resources (Section~\ref{ssec:sci_comp}).

Guidelines for User-Generated Processing, and for user-generated data products
that can be federated with the Rubin-product data products (i.e., joinable tables),
is forthcoming.

\subsection{Deep Drilling Field (DDF)}

A single pointing for which many (e.g., a hundred) visits are obtained 
(usually sequentially) during a single night, and repeated every few 
nights.

As of the Phase 2 SCOC recommendations in \citeds{PSTN-055}, the five 
confirmed DDFs were:

\begin{itemize}
\item Elias S1 (00:37:48, -44:00:00)
\item XMM-LSS (02:22:50, -04:45:00)
\item Extended Chandra Deep Field-South (03:32:30, -28:06:00)
\item COSMOS (10:00:24, +02:10:55)
\item Euclid Deep Field South  (04:04:58, -48:25:23)\footnote{\url{https://www.cosmos.esa.int/web/euclid/euclid-survey}}
\end{itemize}

The DDFs will require Special Processing (Section~\ref{ssec:proc_rdm})
and may also benefit from User-Generated Processing.

\subsection{Mini-, Micro- and Nano-Surveys}

Specific sky areas covered by a few hundred, a hundred, or tens of visits (respectively).
This document will refer to them collectively as mini-surveys.

The sky areas of mini-surveys can be within, adjacent to, or detached from the WFD footprint.
Mini-surveys can have non-standard visits.

For a list of the mini-surveys under consideration,
see \citeds{PSTN-055}.

Most mini-surveys are likely to require Special Processing (to be considered on a case-by-case basis),
and some are also likely to require User-Generated Processing.


\section{Recommended Implementations to Enable Science from Special Programs}\label{sec:sci}

Recommendations regarding implementations by Rubin Data Management to meet the 
requirements related to Special Programs summarized in Section~\ref{sec:req},
which will enable science with data from Special Programs.

\subsection{Region and Program Labels}\label{ssec:sci_labels}

\textbf{All visits should have a region label \textit{and} a program 
label, and these labels should be propagated to processed images and catalog data.}

\textbf{Region Labels -- } based on Right Acension and Declination.
Region labels should include, e.g., WFD low-dust, Galactic Plane, Galactic 
Bulge, SCP, NES, or Virgo; DDF fieldname; mini-survey region name.
Region labels should be propagated to processed images (visit, 
difference, and deep coadds), catalog sources, and catalog static-sky objects. 

\textbf{Program Labels -- } 
Based on survey cadeance or strategy.
Program labels might include, e.g., WFD, DDF, mini-survey name, or other 
options like engineering, commissioning, and director's discretionary.
Program labels should be propagated to single-visit processed and 
difference images and catalog sources.

It would not make sense to propagate program labels to deep coadds or 
object tables, as these data products could be a mix of programs.

The existence and propagation of these labels is not itself a requirement.
It is recommended for implementation for three reasons.

\begin{enumerate}

\item To meet the requirement that metadata for observations associated 
with Special Programs is stored, and is sufficient for triggering 
real-time data processing recipes (Section~\ref{ssec:req_meta}).

\item To enable users to query and retrieve processed image and catalog
data associated with a specific Special Program, and meet the science goals
of that Special Program, when regular processing as been applied 
(e.g., Prompt Processing, Section~\ref{ssec:sci_pproc}).

\item To enable provenance when Special Programs data in included in regular
processing and the WFD program data products (e.g., if used to improve the
all-sky coadd).

\end{enumerate}

\subsection{Special Processing}\label{ssec:sci_sproc}

\textbf{Special Processing should be done by Rubin Data Management to 
produce unique and separate (but joinable) data products
for Special Programs when it is both possible and necessary.}

\textbf{Possible -- } When original or reconfigured versions of the LSST
Science Pipelines can be used, and no new algorithmic or software development,
or significant additional computational resources, are needed.

\textbf{Necessary -- } When the science goals for a Special Program cannot
be met by only including the data in regular processing (e.g., Prompt processing).

\begin{enumerate}

\item To meet the requirement that Rubin Observatory produce
unique, separate, and joinable data products whenever this is possible 
with the original or reconfigured versions of the LSST Science Pipelines
(Section~\ref{sec:req}).

\item To enable science with Special Programs by all users, not just those
with the time and effort to process the data, and to reduce computational
load (and potential redundancy) in User-Generated Processing.

\end{enumerate}

To further illustrate this recommendation, a few examples of Special Processing
cases are provided below.

\begin{itemize}

\item \textbf{Possible and necessary:}

\item \textbf{Possible but not necessary:}
a time-domain mini-survey that uses standard visits \emph{could}
have separate difference-image analysis object and source catalogs
generated, but this is not necessary as the science goals for the
mini-survey can be met by processing its data with regular Prompt
Processing (Section~\ref{ssec:sci_pproc}), and ensuring the
data is properly labeled (Section~\ref{ssec:sci_labels}).

\item \textbf{Necessary but not possible:}

\end{itemize}


RESTART EDITING PROCESS HERE



\textbf{Case B (possible and necessary for RDM to create separate data products) -- }
For some science goals, users will need distinct data products composed solely from visits associated with certain Special Programs (Section~\ref{ssec:proc_rdm}).
These cases are usually associated with image coaddition and the static-sky catalogs 
of sources detected and measured in the deeply coadded images. 

\textbf{Case C (not possible, so user-generated data products will be needed) -- }
For some science goals, users might require unique and separate data products, 
new algorithms or processing routines, and/or significant additional 
computational resources for Special Programs.
In these situations, user processing and user-generated data products will be 
needed (Section~\ref{ssec:proc_user}).
These cases are usually related to time domain science goals that utilize 
intermediate-timescale ``custom" coadds (e.g., weekly, monthly stacks), 
science goals that require special processing like shift-and-stack techniques, 
or non-standard visits that are outside the boundaries of what the LSST
Science Pipelines can process (Section~\ref{ssec:proc_bounds}).


\subsection{Prompt Processing}\label{ssec:sci_pproc}

\textbf{Special Processing should produce unique and separate data products
for Special Programs when it is both possible and necessary.}



\subsection{Computational Resources for User-Generated Processing}\label{ssec:sci_comp}

As mentioned in Section~\ref{ssec:req_ug}, Rubin Observatory will reserve 10\% of its 
total data processing capacity for users.
This component would include {\it all} user processing and re-processing of any and 
all LSST data, including Special Programs. 

As described in Section~\ref{ssec:proc_user} this processing capacity will be 
accessed via Rubin Science Platform, with a supported software environment and 
infrastructure for batch processing \citedsp{dmtn-202}.

Very computationally intense processing (e.g., shift-and-stack for faint moving 
objects) will likely require external resources\footnote{For more details about 
the boundary between what Rubin Observatory will provide (in terms of data products 
and processing resources) and what will be left to the expertise of the science community, 
see \url{https://www.lsst.org/about/dm/data-products}}.


\subsection{Rubin Science Platform Capabilities}\label{ssec:sci_rsp}

Users will need to be able to query for data that restricts by sky region and 
program label. 
This can be accomplished by including those labels in all image and catalog 
metadata as described in Section~\ref{ssec:sci_labels}, as the TAP service 
and butler already provide the mechanism for user-specified queries.

Users will need to be able to discover Special Programs data products when 
browsing data, such as all-sky maps. 




\section{Processing Data from Special Programs}\label{sec:proc}

A discussion of the anticipated details and challenges related to 
obtaining and processing data from Special Programs.

Appendix~\ref{sec:spcs} provides detailed examples for the processing
of data from Special Programs, including scenarios in which regular,
Special, and User-Generated Processing are all involved.

\subsection{Boundaries on Non-Standard Visits} \label{ssec:proc_bounds}

Special Programs that do not use standard or alternative standard visits
might be affected by hardware or processing boundaries.

\subsubsection{Hardware Boundaries}\label{ssec:proc_bounds_hardware}

Appendix~\ref{sec:hardbounds} lists all of the hardware boundaries that 
might constrain the potential diversity of Special Programs data.

In general, the currently-proposed Special Programs in \citeds{PSTN-055}
are not anticipated to be limited by hardware boundaries.

A few potential challenges posed by hardware
boundaries are summarized below.

\begin{itemize}

\item \textbf{Short exposures --}
Special Programs that use short exposures would be limited to the
minimum exposure time is 1 second (stretch goal: 0.1 seconds).
There is a potential hardware boundary that limits the readout rate to 1 
every 15 seconds, which would affect the image acquisition rate and 
increase the overheads on short exposures.

\item \textbf{Repeated pointing --}
Special Programs which require the \emph{exact same} field pointing and 
rotation for \emph{every exposure} (to sub-arcsecond levels) might run 
into hardware boundaries on pointing and tracking.

\item \textbf{Twilight images --}
Special Programs that obtain twilight images will be subject to safe
limits on sky background flux, as with any astronomical camera.

\end{itemize}

Finally, as a side note, Special Programs that request a high number of 
filter changes and/or long slews could be inefficient due to large overheads,
but would not be limited by hardware boundaries.

\subsubsection{Processing Boundaries}\label{ssec:proc_bounds_processing}

Appendix~\ref{sec:procbounds} describes the boundaries on what types of visits 
can be processed and calibrated by the Data Management System and the LSST
Science Pipelines.

Most of the currently-proposed Special Programs in \citeds{PSTN-055}
are not anticipated to be limited by hardware boundaries.
However, those which use non-standard visits, especially those with
short exposure time or those obtained during twilight, might
be affected by processing boundaries.

The most likely challenges posed by processing
boundaries are summarized below.

\begin{itemize}

\item \textbf{Very short exposures --}
Special Programs which use very short ($<$2 sec) exposures 
could be difficult to reduce due to an incompletely-formed PSF 
(Section~\ref{ssec:procbounds_expt}).
The Data Management System is required to be able to process exposure 
times as low as 1 second (Section~\ref{ssec:req_proc}), 
but it is known that such short exposures might have degraded image quality.

\item \textbf{Very short or very long exposures --}
Special Programs that use very short or very long ($>$150 sec) 
exposures could be difficult to calibrate due to having too few 
(or too few unsaturated) stars.

\item \textbf{Twilight images --}
For Special Programs which obtain images with very bright sky backgrounds
(twilight images), it is currently unclear whether they can be processed
with the LSST Science Pipelines; 
User-Generated Processing might be needed (e.g., \citealt{2022AJ....164..168S}).

\item \textbf{Streaked images --}
The full reduction and calibration of data from any Special Programs that 
use non-sidereal tracking, which produce images with star streaks, is
currently beyond the scope of the LSST Science Pipelines; 
User-Generated Processing would be needed.

\end{itemize}

\subsection{Regular (non-special) processing}\label{ssec:proc_reg}

Recall that the term ``regular processing" is used only within this document,
and refers to the image processing pipelines that are designed for, 
and will be applied to, the WFD program's observations (Section~\ref{sec:intro}).

Decisions about when to apply regular processing to Special Programs data,
or when to include it in the data products for the WFD program 
(e.g., if it improves the all-sky coadds), 
are ultimately left to the discretion of the Rubin Observatory's 
Data Management and System Performance teams.


\subsubsection{Prompt Processing and Alert Production}\label{sssec:proc_reg_prompt}

As described in Section~\ref{ssec:sci_pproc}, 
all visits that \emph{can} be processed by the Prompt pipelines and generate 
alerts \emph{should} be, in support of time domain and Solar System science goals.

\textbf{The meaning of ``can be processed".}
This is ultimately left to the discretion of
Rubin Data Management in Rubin Operations, but is expected to include
all standard and alternative visits in sky regions for which a template image exists.
This might also include some non-standard visits (shorter or longer exposures), 
as long as they can be processed by the Prompt pipeline and an appropriate template image exists.
Visits with very short or very long exposure times (or very bright sky 
backgrounds) might be excluded if they would need specialized algorithms for,
e.g., instrument signature removal, difference-imaging, template-generation 
(Section~\ref{ssec:proc_bounds}).
Alert latency requirements do apply to Special Programs data (Section~\ref{ssec:req_proc}),
and if they cannot be met then Prompt processing is ``not possible".

\textbf{The use of specialized (alternative) template images.}
If a Special Program's primary science goal requires specialized templates and 
Prompt processing, the Data Management System will have the capability to load 
and use an alternative template for some sky regions, based on the image metadata 
(i.e., the labels described in Section~\ref{ssec:sci_labels}). 
However, there would not be enough memory to hold alternative templates 
for the whole sky.

\textbf{How WFD and Special Program data would co-exist.}
No ``unique and separate" data products for the Special Progams would be 
produced by regular Prompt processing.
Special Programs data that is processed by the Prompt pipeline would 
contribute to the Prompt data products for the WFD program as 
described in Section 3 of the DPDD \citedsp{lse-163}. 
These data products are the results of Difference Image Analysis (DIA),
such as the difference images, catalogs of sources detected in difference
images ({\tt DiaSources}) and associated static-sky {\tt DiaSources}
into {\tt DiaObjects}, and alert packets.

Including visits from Special Programs in regular Prompt processing alongside
vists from the WFD program is not, in general, anticipated to affect WFD science goals.
For example, analyses for a WFD-only subset could still be done using the program and
region labels described in Section~\ref{ssec:sci_labels}, which would be
propogated to difference images, difference-image catalogss, and alerts.

\textbf{Potential issues with Prompt processing for untiled sequences from Special Programs.}
There are two potential issues with Prompt processing for DDFs, or any mini-survey 
that obtains a sequence of untiled images: images at the same pointing or which overlap.
For example, a DDF which obtains a hour-long series of about a hundred images at the same coordinates,
every few nights for a few months.

\begin{enumerate}

\item \textbf{DIA Object histories may become too large for the sizing model.}
Alert packets contain the full records of all associated 
{\tt DiaSources} from the past 12 months, but the alert
stream bandwidth is sized for the expected histories for
WFD program fields. 
The Prompt pipeline resources are also sized for the
WFD program, and it might not be possible to load up
thousands of epochs at a time.
The Data Management team will have to test the realized
capabilities of Prompt processing and alert distribtuion,
and potentially impose a mitigation strategy such as
limiting histories to the last $N$ observations instead
of the last 12 months in heavily-observed regions.

\item \textbf{The association of new DIA Sources into DIA Objects may be compromised.}
For consecutive images, processing for the second image begins when the processing for the 
first image is only halfway done.
At this point, the {\tt DiaObject} catalog has 
not yet been updated with the new {\tt DiaSources} detected in the first image.
Thus, the {\tt DiaSource}s from images one and two for a new transient 
would not be associated with a single {\tt DiaObject}, but instead would 
each instantiate a new {\tt DiaObject}.

\end{enumerate}

These two potential issues pose challenges, but are not necessarily showstoppers in processing 
Special Programs data with the Prompt pipelines. 
The overall impact on time-domain science would still be positive, even 
if mitigations are needed for these issues.
For example, brokers and users would be able to use the region and program labels
in the data as context (i.e., as flags) and avoid including 
limited-history or potentially-compromised {\tt DiaSources} in their
analyses if necessary.

\subsubsection{Solar System Processing}\label{sssec:proc_reg_ss}

Since Solar System Processing takes {\tt DiaSource}s as input, any 
Special Programs images that are processed by the Prompt pipeline
could be incorporated into Solar System Processing.

\subsubsection{Data Release Processing}\label{sssec:proc_reg_dr}

\textbf{Time-domain DIA data products --}
This includes the results of the annual reprocessing of WFD program data with 
Difference Image Analysis (DIA), and the production of Data Release
versions of the processed images (single-visit and difference images)
and associated catalogs ({\tt DiaSource}, {\tt DiaObject}, {\tt Source},
{\tt ForcedSource}, {\tt DiaForcedSource}, and so on).
These data products will be used primarily for time-domain science.
They should include Special Programs data for the same reasons as
provided in Section~\ref{ssec:sci_pproc}, and with the same 
considerations as discussed in Section~\ref{sssec:proc_reg_prompt}.
This processing should use the same template image for a given field.

\textbf{Static-sky data products --}
This includes the tessellation and coaddition of WFD program images
and the associated multi-band {\tt Object} catalog and survey property maps.
Whether and how to include any Special Programs data in these data products
is left entirely to the discretion of the Rubin Data Management team in Operations.
As an example, perhaps Special Programs images will only be included when they 
assist with uniformity or suppress edge effects or low-order modes in the 
all-sky photometric solutions.

\subsection{Special Processing}\label{ssec:proc_special}

As described in Section~\ref{ssec:sci_sproc}, 
Special Processing should be done by Rubin Data Management to 
produce unique and separate data products
for Special Programs when it is both possible and necessary.
This is a requirement (Section~\ref{ssec:req_dp}).

In short, \emph{possible} means that original or reconfigured versions of the LSST
Science Pipelines can be used , and \emph{necessary} means the primary science goal for Special Program
could not be met without the data products produced by the Special Processing.

\textbf{Joinable tables --}
Any tables for the unique and separate data products should be joinable to the 
data products for the WFD program, when possible.
This is a requirement (Section~\ref{ssec:req_dp}).

\textbf{Survey property maps --}
As for the tesslated all-sky coadded images made from WFD program data, survey property maps
should be made individually as part of any Special Programs that generates
tesselated coadded images. This is \emph{not} a requirement (Section~\ref{ssec:req_dp}).


\textbf{Special Processing timescales --}
The timescales for Special Processing should be adopted that best serve the
primary science goal for the Special Program.
For example, nighly-coadd difference-image analysis for DDFs is the most
scientifically useful if done on a daily cadence, but the deeply coadded
images for DDF fields could be released annually.

The interpretations of possible and necessary, and the scope of Special Processing,
are ultimately left to the discretion of Rubin Data Management in Operations.

Specific examples are provided only to illustrate this interpretation of possible and necessary.
These examples do not place limits on what Special Processing might be done.

\begin{itemize}

\item \textbf{Possible and necessary:}
in order to detect high-redshift (faint) galaxies in the DDFs,
Rubin Data Management uses the LSST Science Pipelines to deeply
coadd images for each field, and store the results of source
detection and characterization in unique and separate tables that
are included in the annual data release.

\item \textbf{Possible but not necessary:}
a time-domain mini-survey that uses standard visits \emph{could}
have separate difference-image analysis object and source catalogs
generated, but this is not necessary as the science goals for the
mini-survey can be met by processing its data with regular Prompt
Processing, and ensuring the data is properly labeled.

\item \textbf{Possible but not necessary (secondary science goals):}
a time-domain mini-survey (or DDF) has a secondary science goal of detecting
precursor outbursts for transients, which requires coadding images
within windows of days, weeks, and months to reach various depths.
This set of custom coadds may be considered as ``not necessary" and requiring 
User-Generated Processing.
Note that for the WFD program data, such custom coadds are also considered 
as beyond scope and in need of User-Generated Processing.

\item \textbf{Necessary but not possible:}
in order to find the most distant, faint Kuiper Belt Objects in the DDF,
a specialized, computationally intensive form of ``shift-and-stack" processing
is required for detection, but such algorithms are not used by the 
LSST Science Pipelines and so User-Generated Processing will be needed.
A second example is a twilight survey that uses non-standard visits 
which are outside the boundaries of what the LSST Science Pipelines can process.

\end{itemize}

Further examples of potential Special Processing for anticipated Special Programs are provided below.

\subsubsection{Deep Drilling Fields (DDFs)}\label{sssec:proc_special_ddf}

As the DDFs will likely be observed with standard or alternative standard 
visits, Data Management will be able to reconfigure existing pipelines for
Special Processing to produce unique and separate DDF data products.

For example, Special Processing for the DDF data products might include:
\begin{itemize}
\item nightly-coadded images (24 h)
\item nightly-coadded difference images (24 h)
\item {\tt DiaSource}- and {\tt DiaObject}-like catalogs for the nightly-coadds (24 h)
\item deeply-coadded images (all images to date; yearly)
\item templates for the nightly-coadded difference images (yearly)
\item {\tt Source}- and {\tt Object}-like catalogs for the nightly-coadded and deeply-coadded images (yearly)
\end{itemize}

\subsubsection{Short-Exposure Mini-Surveys}

As described in \citedsp{PSTN-055}, there are a few
short-exposure mini-surveys are under consideration.
Two examples are a short exposure map of the sky in $ugrizy$ for calibration,
and a Near-Earth Objects (NEO) twilight survey.

Special Processing for short-exposure mini-surveys remains to
be determined.
Since it falls under the remit of Data Management to perform proper calibration,
an evaluatation of whether short exposures for calibration are necessary
will be done. 
If Data Management does find that short-exposure and/or high-sky brightness images
can be processed with reconfigured versions of the LSST Science Pipelines,
then unique and separate data products could be generated with Special Processing.
For the NEO twilight survey, these data products would likely be similar to the
Prompt or DIA data products.
For the calibration survey, these data products would like be similar to the
annual data release tessellated coadds and associated catalog.

\subsubsection{Standard Visit Mini-Surveys}

Consider a Special Program in which a special region of sky is observed with standard
visits but a a special strategy or cadence which is significantly distinct from the WFD program,
and lasts for a limited amount of time.
For example, a short-term survey of the Magellanic Clouds, with dual primary science
goals in time-domain and static-sky science.

In cases like this, a set of unique and separate data products with the same formats as the 
time-domain DIA and static-sky data products described in Section~\ref{sssec:proc_reg_dr}
should be created with Special Processing.
They might be released with an annual data release or on an intermediate timescale, e.g.,
within six months of the conclusion of the mini-survey observations.

\subsubsection{Target-of-Opportunity (TOO) Observations}

Options for RDM to process TOO observations, especially during the first year of Operations 
when the template coverage will be low, are discussed in more detail in 
\citeds{rtn-008}.

\subsection{User-Generated Processing}\label{ssec:proc_user}

Science goals that require data products which are not possible to create with 
the original or reconfigured versions of the LSST Science Pipelines, 
and/or for which new algorithmic development or significant computational resources 
are needed, will require user processing and user-generated data products.
As described above, custom coadds (e.g., weekly, monthly) are also left to users 
to generate, as required by their specific science goals.

\textbf{Computational resources -- }
Users will have access to the LSST Science Pipelines and data processing 
infrastructure, as well as dedicated computational resources next-to-the-data, 
via the Rubin Science Platform; \citeds{lse-319}.
Details of the planned ``User Batch" facility for data processing are described in 
\citeds{dmtn-202}.
Very computationally intense user processing might require external resources. 

\textbf{Adopting user code or data products --}
It is expected that some User-Generated pipelines and data products 
might be ``adopted" or ``federated" into the LSST Science Pipelines and the Prompt 
and Data Release data products.
Details regarding this are to be provided elsewhere.

\textbf{Alert production is restricted -- }
User-Generated Processing will not be able to release alert packets in the LSST alert stream.
As the latency on processed visit image availability has an 80-hour
embargo, no user-generated pipeline will be able to process Special Programs 
data on a timescale similar to prompt processing and alert production 
(60 seconds to 24 hours).
Thus, no User-Generated Processing may contribute alerts to the LSST alert stream on 
any timescale.

Further examples of potential User-Generated Processing for anticipated Special Programs are provided below.

\subsubsection{Deep Drilling Fields}

User-Generated Processing and data products might include, for example, DDF images coadded 
on custom timescales (e.g., weekly, monthly), or coadded using algorithms outside of the 
LSST Science Pipelines.

\subsubsection{Short-Exposure Mini-Surveys}

Short-exposure images obtained during twilight, which will have a very bright sky 
background unlike other LSST images, might require specialized algorithms
to subtract the high sky background which are not available in the LSST Science Pipelines,
and might require User-Generated Processing.

Short-exposure images obtained during the night might have too-few stars to satisfy the
astrometric and photometric calibration routines in the LSST Science Pipelines,
and might require User-Generated Processing.

\subsubsection{Mini-Surveys}

Mini-surveys with time-domain science goals that aren't met by the Prompt pipelines, 
e.g., those that require difference imaging with coadded images on an intermediate 
timescale (e.g., a weekly stack), would require User-Generated Processing.


\section{Requirements Related to Special Programs}\label{sec:req}
% \lsrreq, \ossreq, \dmreq
% \reqparam

Appendix~\ref{sec:docrev} provides a full description of the requirements 
on the Data Management System related to Special Programs, which are summarized here.

\subsection{Metadata}\label{ssec:req_meta}

\textbf{Program metadata that is sufficient to trigger Special Processing should be stored. -- }
In order to support Special Programs processing, the LSST system is 
required to store metadata that includes program information for every raw 
image, such as identifiers for images obtained as part of the Main Survey 
or a Special Program (DMS-REQ-0068).
It is required that this metadata be sufficient for Special Programs to 
trigger their own real-time data processing recipes ``whenever possible" 
(DMS-REQ-0320), and be included in alert packets (DMS-REQ-0274).

\subsection{Data Products}\label{ssec:req_dp}

\textbf{Produce unique and special (and joinable) data products when possible. --}
Rubin Observatory and the LSST system (the observatory and the data 
management systems) are required to process Special Programs data to 
produce unique and separate data products ``whenever possible" 
(LSR-REQ-0121).
It is a requirement that these Special Programs data products be distinct 
and joinable with the Prompt and/or Data Release data products (DMS-REQ-0322).

The term ``whenever possible" includes cases where the original or 
reconfigured versions of the LSST Science Pipelines can be run, and 
excludes cases where the development of new algorithms or the allocation 
of significant additional computational resources are required 
(LSR-REQ-0121).

The statement ``to produce unique and separate data products" typically 
refers to producing the same kinds of data products as will be generated 
by the Prompt and Data Release pipelines (processed visit images, coadded 
images, difference images, and catalogs of sources and objects for those 
images).

The term ``joinable" means the Special Programs data products can be federated 
or cross-matched with the relevant Prompt and Data Release data products, and that
a column of cross-matched object identifiers is provided to enable table joins.

\textbf{The size of Special Programs data products should be about 10\% of the total. --}
It is a requirement that the cumulative size of the Special Programs data 
products generated by Rubin Observatory be no more than $\sim$10\% the 
size of the Data Release data products (LSR-REQ-0121).

The spirit of this requirement on data volume is that the size be proportional to the fraction 
of survey time spent on Special Programs.

The derivation of value-added data products, such as HiPS or MOC maps, for 
Special Programs remains an open question (DMS-REQ-0379, 0383), and is not required.
However, they are suggested to be produced with Special Processing where
appropriate (Section~\ref{ssec:proc_special}).

\subsection{Processing}\label{ssec:req_proc}

\textbf{Latency requirements apply to alerts from Special Programs images. --}
It is a requirement that any Special Programs processing done with the 
Prompt pipeline (or a reconfigured version of it) is subject to the same 
timescales and latency constraints of 24 hours for the release of Prompt 
data products and 1 minute for the transmission of Alert packets 
(DMS-REQ-0344).

\textbf{Intermediate timescales should be used to enable science when needed. --}
It is also a requirement that Special Programs processing be done on 
timescales intermediate to the Prompt and Data Release processing, 
``whenever possible" and whenever necessary to enable the intended science 
goals of the Special Program (LSR-REQ-0032).

\textbf{Exposure times of 1 second should be processable. -- }
It is a requirement that the LSST system be able to process non-standard 
visits with short exposure times as low as 1 second, with a discussion 
note that such short exposures might have degraded image quality 
(LSR-REQ-0111).

It is not a requirement, but processing for Special Programs by Rubin Observatory is expected to use no 
more than $\sim$10\% of computational and storage capacity of the Rubin data processing cluster 
(i.e., proportional to the fraction of survey time spent; Section 6 of LSE-163).

The 10\% of the total data processing capacity that Rubin Observatory is 
required to reserve for \emph{all} User-Generated Processing includes that 
applied by users to Special Programs data.
There is no additional capacity {\it for users} that will be reserved only for Special 
Programs data (LSR-REQ-0041).


% % % % % % % % % % % % % % % % % %
\bibliography{local,lsst,refs,books,refs_ads}

% % % % % % % % % % % % % % % % % %
\appendix

\section{Special Programs Processing Examples}\label{sec:spcs}

{\bf This section has not been updated since 2018.}

For further insight to the DM-related needs of potential Special Programs, we can write out all of the data acquisition and processing steps, in order, that some of the proposed Special Programs might use.
Note that we are not including any analysis in these descriptions, only processing and products. These are not necessarily complete and may even be incorrect in some places, as we are not experts in the science needs of these potential Special Programs; they could use some more thought and input.

Basic steps that we use to describe a processing case study: \\
Step 1. Data Acquisition. \\
Step 2. Inclusion in the Prompt Pipeline and Alert Generation. \\
Step 3. Delivery of LSST Processed Images. \\
Step 4. Reconfigured Processing Pipelines and Separate Data Products. \\
Step 5. Inclusion in the DRP Data Products for the WFD Main Survey. \\
Step 6. User-Generated Pipelines and Products. \\

\subsection{Searching for TNOs with Shift-and-Stack}\label{ssec:SPCS_TNO}

This Special Programs processing summary is based on Becker et al. (2011) white paper to find TNOs with shift-and stack (SAS) \citedsp{Document-11013}.

Step 1. Data Acquisition. \\
The observational sequence is triggered. In a single night, the 9 adjacent fields in a 3x3 grid are observed with $336$ $\times$ $15$ second $r$-band exposures. This sequence is always repeated 2-3 nights later. This re-visit sequence is repeated 3 more times: 1.5 months, 3 months, and 13.5 months later. Data obtained in the $g$-band filter is also acceptable. \citedsp{Document-11013}

Step 2. Inclusion in the Prompt Pipeline and Alert Generation. \\
Each $2\times15$ second visit is processed in the Prompt pipeline and Alerts are released within 60 seconds.

Step 3. Delivery of LSST Processed Images. \\
The raw, reduced, and calibrated exposures and difference images from the Prompt pipeline are made available within \texttt{L1PublicT} (currently 24 hours; LSR-REQ-0104), but this is not very relevant for this program, which requires a year of dispersed observations before the processing pipelines for SAS can be run.

Step 4. Reconfigured Processing Pipelines and Separate Data Products. \\
Shift-and-stack processing is beyond the scope of DM's algorithms.

Step 5. Inclusion in the DRP Data Products for the WFD Main Survey. \\
As with all Special Programs data, they might be included in the products of the WFD main survey if DM decides it is beneficial. However, since these images are much deeper than stacks made from the WFD survey, and the strict timing of the observations might lead to their acquisition in sub-optimal conditions, it is unlikely that they would \textit{all} be incorporated.

Step 6. User-Generated Pipelines and Products. \\
The User-Generated pipeline running the shift-and-stack processing will be set up and submitted for batch processing by the user through the Science Platform or on an external processor. Pipeline inputs will be the 336 processed exposures per field per re-visit sequence. The DRP difference imaging routine will be used with the same template tract/patch for all. Custom, User-Generated algorithms will shift the exposures and create difference images, then DRP routines can stack and do source detection and characterization and generate an object database. Custom code will derive orbital parameters for the detections and add them to a {\tt SSObjects}-like database.


%% % % % % % %
\subsection{A More General Level 3 Shift-and-Stack Case Study, by Mario}\label{ssec:SPCS_SAS}

A Level 3 processing case study for shift-and-stack on a large number images. By Mario.

\#1. The scheduler is configured to repeatedly (e.g., 10 times) observe a field during the same night with longer exposure than usual (e.g., 120 sec). [ and we should take the actual numbers from the TNO-DDF whitepaper; don't have the internet right now or I would].

\#2. The images are processed as regular "Level 1" products within 60 seconds, and transmitted as alerts, with results stored into the regular L1 database. This will happen automatically for all images (perhaps within some range of exposure times?).

\#3. The raw images (and all necessary calibrations), calexps, and standard L1 diffims are made available within 10 minutes to the batch system for processing with special programs-specific codes. This is the same batch system we make available to the users for running Level 3 codes [Q for us: is it? or is is the same one that's used to process calibrations? have these systems been sized?].

\#3 a). The code running the shift-and-stack processing will be externally developed and delivered, but will be installed and operated (and change controlled!) by the LSST Operations team. That is, we don't expect someone external to the ops team to babysit the code on a nightly basis. In fact, it's the opposite: once the codes are delivered, any changes will go through LSST's software change control process.

\#4. There will be a facility to trigger program-specific processing on the batch system upon the arrival of a new image (above); this processing will then be queued up for execution. We assume that the policy for processing of special programs data may give it preferential treatment relative to general-purpose L3.

\#5. Once the processing finishes, the results of will be stored to a program-specific database. No alerts (in VOEvent sense) will be issued. We will provide a generic notification facility (perhaps something as simple as an RSS feed) that new data has been made available in a certain database/data store. [This is an example where I'd want to make sure somebody within DM is planning to provide such a facility.].

\#6. The outputs stored can be special-program specific (i.e., tables with nearly arbitrary schemas -- some columns -- like ra/dec for spatial joins -- should be present in main tables). The outputs can also contain images (stored in also special-program specific repository), or custom products (treated like opaque files). The visualizations available for these (catalogs, images, arbitrary files) through the Portal will be limited (e.g., generic table visualizations or x-y plots).

\#7. When the images are made available to the batch system (step \#3), they also become available to *everyone*. I.e., someone else could also run a custom L3 pipeline on these data, feeding their custom L3 database. (This isn't in the requirements right now -- we say that images will become available in 24hrs -- and is addressed in Section \ref{ssec:dmplans_user}).

\subsection{Searching for Supernovae in Deep Drilling Fields}\label{ssec:SPCS_SNDDF}

Step 1. Data Acquisition. \\
On a single deep drilling field, the scheduler obtains e.g., 5, 10, 10, 9, and 10 visits with $2\times15$ second exposures in $grizy$ (or similar for the night's filter set) and a small dither pattern between visits.

Step 2. Inclusion in the Prompt Pipeline and Alert Generation. \\
Each $2\times15$ second visit is processed by the Prompt pipeline's DIA, and Alerts are released within 60 seconds. They are flagged to denote the image source is a DDF and that source association might be compromised.

Step 3. Delivery of LSST Processed Images. \\
The raw, reduced, and calibrated exposures and difference images from the Prompt pipeline are made available within \texttt{L1PublicT} (currently 24 hours; LSR-REQ-0104).

Step 4. Reconfigured Processing Pipelines and Separate Data Products. \\
The required data products for this science goal can be met by reconfiguring the DM pipelines. First, a template image for the field will be made using DM stacking algorithms. On nights when this DDF is observed, at the end of the sequence of observations, DM algorithms are used to create a nightly deep stack, PSF-match it with the template, create a deep difference image, run source detection on the differences, and create separate databases of \texttt{DIAObject}, \texttt{DIASource}, and \texttt{Object} that are unique to this DDF. The LSST codes for alert packet and transport could be used to distribute the detected objects e.g., to the same brokers that receive the Alert Stream, or alternative destinations. However, these packets would not be distributed via the LSST {\tt Alert Stream}, and would need to be identified as, e.g., DDF Alerts.

Step 5. Inclusion in the DRP Data Products for the WFD Main Survey. \\
As with all Special Programs data, they might be included in the products of the WFD main survey if DM decides it is beneficial.

Step 6. User-Generated Pipelines and Products. \\
For the science goal of searching for supernovae in nightly stacked DDF images, no separate User-Generated software appears necessary.


\subsection{A Twilight Survey with Short Exposures}\label{ssec:SPCS_Twilight}

Several kinds of twilight surveys with short exposures have been or might be proposed: to put brighter stars (or transients such as supernovae) that saturate in a $15$ second image onto the LSST photometric system and/or to observe the Sweetspot, 60 degrees from the sun, for near-Earth objects. The processing case study for these is currently limited by unknowns about the first step: the reduction of processed single visit images.

Step 1. Data Acquisition. \\
At a specified time (or e.g., 6 degree twilight), the scheduler begins dither pattern of short exposures. Location and exposure times are set by the sky brightness and desired saturation limits.

Step 2. Inclusion in the Prompt Pipeline and Alert Generation. \\
Pending studies of short-exposure suitability for DIA (see Section \ref{sec:procbounds}) and scalable processing capabilities to incorporate a faster image-input rate than $1$ every $30$ seconds, these data could {\it potentially} be incorporated and spawn Alerts.

Step 3. Delivery of LSST Processed Images. \\
Pending the issues mentioned above, the raw, reduced, and calibrated exposures and difference images from the Prompt pipeline are made available within  \texttt{L1PublicT} (currently 24 hours; LSR-REQ-0104).

Step 4. Reconfigured Processing Pipelines and Separate Data Products. \\
This is officially not determined, but so long as the short-exposure images can be processed and have enough stars for photometric and astrometric calibration, reconfigured DM pipelines will probably be sufficient for creating image and catalog products from this kind of data.

Step 5. Inclusion in the DRP Data Products for the WFD Main Survey. \\
These short-exposure, high sky background images would not contribute to the DRP data products created for the WFD survey.

Step 6. User-Generated Pipelines and Products. \\
If short-exposure images cannot be processed with the existing DM algorithms, a User-Generated processing pipeline might be needed to reduce the raw data. 

Side note: A short-exposure survey of the bright stars of M67, described in Chapter 10.4 of the Observing Strategy White Paper \citep{2017arXiv170804058L}, suggests using the stretch goal of 0.1 second exposures or, if that is not possible, \textit{"custom pixel masks to accurately perform photometry on stars as much as 6 magnitudes brighter than the saturation level"}. This would be considered a User-Generated algorithm.

\subsection{The Galactic Plane Survey for Variable Stars and/or Exoplanets}\label{ssec:SPCS_GPVSEx}

Step 1. Data Acquisition. \\
The schedule incorporates fields in the Galactic Plane, and executes $2\times15$ second visits in these fields (or shorter, for a shallower depth than the WFD main survey).

Step 2. Inclusion in the Prompt Pipeline and Alert Generation. \\
Each $2\times15$ second visit is processed in the Prompt pipeline and Alerts are released within 60 seconds. Extremely crowded fields might have to be skipped if they take longer to process and violate the $60$ second latency for Alerts. 

Step 3. Delivery of LSST Processed Images. \\
The raw, reduced, and calibrated exposures and difference images from the Prompt pipeline are made available within  \texttt{L1PublicT} (currently 24 hours; LSR-REQ-0104).

Step 4. Reconfigured Processing Pipelines and Separate Data Products. \\
The image and catalog products needed for science with the Galactic Plane are very similar to the products of the Prompt and DRP pipelines, so it seems that not much reconfiguration would be needed. The biggest difference might be the incorporation of a user-supplied deblender algorithm optimized for very crowded fields.

Step 5. Inclusion in the DRP Data Products for the WFD Main Survey. \\
It is quite likely that images from the Galactic Plane will be included into the products of the WFD main survey, as they could e.g., reduce edge effects and help with global photometric classification, but this will depend on deblender performance, and left to the discretion of DM. 

Step 6. User-Generated Pipelines and Products. \\
It seems likely that science users will want to deploy their alternative deblending algorithms on this data set and create their own catalogs.

\subsection{Gravitational Wave Event Follow-Up}\label{ssec:SPCS_GW}

For a description of how target of opportunity data to search for the optical counterparts of gravitational wave events would be processed, see \citeds{rtn-008}.


% \subsection{A More General Level 3 Shift-and-Stack Case Study, by Mario}\label{ssec:SPCS_SAS}

% A Level 3 processing case study for shift-and-stack on a large number images. By Mario.

% \#1. The scheduler is configured to repeatedly (e.g., 10 times) observe a field during the same night with longer exposure than usual (e.g., 120 sec). [ and we should take the actual numbers from the TNO-DDF whitepaper; don't have the internet right now or I would].

% \#2. The images are processed as regular "Level 1" products within 60 seconds, and transmitted as alerts, with results stored into the regular L1 database. This will happen automatically for all images (perhaps within some range of exposure times?).

% \#3. The raw images (and all necessary calibrations), calexps, and standard L1 diffims are made available within 10 minutes to the batch system for processing with special programs-specific codes. This is the same batch system we make available to the users for running Level 3 codes [Q for us: is it? or is is the same one that's used to process calibrations? have these systems been sized?].

% \#3 a). The code running the shift-and-stack processing will be externally developed and delivered, but will be installed and operated (and change controlled!) by the LSST Operations team. That is, we don't expect someone external to the ops team to babysit the code on a nightly basis. In fact, it's the opposite: once the codes are delivered, any changes will go through LSST's software change control process.

% \#4. There will be a facility to trigger program-specific processing on the batch system upon the arrival of a new image (above); this processing will then be queued up for execution. We assume that the policy for processing of special programs data may give it preferential treatment relative to general-purpose L3.

% \#5. Once the processing finishes, the results of will be stored to a program-specific database. No alerts (in VOEvent sense) will be issued. We will provide a generic notification facility (perhaps something as simple as an RSS feed) that new data has been made available in a certain database/data store. [This is an example where I'd want to make sure somebody within DM is planning to provide such a facility.].

% \#6. The outputs stored can be special-program specific (i.e., tables with nearly arbitrary schemas -- some columns -- like ra/dec for spatial joins -- should be present in main tables). The outputs can also contain images (stored in also special-program specific repository), or custom products (treated like opaque files). The visualizations available for these (catalogs, images, arbitrary files) through the Portal will be limited (e.g., generic table visualizations or x-y plots).

% \#7. When the images are made available to the batch system (step \#3), they also become available to *everyone*. I.e., someone else could also run a custom L3 pipeline on these data, feeding their custom L3 database. (This isn't in the requirements right now -- we say that images will become available in 24hrs -- and is addressed in Section \ref{ssec:dmplans_user}).


\section{Potential Hardware Boundaries on Data Diversity}\label{sec:hardbounds}

The potential boundaries on the diversity of data products that could be imposed by 
limitations from the Rubin Observatory hardware (camera, telescope, and/or site) 
are considered.

\subsection{Filter Changes}

The maximum time for filter change is 120 seconds: 30 seconds for the telescope to 
reorient the camera to its nominal zero angle position on the rotator, and 90 seconds to 
the camera subsystem for executing the change (OSS-REQ-0293; \citeds{LSE-30}).

Assuming that most Special Programs would be designed to keep overheads $<$100\% and would 
be using standard 30 second visits, the filter change time indicates that it is likely 
that at least 4 exposures in a given filter would be obtained between filter changes, but 
this is not actually a hardware boundary.

The filter change mechanism is designed to undergo a total of 100000 changes over its 
lifetime, and each filter is designed to support up to 30000 changes over its lifetime, 
where lifetime is 15 years.

That is an average of $\sim$27 changes per day, some of which would occur in they day 
during calibrations (estimate, $\sim$10) and the rest at night.

As stated in the filter change memorandum (\url{ls.st/spt-494}), {\it ``the system could 
support as many changes involving the 5 filters loaded in the carousel as desired, without 
any practical limitation"}.

\subsection{Filter Carousel Loads}

The filter carousel can hold five of the six LSST filters at a time. 
The system is designed to support $3000$ loads in $15$ years (\url{ls.st/spt-494}). 
Filter loads are only done in the day, and there will never be data in more than five 
filters in a given night.

\subsection{Exposure Times}

The minimum exposure time is $1$ second, with a stretch goal of $0.1$ seconds 
(OSS-REQ-0291; \citeds{LSE-30}). 
The maximum exposure time is not restricted.

\subsection{Readout Time}

The readout time is $2$ seconds, and would be significant overhead on short exposures.

\subsection{Inter-Image Time}

Images with exposure times $<15$ seconds {\it might} still have to be separated by $15$ 
seconds for thermal tolerance; i.e., the minimum readout rate might be one image every 
$15$ seconds, regardless of exposure time (OSS-REQ-0291; \citeds{LSE-30}).

As discussed in Jira ticket DM-12573, the main issue is thermal and is related to the 
shutter, both the motors and the brakes; an elevated Camera skin temperature would affect 
image quality.

As of 2022, early tests suggest that a sustained (30 minutes) sequence that increases the 
heat load by large factors would not work, but further functional testing of the system 
once the Camera was fully assembled are needed for full characterization of the issue.

This potential $15$ interval between images is also a potential hardware boundary on the 
potential diversity of data products.

\subsection{Telescope Slew}

As described in \citeds{Document-28382}, large slews would have considerable overheads, 
but there are no hardware boundaries on the size of a single slew or the accrued slew 
distance.

\subsection{Pointing and Sidereal Tracking}

The specifications for the telescope's pointing and tracking in \citeds{LSE-30} indicate 
that $<$0.2 arcsecond precision in field pointing (OSS-REQ-0302) and $<$1 arcsecond in 
open-loop tracking (OSS-REQ-0303) would not be possible, but guiding would improve the 
latter (OSS-REQ-0305).

Furthermore, obtaining the \emph{exact same} alignment of the pixel grid in RA-Dec 
{\it ``would put demands on the camera rotator that were not planned"}\footnote{As per 
C. Claver's comments in ticket DM-12573.}.

\subsection{Non-Sidereal Tracking}

The requirement that the LSST system be able to perform non-sidereal tracking is set by 
OSS-REQ-0380 in \citeds{LSE-30}. 
This capability will include angular rates of up to 220 arcseconds per second in both 
azimuth and elevation. 

\subsection{Camera Rotation}

The requirements on the rotator's capabilities do not set any limits on the per-night or 
total lifetime rotation (OSS-REQ-0301, -0300; \citeds{LSE-30}) which might put boundaries 
on the distance between successive visits or the ability to jump between two widely 
separated fields.

Currently, there are no hardware boundaries imposed by camera rotation constraints on the potential diversity of data products.


\section{Potential Processing Boundaries}\label{sec:procbounds}

The capability of the LSST Science Pipelines to process diverse data is explored below.

Note that processing boundaries might ultimately be defined not by what is technically 
possible, but by the resulting image quality parameters, e.g., the number of stars with 
sufficient flux for photometric calibration.

Furthermore, the processing boundaries might not be fully constrained until the final 
performance of the LSST Science Pipelines, as described in the Data Management 
Applications Design, \citeds{LDM-151}) document, is fully characterized.

{\bf Summary.}\\

\begin{itemize}
\item \bf{Short or long exposures.} Very short ($<$2 sec) exposures could be difficult to 
process due to an incompletely-formed PSF. Short (or long, $>$150 sec) exposures 
could be difficult to calibrate due to having too few (or too few unsaturated) stars.
\item \bf{Bright backgrounds (twilight).} It is currently unclear whether images with very 
bright sky backgrounds (twilight images) can be processed with the LSST Science Pipelines, 
or whether user generated pipelines will be needed.
\item \bf{Non-sidereal tracking.} The full reduction and calibration of images obtained 
with non-sidereal tracking, in which the stars are streaked, is currently beyond the scope 
of the LSST Science Pipelines, and will require a user generated pipeline.
\end{itemize}


\subsection{Exposure Times}\label{ssec:procbounds_expt}

Images which deviate significantly from the $15$ second duration for the WFD main survey 
may encounter issues in the instrument signature removal routine, in the correction for 
differential chromatic refraction, in the difference imaging analysis pipeline, and/or in 
the photometric and astrometric calibrations due to a differently sampled set of standard 
stars per CCD.

\subsubsection{Short Exposures (Non-Standard Visits of $<$30 sec)}

The LSST System Requirements document states that {\it ``The LSST shall be capable of 
obtaining and processing exposures not taken in a standard visit mode including those 
with a minimum exposure time of} {\tt minExpTime}", which is 1 second (stretch goal 0.1 
seconds; LSR-REQ-0111 in \citeds{LSE-29}).

However, for exposure times there are other considerations, as changing the exposure time 
also affects the photometric and astrometric calibrations.
Assuming that 1 second exposure can be reduced and calibrated, its detected point sources 
will span a dynamic range of $r$$\approx$ 13 to 21 magnitudes.
A template image built on 15 second exposures will saturate at $r$$\approx$15.8 mag, but 
this still leaves stars between 15.8 and 21.0 magnitudes to be used in the PSF-matching 
(and all other filters have a similarly large overlap).

In order for an image to be successfully PSF-matched to the template, the PSF must be 
well formed (no speckle pattern), and have a spatial variance that the pipeline is 
capable of modeling (be smoothly varying on some minimal scale).
As a simple demonstration, Figure \ref{fig:expt} shows that perhaps exposure times 
shorter than $2$ seconds do not have a well-formed PSF (using the centroid of a 2D 
Gaussian fit as a proxy for "well-formed").

\begin{figure}
\begin{center}
\includegraphics[width=14cm,trim={0cm 0cm 0cm 0cm}, clip]{figures/exptime.png}
\caption{At left, Arroyo atmosphere-only simulated PSF for LSST (with oversampled pixels) 
with exposure times of 0.5, 2, and 15 seconds (top to bottom), courtesy of Bo Xin. At 
right, blue and purple lines show the location of the centroid derived from a 2D Gaussian 
fit to the PSF as a function of exposure time, with the red dashed line showing the true 
center. We can see that for exposure times greater than 2 seconds, the centroid converges 
near its true value. \label{fig:expt}}
\end{center}
\end{figure}

\subsubsection{Long Exposures (Non-Standard Visits of $>$30 sec)}

There is no maximum exposure time specified for an LSST image.
Given that the template image will be a stack of at least a year or two of data, 
processing a $5$--$10$ times deeper single image through the difference imaging pipeline 
should be fine.

However, a $2\times150$ second exposure would saturate at $r \approx 18.3$, perhaps 
leaving too few stars overlapping with e.g., templates or WFD images, for astrometric and 
photometric calibrations.

Furthermore, cosmic-ray rejection completeness might be reduced for longer exposures 
(unknown), which could impact the quality of a difference image and the detected sources. 
Additionally, any system qualities that vary on short (but $>30$ second) timescales could 
inhibit photometric calibration (e.g., tracking).

\subsection{Twilight Images with a Bright Background}

Images obtained during twilight for scientific purposes are also likely to have shorter 
exposure times, and so the issues described in Section \ref{ssec:procbounds_expt} also 
apply here.

Whether or not bright-background images can (or shall) be fully processed -- reduced, 
calibrated, background-subtracted, and delivered with astrometric and photometric 
solutions -- or whether this will require a user generated pipeline, remains to be 
determined (see also the example in Section \ref{ssec:SPCS_Twilight}). 
This may depend on the exposure time and the number of stars available in the image.

\subsection{Images Obtained with Non-Sidereal Tracking}

Non-sidereal tracking leads to images in which stars are streaked, but the moving object 
appears as a point source.

Full processing -- providing reduced, calibrated, background-subtracted images that are 
delivered with astrometric and photometric solutions -- of these images is beyond the 
scope of the DM pipelines as it would require the development of new algorithms, and will 
need to be done as a user generated pipeline. 
The first steps of such a pipeline, such as Instrument Signature Removal, will probably 
be possible to achieve by reconfiguring the relevant DM software tasks.


\section{Documentation Review for Requirements Related to Special Programs}\label{sec:docrev}

This review is limited to requirements related to Special Programs processing or data products,
or requirements that constrain the diversity of images (e.g., exposure time limits).

Updates to Rubin documents related to Special Programs that were motivated by past versions of this DMTN
were made via LCR-1309 and LCR-2265. 

% LCR 1309: https://project.lsst.org/groups/ccb/node/2383
% LCR 2265: https://project.lsst.org/groups/ccb/node/4036


\subsection{Science Requirements Document (SRD)}

Version 5.2.4 (revision 2018-01-30), \citedsp{LPM-17}.

Section 3.4 ``The Full Survey Specifications" states the SRD's assumption that 90\% of the total 
available survey time would be spent on the main survey, and that the remaining 10\% would be spent 
{\it ``to obtain improved coverage of parameter space ... [or to] observe special regions"}.



\subsection{LSST System Requirements (LSR)}

Version 7.1 (revision 2020-03-05), \citeds{LSE-29}.

Note that Version 5 (2018-06-26) was an update for LCR-1309, which added requirements, specifications, 
and discussions regarding the processing of Special Programs data based on earlier versions of this DMTN.

In Section 1.5.1.3, ``Processing Data from Special Programs", LSR-REQ-0122\lsrreq{0122} is a requirement 
that the LSST system {\it ``shall deliver unique and separate data products for visits from Special Programs"} 
whenever possible, and that they {\it ``shall be delivered on timescales intermediate"} to the Prompt and 
Data Release timescales {\it ``when this enables the intended science of the Special Program"}.
The discussion clarifies that {\it ``the term 'whenever possible' includes cases where the Data Management System 
can run original or reconfigured versions of existing pipelines, and excludes cases where the development of new 
algorithms, or the allocation of significant additional computational resources, are required"}.

In Section 2.4.1.1.2, ``Non-Standard Visit", LSR-REQ-0111\lsrreq{0111} requires that the LSST system 
{\it ``be capable of obtaining and processing exposures not taken in a standard visit mode including those with a 
minimum exposure time of"} 1 second ({\tt minExpTime}\reqparam{minExpTime}).
The discussion notes that {\it ``non-standard visit exposures may possibly be degraded in some aspects of performance 
(e.g. cosmic ray rejection on visits consisting of a single exposure), and might be incompatible with difference 
imaging and alert production (e.g., short exposures in which the PSF is not fully formed)"}.

The requirement in Section 1.5.1.3 is echoed in Section 2.6.1.1, ``Organization of Data Products", in which 
LSR-REQ-0032\lsrreq{0032} is a requirement that the data processing system provide the means for three 'classes' of 
data products on different timescales (Prompt, Data Release, and User-Generated), and also to provide a means for 
processing Special Programs data because the {\it ``science goals of Special Programs may require that their processed 
data products be made available in an additional fourth class, and possibly with intermediate timescales"}.

In Section 2.6.1.1.3, ``Level 3 Data Products", LSR-REQ-0041\lsrreq{0041} specifies that the LSST system 
{\it ``shall support"} User-Generated data products.
The discussion clarifies that ``{\it there will be technical limits on DM's ability to meet this requirement, such as 
cases where an intensive amount of additional computational resources is required, because only ~10\% of the total 
computational system is allocated for user processing"}.
This level of support applies also to user processing of Special Programs data.
See also the reference to LSR-REQ-0055 below.

Section 2.6.1.1.4, ``Data Products for Special Programs", LSR-REQ-0121\lsrreq{0121} specifies that the LSST system 
{\it ``shall produce unique and separate Data Products as the result of processing data from Special Programs whenever 
possible, on a timescale that enables the intended science goals of the Special Program.
The cumulative size of the online Special Programs data products shall be no more than ~10\% of the size of the DRP 
data products from the most recent data release"}.
The discussion clarifies that {\it ``the term 'whenever possible' includes cases where the Data Management System can 
run original or reconfigured versions of existing pipelines, and excludes cases where the development of new 
algorithms, or the allocation of significant additional computational resources, are required.
The cumulative size of the Special Programs data products is capped at ~10\% of the most recent DR because this 
matches the expected fractional survey area of Special Programs compared to the main survey"}.

In Section 2.7.1.6, ``Community Computing Services", LSR-REQ-0055\lsrreq{0055} requires that the LSST system 
{\it ``shall provide and maintain an amount of computing capacity equivalent to at least"} 10\% 
({\tt userComputingFraction}\reqparam{userComputingFraction}) {\it ``of the total LSST data processing capacity 
(computing and storage) for the purpose of scientific analysis of LSST data and the production of"} User-Generated 
data products. 
The discussion clarifies that the scope of this service remains to be determined.
This level of computational resources includes user processing of Special Programs data.

In Section 3.1.3.1, ``Survey Time Allocation", LSR-REQ-0075\lsrreq{0075} requires that the {\it ``survey performance 
requirements shall be met utilizing approximately 90\% of the historically available observing time, leaving the 
remaining time available for yet to be defined special programs"}.


\subsection{Observatory System Specifications (OSS)}

Version 19.3 (revision 2022-12-12), \citeds{LSE-30}.

Note that Version 13 (2018-06-26) was an update for LCR-1309, which added requirements, specifications, and discussions regarding the processing of Special Programs data based on earlier versions of this DMTN.

In Section 2.2.3.1, ``Standard Operating States", OSS-REQ-0044\ossreq{0044} specifies that {\it ``the LSST 
observatory system shall be designed and constructed to support ... manual observing - used for specific 
non-scheduler driven observing to support system verification and testing or specialized science programs"}. 
Although most Special Programs will be executed via the survey scheduler as part of {\it ``fully automated 
observing"}, manual observing might be necessary for, e.g., target-of-opportunity Special Programs.

Section 3.1.5.1.2, ``Data Products Handling for Special Programs", OSS-REQ-0392\ossreq{0392} is a flow-down of 
requirements from the LSR (0122, 0075, and 0121; LSE-29), and specifies that {\it ``the handling of data products 
from Special Programs shall be compliant with the approach defined in LSE-163"}.

In Section 3.6.1.3, ``Continuous Exposures", OSS-REQ-0319\ossreq{0319} requires that {\it ``The Observatory shall be 
capable of continuous operation throughout a night with the interval between successive visits equal to the FPA 
readout time"}.
The discussion clarifies that {\it ``this mode of observing is needed to support observations when the telescope is 
not being re-pointed. For example observing ``deep drilling" fields..."}.

In Section 3.6.1.4, ``Minimum Exposure Time", OSS-REQ-0291\ossreq{0291} specifies that {\it ``the camera shall be 
able to obtain a single exposure with an effective minimum exposure time of no more than"} 1 second 
({\tt minExpTime}\reqparam{minExpTime}) {\it ``with a goal of an effective minimum exposure time of"} 0.1 seconds 
({\tt minExpTimeGoal}\reqparam{minExpTimeGoal}). 
The discussion clarifies that {\it ``if the exposure is shortened from the 15 second nominal, the spacing between 
successive exposures should be extended to maintain the average readout rate consistent with a 15 second exposure"}, 
which may increase the overheads of Special Programs using short exposure times.
The discussion also clarifies that {\it ``if the exposure is lengthened from the 15 second nominal, the thermal 
stability may be affected, which may affect photometric accuracy"}.

In Section 3.6.1.5, ``Publish Visit Type", OSS-REQ-0384\ossreq{0384} specifies that {\it ``the OCS [Observatory 
Control System] shall configure the [Data Management System] DMS (in particular Prompt Processing) with the type 
of visits to be processed: Standard, Alternate, or a specific type of Non-Standard"}.
The discussion clarifies that this allows the Prompt processing pipeline to be reconfigured on-the-fly in order to 
incorporate non-standard visits from, e.g., Special Programs.
The time required for reconfiguration might introduce some latency or cause some images to not be processed by the 
Prompt pipeline.

In Section 3.6.2.1.2, ``Maximum time for operational filter change", OSS-REQ-0293\ossreq{0293} specifies that 
{\it ``the camera system shall provide the capability of changing the operational filter with any other internal 
filter in a time less than"} 120 seconds ({\tt tFilterChange}\reqparam{tFilterChange}).
This would impose a large overhead on, e.g., a Special Program that changes filters often without slewing.
See also OSS-REQ-0295\ossreq{0295}, Appendix~\ref{sec:hardbounds} of this document, and/or the filter change 
memorandum (\url{ls.st/spt-494}), for more information about the total lifetime number of filter changes.

In Section 3.6.3.1, ``Absolute Pointing", OSS-REQ-0298\ossreq{0298} specifies that {\it ``the LSST shall point to 
a defined set of sky coordinates with an RMS accuracy of"} 2 arcseconds ({\tt absPointErr}\reqparam{absPointErr}).

In Section 3.6.3.3, ``Rotator tracking Time, OSS-REQ-0301\ossreq{0301} specifies that {\it ``the LSST shall be able 
to maintain field rotation tracking over a period of at least"} 1 hour ({\tt rotTrackTime}\reqparam{rotTrackTime}).
The discussion clarifies that this {\it ``is driven by the need to conduct extended 'deep drilling' observations on 
a single field"}.

In Section 3.6.3.5, ``Offset Pointing", OSS-REQ-0302\ossreq{0302} specifies that {\it ``the LSST shall be capable of 
offset pointing within a single field-of-view with a precision of no more than"} 0.2 arcseconds 
({\tt offsetPointingErr}\reqparam{offsetPointingErr}).

In Section 3.6.3.6, ``Open Loop Tracking", OSS-REQ-0303\ossreq{0303} specifies that {\it ``The LSST shall be 
capable of open loop tracking without the assistance of real time optical feedback to an accuracy of"} 1.0 arcseconds 
({\tt openTrackErr}\reqparam{openTrackErr}) {\it ``over any 10 minute duration during normal night time operations"}.
Note that the open loop tracking requirement is \emph{without guiding}.

In Section 3.6.3.10, ``Non-Sidereal Tracking", OSS-REQ-0380\ossreq{0380} specifies that {\it ``the LSST system shall 
be capable of tracking in an arbitrary direction on the sky along a parametric RA(t) and DEC(t) trajectory, at 
angular rates of up to"} 220 arcseconds per second ({\tt nonsiderealAngularRateEl}\reqparam{nonsiderealAngularRateEl} 
and {\tt nonsiderealAngularRateAZ}\reqparam{nonsiderealAngularRateAZ}) {\it ``with a tracking error not to exceed"} 
0.5 arcseconds per minute ({\tt nonsiderealTrackingError}\reqparam{nonsiderealTrackingError}).
The discussion notes that {\it ``this is standard capability for modern telescopes"}, but might be relevant to some 
Special Programs.


\subsection{Data Management Subsystems Requirements (DMSR)}

Version 9 (revision 2021-02-12), \citeds{LSE-61}. 

Note that Version 8.3 (2020-05-04) was an update for LCR-2265, which updated requirements, specifications, and 
discussions regarding the processing of Special Programs data based on earlier versions of this DMTN.

In Section 1.2.3, ``Raw Science Image Metadata", DMS-REQ-0068\dmreq{0068} specifies that {\it ``for each raw 
science image, the DMS shall store image metadata"} including {\it ``Program metadata (identifier for main survey, 
deep drilling, etc.)"}.
The discussion clarifies that {\it ``the program metadata should be sufficient to associate an image with a specific 
Special Program so that DMS-REQ-0320 and DMS-REQ-0397 can be satisfied"}.

In Section 1.3.13, ``Alert Content", the discussion for DMS-REQ-0274\dmreq{0274} explains that the {\it ``program 
and/or scheduler metadata"} included in an alert packet {\it ``should be sufficient to identify whether the image is 
associated with a Special Program (such as an in-progress Deep Drilling Field)"}.

In Section 1.4.18.1, ``Produce All-Sky HiPS Map", the discussion for DMS-REQ-0379\dmreq{0379} raises the point that 
generating separate HiPS maps for Special Programs (e.g., DDFs) remains an open question.

In Section 1.4.18.5, ``Produce MOC Maps", DMS-REQ-0383\dmreq{0383} specifies that Data Release processing 
{\it ``shall include the production of Multi-Order Coverage maps for the survey data"}, and that {\it ``additional 
MOCs SHOULD be produced to represent special-programs datasets"}.
It is noted that a separate technical note would be created to define these MOCs.

The bulk of the DMS's requirements related to Special Programs are in Section 1.6 of the DMSR.

In Section 1.6.1, ``Processing of Data From Special Programs", DMS-REQ-0320\dmreq{0320} specifies that {\it ``it 
shall be possible for special programs to trigger their own data processing recipes, during the night instead of 
the nightly Alert Processing (but the recipes may still issue Alerts), or on alternative timescales"}.
The discussion clarifies that the {\it ``LSST will provide these recipes ... when possible, which includes cases 
where DM can run original or reconfigured versions of existing pipelines, and excludes cases where the development 
of new algorithms, or the allocation of significant additional computational resources, are required. An example of 
an alternative timescale is a nightly trigger to coadd all the deep-drilling field images. Decisions about which 
recipes are applied to which Special Programs will be made by the Operations team, after consideration of the 
scientific goals, computational resources, and data rights policy"}.
This requirement is derived from OSS-REQ-0392, which is essentially a flow-down of requirements from the LSR (0122, 
0075, and 0121).

In Section 1.6.2, ``Prompt/DR Processing of Data from Special Programs", DMS-REQ-0397\dmreq{0397} specifies that 
{\it ``it shall be possible for special programs data to be processed with the prompt and/or annual-release pipelines 
alongside data from the main survey"}.
The discussion further clarifies that {\it ``the data from Special Programs should only be included ... when it is 
(a) possible ... to do so without additional effort and (b) beneficial to the LSST's main science objectives. 
Decisions about which data are included ... will be made by the Operations team"}.
This requirement is also derived from OSS-REQ-0392, which is essentially a flow-down of requirements from the LSR 
(0122, 0075, and 0121).

In Section 1.6.3, ``Level 1 Processing of Special Programs Data", DMS-REQ-0321\dmreq{0321} specifies that {\it ``all 
[Prompt] processing from special programs shall be completed before data arrives from the following night's 
observations"}.
This is essentially adding a quantifier to DMS-REQ-0397, to specify that {\it ``when it is (a) possible ... to do so"} 
means when it is possible to complete the processing before the next night's observations.
This requirement is also derived from OSS-REQ-0392, which is essentially a flow-down of requirements from the LSR 
(0122, 0075, and 0121).

In Section 1.6.4, ``Constraints on Level 1 Special Program Products Generation", DMS-REQ-0344\dmreq{0344} specifies 
that {\it ``the publishing of [Prompt] data products from Special Programs shall be subject to the same performance 
requirements of"} 24 hours ({\tt L1PublicT}\reqparam{L1PublicT}) for the release of Prompt data products and 1 minute 
({\tt OTT1}\reqparam{OTT1}) for the transmission of Alert packets.
This is essentially a more detailed version of DMS-REQ-0321 which includes the Alert production timescale.
This requirement is also derived from OSS-REQ-0392, which is essentially a flow-down of requirents from the LSR 
(0122, 0075, and 0121).

In Section 1.6.5, ``Special Programs Database", DMS-REQ-0322\dmreq{0322} specifies that {\it ``data products for 
special programs shall be stored in databases that are distinct from those used to store standard [Prompt] and 
[Data Release] data products"} and that {\it ``it shall be possible for these databases to be federated ... to allow 
cross-queries and joins"}.
This requirement is also derived from OSS-REQ-0392, which is essentially a flow-down of requirements from the LSR 
(0122, 0075, and 0121).

In Section 4.1.16, ``Level 2 and Reprocessed Level 1 Catalog Access", DMS-REQ-0313\dmreq{0313} specifies that 
{\it ``the DMS shall maintain ... versions of the most recent catalogs generated from Special Programs data"}. 
As with all LSST data, {\it ``there is no requirement for older data releases to be queryable"}.


\subsection{Data Management Applications Design (DMAD)}

Version 4.3 (revision 2020-11-10), \citeds{LDM-151}.

The DMAD is not a requirements document.
Instead, it describes the scientific design of the LSST Science Pipelines: the algorithms and software that will 
be implemented to meet the requirements for processing the LSST data. 

Special Programs are only mentioned a few times, either as a potential source of single-snap visits or as a 
potential source of reference images or catalogs (e.g., training sets).

As described above (e.g., LSR-REQ-0122), the LSST system shall deliver unique and separate data products for visits 
from Special Programs whenever this (1) enables the intended science of the Special Program and (2) can be 
accomplished using the original or reconfigured versions of the LSST Science Pipelines.
For cases in which the development of new algorithms or the allocation of significant additional computational 
resources are required to produce Special Programs data products, User-Generated pipelines and processing will be 
necessary.

The DMAD can be used as the reference document to decide whether a given Special Program will require User-Generated pipelines and processing.


\subsection{Data Products Definitions Document (DPDD)}

Version 3.6 (revision 2021-12-17), \citeds{LSE-163}.

Section 6 describes the data products for Special Programs.
The DPDD is not a requirements document; Section 6 summarizes the requirements presented above and does not introduce 
any new constraints or new information about Special Programs. 


\end{document}
