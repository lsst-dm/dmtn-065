\documentclass[DM,lsstdoc,toc]{lsstdoc}
\usepackage{graphicx}
\usepackage{url}
\usepackage{latexsym}
% \usepackage{color}
% black, blue, brown, cyan, darkgray, gray, green, lightgray, lime, magenta, blue, orange, pink, purple, red, teal, violet, white, yellow.
\usepackage{enumitem}

\title[LSST Special Programs]{Data Management \\ and LSST Special Programs}

\author{M.~L.~Graham, M.~Juri\'{c}, K.-T.~Lim, E.~Bellm, G.~Dubois-Felsmann, L.~P.~Guy, and the Data Management System Science Team}

\setDocRef{DMTN-065}
\date{\today}
\setDocUpstreamLocation{\url{https://github.com/lsst-dm/dmtn-065}}

\setDocAbstract{
The core LSST science goals will be met by the Wide-Fast-Deep (WFD) Main Survey, which is expected to be accomplished with 85--90\% of the observing time.
The remaining 10--15\% of the time will be spent on Special Programs: additional survey areas and/or observing strategies that are driven by specific science goals which build on, or are beyond, the core science pillars of the LSST.

\medskip
This document provides a summary of the Special Programs being considered, the potential diversity of data they might produce (e.g., shorter or longer exposure times), and the role of Rubin Observatory in reducing, processing, and serving data products from Special Programs.
The latter includes a discussion about the relevant system requirements, the needed capabilities of the LSST Science Pipelines and the Rubin Science Platform, and the cases in which Special Programs data products would be User-Generated.

\medskip
The specialized, science-specific aspects of Special Programs processing and analysis that are best left to the science community are also described, and illustrated with case-study examples.

\medskip
The main target audience of this document is Rubin Observatory staff -- the construction-era Data Management (DM) team and the operations-era Data Production (RDP) team -- but members of the science community who are planning to use Special Programs to reach their science goals may also find this document useful.
}

\setDocChangeRecord{%
\addtohist{0}{2017-11-14}{Status: internal working document.}{Melissa Graham}
\addtohist{1}{2018-06-17}{Updated to finalize and issue.}{Melissa Graham}
\addtohist{2}{2021-12-01}{Updates per DM-20375.}{Melissa Graham}
}

\begin{document}

\maketitle

% CITATION EXAMPLES
% \verb|\citellp|: \citellp{LPM-17, LSE-30} \\
% \verb|\citell|: (SRD; \citell{LPM-17,LSE-29}) \\
% \verb|\citep[][]|: \citep[e.g.,][are interesting]{LPM-17,LSE-29} \\
% \verb|\cite|: \cite{LPM-17,LSE-29}

% % % % % % % % % % % % % % % % % % 
\section{Terms and Definitions} \label{sec:terms}

\subsection{Wide-Fast-Deep (WFD) Main Survey}\label{ssec:terms_wfd}
The WFD Main Survey is the core science program of the LSST, designed to achieve the science goals defined by the Science Requirements Document (SRD; \citeds{LPM-17}).
The WFD will cover at least $\sim$18000 deg$^2$ and is expected to be accomplished with 85--90\% of the observing time.
As of the Phase 1 SCOC recommendations in \citeds{PSTN-053}, the WFD Main Survey is thought of as several contiguous areas.
The largest is the low-dust-extinction survey area, which would also receive the most ($>$800) visits over the 10-year survey and be completed to uniform depth, likely with a rolling cadence.
Smaller regions cover the higher-extinction high stellar density Galactic Bulge and Magellanic Clouds, the North Ecliptic Spur (NES), and the remainder of the Galactic Plane and the South Celestial Pole.
Boundaries, re-visit cadences, and depths for all WFD areas remain to be determined \citedsp{PSTN-053}.

\subsubsection{Mini-Surveys}\label{sssec:terms_wfd_mini}
As of the Phase 1 SCOC recommendations in \citeds{PSTN-053}, the term ``mini-survey" was being used to refer to the areas of the WFD / Main Survey outside of the large, low-dust-extinction survey area.
This term used to refer to non-WFD regions (Special Programs).

\subsection{Special Programs}\label{ssec:terms_sp}
Special Programs are additional survey areas and/or observing strategies that are driven by specific science goals which either build on, or are outside of, the core science pillars of the LSST.
Special Programs will fill the remaining 10--15\% of the available survey time, and include both ``deep drilling" fields and ``micro-surveys".

\subsubsection{Deep Drilling Field (DDF)}\label{sssec:terms_sp_ddf}
A DDF is a single pointing for which many exposures are obtained during the night.
The five DDFs (below) are expected to take $\sim$5\% of the total available time.
Generally, the LSST observing strategy for a DDF is to obtain some or all of these exposures consecutively, and to maintain a high inter-night cadence over a short season (e.g., returning every two nights over four months, \citealt{2019ApJ...873..111I}) -- but the exact strategies and total time for the LSST DDFs remain to be determined \citedsp{PSTN-053}.

\begin{itemize}
\item Elias S1 (00:37:48, -44:00:00)
\item XMM-LSS (02:22:50, -04:45:00)
\item Extended Chandra Deep Field-South (03:32:30, -28:06:00)
\item COSMOS (10:00:24, +02:10:55)
\item Euclid Deep Field South  (04:04:58, -48:25:23)\footnote{\url{https://www.cosmos.esa.int/web/euclid/euclid-survey}}
\end{itemize}

\subsubsection{Micro-Surveys}\label{sssec:terms_sp_micro}
As of the Phase 1 SCOC recommendations in \citeds{PSTN-053}, the term ``micro-survey" refers to either new sky areas observed with a WFD-like survey, or sky areas within the WFD but observed with a specialized strategy.
Micro-surveys are expected to take up the remaining $\sim$10\% of the total available survey time over LSST's 10 years, and those which would use $>$0.3\% of the time and are currently being considered by the SCOC are listed below (see also \citeds{PSTN-053}).

\begin{itemize}
\item short-exposure twilight visits (for near-Sun and near-Earth objects; NEOs)
\item a static short-exposure (5 sec) map of the sky in $ugrizy$ in the first year (for calibration)
\item an extended short-exposure survey of the sky in $ugrizy$ (for transient detection and static-sky calibration)
\item target-of-opportunity (ToO) follow-up (to identify optical counterparts to gravitational wave sources)
\item coverage of the Roman microlensing bulge field (potentially as a DDF)
\item deeper $g$-band imaging of 10 local volume galaxies
\item a high-cadence survey of 2 fields in the SMC (for microlensing)
\item annual week-long surveys of the Carina nebula and surrounding star-forming regions
\item a limited-visit (i.e., shallow) ``northern stripe" survey to declination $<$+30 degrees
\item a survey of the Virgo cluster to WFD depth
\end{itemize}

Note that there remains some grey area between what is a mini-survey within the WFD, and what is a WFD-like micro-survey.
For example, the last two on the list above (the northern stripe and the Virgo cluster) could also be (or might be in the future) called WFD mini-surveys, and not Special Programs micro-surveys.

\subsection{Visit Types}\label{ssec:terms_visits}
A visit is an observation of a single pointing at a given time, of which there are three types as listed below.
As of the Phase 1 SCOC recommendations in \citeds{PSTN-053}, most WFD and DDF visits would be standard or alternative standard visits.
However, non-standard visits with longer exposure times are being considered, especially for $u$-bands, and several of the micro-surveys are considering shorter exposure times.
The full potential for diversity in Special Programs data is reviewed in Section~\ref{ssec:proc_datadiv}.

\begin{itemize}
\item Standard Visit -- Composed of $2\times15$ second exposures (commonly referred to as ``snaps").
\item Alternative Standard Visit -- Composed of a single $30$ second exposure.
\item Non-Standard Visit -- Any other exposure time(s) or number of snaps.
\end{itemize}



\clearpage
% % % % % % % % % % % % % % % % % % 
\section{Reduction and Processing for Special Programs Data Products}\label{sec:proc}


\subsection{Requirements Related to Special Programs}\label{ssec:proc_reqs}
% \lsrreq, \ossreq, \dmreq
% \reqparam

A detailed list of the requirements related to Special Programs is provided in Appendix~\ref{sec:docrev}.
The most important requirements are summarized below.

\subsubsection{Data Products}

Rubin Observatory and the LSST system (the observatory and the data management systems) are required to process Special Programs data to produce unique and separate data products ``whenever possible" (LSR-REQ-0121).

The term ``whenever possible" includes cases where the original or reconfigured versions of the LSST Science Pipelines can be run, and excludes cases where the development of new algorithms or the allocation of significant additional computational resources are required (LSR-REQ-0121).

The statement ``to produce unique and separate data products" typically refers to producing the same kinds of data products as will be generated by the Prompt and Data Release pipelines (processed visit images, coadded images, difference images, and catalogs of sources and objects for those images).

It is a requirement that the cumulative size of the Special Programs data products generated by Rubin Observatory be no more than $\sim$10\% the size of the Data Release data products (i.e., proportional to the fraction of survey time spent; LSR-REQ-0121).

It is a requirement that these Special Programs data products be distinct, and joinable with (in other words, they can be federated or cross-matched with) the Prompt and Data Release data products (DMS-REQ-0322).

The derivation of value-added data products, such as HiPS or MOC maps, for Special Programs remains an open question (DMS-REQ-0379, 0383).

\subsubsection{Metadata}

In order to support Special Programs processing, the LSST system is required to store metadata that includes program information for every raw image, such as identifiers for images obtained as part of the Main Survey or a Special Program (DMS-REQ-0068).

It is required that this metadata be sufficient for Special Programs to trigger their own real-time data processing recipes ``whenever possible" (DMS-REQ-0320), and be included in alert packets (DMS-REQ-0274).

\subsubsection{Processing}

It is a requirement that Special Programs processing with the Prompt pipeline (or a reconfigured version of it) is subject to the same timescales and latency constraints of 24 hours for the release of Prompt data products and 1 minute for the transmission of Alert packets (DMS-REQ-0344).

It is also a requirement that Special Programs processing be done on timescales intermediate to the Prompt and Data Release processing, ``whenever possible" and whenever necessary to enable the intended science goals of the Special Program (LSR-REQ-0032).

It is a requirement that the LSST system be able to process non-standard visits with short exposure times as low as 1 second, with a discussion note that such short exposures might have degraded image quality (LSR-REQ-0111).

Processing for Special Programs by Rubin Observatory is expected to use no more than $\sim$10\% of computational and storage capacity of the Rubin data processing cluster (i.e., proportional to the fraction of survey time spent and the size of the Rubin-processed Special Programs data products; Section 6 of the DPDD).

\subsubsection{User Processing}

In cases where the science goals of a Special Program require that new algorithms or software be developed, User-Generated pipelines and data products will be needed.

The 10\% of the total data processing capacity that Rubin Observatory is required to reserve for all User-Generated processing includes that applied by users to Special Programs data -- in other words, there is no additional capacity {\it for users} that will be reserved only for Special Programs data (LSR-REQ-0041).



\subsection{The Potential Diversity of Special Programs Data} \label{ssec:proc_datadiv}

As discussed in Section~\ref{sec:terms}, most of the Special Programs that are currently under consideration will use standard or alternative standard visits.
However, some are likely to require non-standard visits with shorter exposures (5 sec).
Furthermore, some are likely to acquire images with a significantly brighter sky background (e.g., twilight images) than most nighttime survey images.

The cadence and patterns of Special Programs might also differ from the WFD, such as long series of exposures obtained of the same field (e.g., DDFs), or a strategy optimized to find very fast-moving objects (e.g., NEOs).

It does not appear that any of the currently-proposed Special Programs are likely to violate boundaries imposed by the Rubin Observatory hardware, but there remain a few open questions about boundaries on data processing imposed by the LSST Science Pipelines.

\subsubsection{Hardware Boundaries}

Appendix~\ref{sec:hardbounds} lists all of the hardware boundaries that might constrain the potential diversity of Special Programs data.

The minimum exposure time is 1 second (stretch goal: 0.1 seconds), and there is a {\it potential} hardware boundary that limits the readout rate to 1 every 15 seconds.
This {\it might} affect the image acquisition rate and increase the overheads of the proposed short-exposure micro-surveys -- but current testing of the camera system indicates that imposing waits between short exposures will not be necessary.

Hardware imposes no other boundaries on how data can be obtained, but Special Programs that request a high number of filter changes and/or long slews could be inefficient due to large overheads.

\subsubsection{Processing Boundaries}

Appendix~\ref{sec:procbounds} describes the boundaries on what types of visits can be processed and calibrated by the LSST Science Pipelines, which are designed to process standard (or alternative standard) visits.

Very short ($<$2 sec) exposures could be difficult to reduce due to an incompletely-formed PSF, and very short or very long ($>$150 sec) exposures could be difficult to calibrate due to having too few (or too few unsaturated) stars.
As mentioned above in Section~\ref{ssec:proc_reqs}, the LSST system is required to be able to process exposure times as low as 1 second, and it is known that such short exposures might have degraded image quality.

It is currently unclear whether images with very bright sky backgrounds (twilight images) can be processed with the LSST Science Pipelines, or whether user generated pipelines will be needed.

The full reduction and calibration of images obtained with non-sidereal tracking, in which the stars are streaked, is currently beyond the scope of the LSST Science Pipelines, and will require a user generated pipeline.




\subsection{Including Special Programs Data in the WFD Main Survey's Data Products}\label{ssec:proc_wfd}

Instances where the Special Programs data might be included in the Prompt and/or Data Release data products alongside data from the WFD Main Survey -- i.e., the data products described by \citeds{lse-163} -- are preliminarily described.

Generally, Rubin Observatory might incorporate Special Programs data into the data products for the WFD Main Survey whenever this is (1) possible and (2) scientifically beneficial.
{\bf Decisions about when and whether to include Special Programs data in the WFD Main Survey's data products are left to the discretion of the Rubin Operations Data Production and System Performance teams.}

\subsubsection{Prompt}

It would be both possible and scientifically beneficial to include all standard and alternative standard visits from Special Programs in Prompt processing and Alert Production, alongside the standard visits from the WFD Main Survey.

Non-standard visits that can be processed with the Prompt pipeline in accordance with the relevant requirements might also be included.

Very short and very long exposure times might be excluded if they would need specialized algorithms or templates\footnote{If a Special Program's science goals do require specialized templates and Prompt processing, the DMS will have the capability to load and use an alternative template for some regions, based on the image metadata. However, there would not be enough memory to hold alternative templates for the whole sky region.}.
Visits in sky regions that Rubin had not previously observed would not be able to contribute to Alert Production until a template could be generated.

The data products from Special Programs images would thus contribute to all of the Prompt data products described in Section 3 of the DPDD \citedsp{lse-163}. 
Images and catalog rows would be tagged with a program identifier indicating their Special Program of origin, so that the associated data products can be discovered and extracted by users.

There are two potential issues which might complicate Prompt processing for Special Programs which obtain a sequence of images without slewing, such as the Deep Drilling Fields.
(1) As alert packets contain the full records of all associated {\tt DIASources} from the past 12 months, alerts for {\tt DIAObjects} in the DDFs might become very large.
(2) The association of {\it new} {\tt DIASources} into {\tt DIAObjects} will be somewhat compromised for a DDF sequence.
The processing for the second image begins when the processing for the first image is only halfway done -- when the {\tt DIAObject} catalog has not yet updated with the new {\tt DIASources} detected in the first image.
Thus, the {\tt DIASource} from images one and two for a new transient would not be associated with a single {\tt DIAObject}, but instead would each instantiate a new {\tt DIAObject}.

As a side note, no User-Generated pipeline may contribute Alerts to the LSST Alert Stream, and since the latency on image availability is ``within 24 hours" ({\tt L1PublicT}\reqparam{L1PublicT}) no User-Generated pipeline would be able to obtain and process Special Programs data on a timescale similar to the Alert Stream (60 seconds, {\tt OTT1}\reqparam{OTT1}).

\subsubsection{Solar System Processing}

Since Solar System Processing takes \texttt{DIASources} as input, so any Special Programs images that can be run through the Alert Pipeline can also be incorporated into Solar System Processing.

\subsubsection{Data Release}

It would be both possible and scientifically beneficial to include all standard and alternative standard visits from Special Programs in {\it some aspects} of the Data Release processing -- such as the repository of processed visit images and the {\tt Source} and {\tt Forced Source} catalogs -- alongside the standard visits from the WFD Main Survey.
Images and catalog rows would be tagged with a program identifier indicating their Special Program of origin, so that the associated data products can be discovered and extracted by users.

However, some of the core LSST science goals require a WFD Main Survey data products of {\it uniform depth}.
Whether and how to include any Special Programs data in the deep image coadds and the corresponding {\tt Object} catalog is left to the discretion of the appropriate Rubin teams in Operations.

For example, perhaps Special Programs images will only be included when they bring additional area to the same depth as the rest of the WFD Main Survey, or when they suppress edge effects or low-order modes in the all-sky photometric solutions.


\subsection{Anticipated Rubin- and User-Generated Special Programs Data Products}\label{ssec:proc_spdp}

As mentioned in Section~\ref{ssec:proc_reqs}, Rubin Observatory and the LSST system are required to process Special Programs data to produce unique and separate data products whenever the original or reconfigured versions of the LSST Science Pipelines can be used.

Cases where the development of new algorithms, or the allocation of significant additional computational resources, are required in order to produce unique and separate data products will require User-Generated processing.

The anticipated divisions between what will be Rubin- versus User-Generated for the variety of anticipated ``unique and separate" Special Programs data products, which are designed to meet the specific science goals of the Special Programs, are {\it preliminarily} described.
{\bf Decisions about when and whether Rubin Observatory will generate Special Programs data products are left to the discretion of the Rubin Operations Data Production and System Performance teams.}

For Special Programs which will require User-Generated pipelines and data products, note that users will have access to all of the LSST Science Pipelines and its data processing infrastructure, as well as dedicated computational resources next-to-the-data (via the Rubin Science Platform; \citeds{lse-319}).
Details of the planned ``User Batch" facility for data processing are described in \citeds{dmtn-202}.
However, very computationally intense processing might require external resources. 

Furthermore, it is expected that some User-Generated pipelines and data products might be ``adopted" or ``federated" into the LSST Science Pipelines and the Prompt and Data Release data products. 
Details regarding this process are to be provided elsewhere.

Appendix~\ref{sec:spcs} provides more detailed, step-by-step data processing examples for some potential Special Programs as further illustration. 

\subsubsection{Deep Drilling Fields}

As the DDFs will likely be observed with standard or alternative standard visits, it is likely that Rubin Observatory will be able to reconfigure existing pipelines to process the DDF data and produce ``unique and separate" DDF data products.

Rubin-generated data products might include:\\
 - nightly-coadded images (24 h)\\
 - nightly-coadded difference images (24 h)\\
 - deeply-coadded images (all images to date; yearly)\\
 - templates for the nightly-coadded difference images (yearly)\\
 - Source- and Object-like catalogs for the night and deep coadds (yearly)\\
 - DIASource- and DIAObject-like catalogs for the nightly-coadds (24 h)\\

User-Generated data products might include, for example, DDF images coadded on other timescales, or using algorithms outside of the LSST Science Pipelines.

\subsubsection{Short-Exposure Surveys}

Short-exposure images obtained during twilight, which will have a very bright sky background unlike other LSST images, might require User-Generated algorithms to subtract the high sky background. 
Short-exposure images obtained during the night would have fewer stars for astrometric and photometric calibration, and might require User-Generated processing pipelines.

If the Rubin Operations team decides that the LSST Science Pipelines can be reconfigured and used for short exposures, then the Rubin-generated data products would likely be similar to the Data Release data products (Section 4, \citeds{lse-163}).
For example, a separate repository of processed visit images and coadded images, and separate (but joinable) {\tt Source}, {\tt Forced Source}, and {\tt Object} catalogs.

\subsubsection{GW Target-of-Opportunity}

As the science goals of any target-of-opportunity program are the immediate processing and discovery of new transient phenomena, it is very likely that all ToO programs would be designed to use standard or alternative standard visits and Prompt processing data products such as Alerts.
Prompt processing would proceed automatically for all ToO for which an LSST template image exists.

Options for Rubin Observatory to assist with or expedite the processing of ToO programs, especially during the first year of Operations when the template coverage will be low, are discussed in more detail in \citeds{rtn-008}.

Other science-driven data products from ToO programs, such as custom deep stacks and difference images to find $<$5$\sigma$ transients in the fields on intermediate timescales in cases where the optical counterpart was not detected, would require User-Generated processing.

\subsubsection{Micro-Surveys}

Potential micro-surveys include additional sky areas, in some cases covered to a depth that is different from the WFD Main Survey, such as the proposed micro-surveys for the Roman bulge field; deeper $g$-band imaging of 10 local volume galaxies; the SMC; the Carina nebula; the northern stripe; and the Virgo Cluster.

In all of these cases, if the Rubin Operations team decides that the LSST Science Pipelines can be reconfigured and used for the visits associated with these micro-surveys -- which is likely if they're using standard or alternative standard visits --  then the Rubin-generated data products would be similar to the Data Release data products (Section 4, \citeds{lse-163}).
For example, separate repositories of processed visit images and coadded images, and separate (but joinable) {\tt Source}, {\tt Forced Source}, and {\tt Object} catalogs would be created for all of these micro-surveys.

In cases where data for the micro-survey is obtained throughout the year, then the separate data products would also be produced on a yearly basis.
In cases where the data might be all obtained within a week (e.g., Carina nebular survey), such data products would likely be generated on a shorter (intermediate) timescales. 

Micro-surveys with time-domain science goals that aren't met by the Prompt pipelines -- for example if they require difference imaging with coadded images on an intermediate timescale (e.g., a weekly stack) --  might require User-Generated processing.


\clearpage
% % % % % % % % % % % % % % % % % % 
\section{Enabling the Discovery and Analysis of Special Programs Data Products}\label{sec:analysis}

\subsection{Anticipated Rubin Science Platform Capabilities for Special Programs }

\begin{itemize}
\item discoverability of SP data when browsing an all-sky map
\item to query data by tag of WFD, WFD mini-survey, or Special Program (e.g., DDF field, micro-survey identifier)
\end{itemize}




% % % % % % % % % % % % % % % % % %
\clearpage
\bibliography{local,lsst,refs,books,refs_ads}


% % % % % % % % % % % % % % % % % %
\clearpage
\appendix

\section{Documentation Review for Requirements Related to Special Programs}\label{sec:docrev}

Below are described mentions of Special Programs in Rubin documentations, with a focus on requirements and specifications.

Updates to Rubin documents related to Special Programs -- motivated by past versions of this DMTN -- were made via LCR-1309 and LCR-2265. 

% LCR 1309: https://project.lsst.org/groups/ccb/node/2383
% LCR 2265: https://project.lsst.org/groups/ccb/node/4036


\subsection{Science Requirements Document (SRD)}

Version 5.2.4 (revision 2018-01-30), \citedsp{LPM-17}.

Section 3.4 ``The Full Survey Specifications" states the SRD's assumption that 90\% of the total available survey time would be spent on the main survey, and that the remaining 10\% would be spent {\it ``to obtain improved coverage of parameter space ... [or to] observe special regions"}.



\subsection{LSST System Requirements (LSR)}

Version 7.1 (revision 2020-03-05), \citeds{LSE-29}.

Note that Version 5 (2018-06-26) was an update for LCR-1309, which added requirements, specifications, and discussions regarding the processing of Special Programs data based on earlier versions of this DMTN.

In Section 1.5.1.3, ``Processing Data from Special Programs", LSR-REQ-0122\lsrreq{0122} is a requirement that the LSST system {\it ``shall deliver unique and separate data products for visits from Special Programs"} whenever possible, and that they {\it ``shall be delivered on timescales intermediate"} to the Prompt and Data Release timescales {\it ``when this enables the intended science of the Special Program"}.
The discussion clarifies that {\it ``the term 'whenever possible' includes cases where the Data Management System can run original or reconfigured versions of existing pipelines, and excludes cases where the development of new algorithms, or the allocation of significant additional computational resources, are required"}.

In Section 2.4.1.1.2, ``Non-Standard Visit", LSR-REQ-0111\req{0111} requires that the LSST system {\it ``be capable of obtaining and processing exposures not taken in a standard visit mode including those with a minimum exposure time of"} 1 second (\reqparam{minExpTime}).
The discussion notes that {\it ``non-standard visit exposures may possibly be degraded in some aspects of performance (e.g. cosmic ray rejection on visits consisting of a single exposure), and might be incompatible with difference imaging and alert production (e.g., short exposures in which the PSF is
not fully formed)"}.

The above requirement indicates that the LSST system shall be able to process non-standard visits from Special Programs, but that the image quality might be degraded.
Improvements that require algorithms or processing outside of what the LSST Science Pipelines can provide would be left to the science community and require User-Generated pipelines and data products.

The requirement in Section 1.5.1.3 is echoed in Section 2.6.1.1, ``Organization of Data Products", in which LSR-REQ-0032\lsrreq{0032} is a requirement that the data processing system provide the means for three 'classes' of data products on different timescales (Prompt, Data Release, and User-Generated), and also to provide a means for processing Special Programs data because the {\it ``science goals of Special Programs may require that their processed data products be made available in an additional fourth class, and possibly with intermediate timescales"}.

In Section 2.6.1.1.3, ``Level 3 Data Products", LSR-REQ-0041\lsrreq{0041} specifies that the LSST system {\it ``shall support"} User-Generated data products.
The discussion clarifies that ``{\it there will be technical limits on DM's ability to meet this requirement, such as cases where an intensive amount of additional computational resources is required, because only ~10\% of the total computational system is allocated for user processing"}.
This level of support applies also to user processing of Special Programs data.
See also the reference to LSR-REQ-0055 below.

Section 2.6.1.1.4, ``Data Products for Special Programs", LSR-REQ-0121\lsrreq{0121} specifies that the LSST system {\it ``shall produce unique and separate Data Products as the result of processing data from Special Programs whenever possible, on a timescale that enables the intended science goals of the Special Program.
The cumulative size of the online Special Programs data products shall be no more than ~10\% of the size of the DRP data products from the most recent data
release"}.
The discussion clarifies that {\it ``the term 'whenever possible' includes cases where the Data Management System can run original or reconfigured versions of existing pipelines, and excludes cases where the development of new algorithms, or the allocation of significant additional computational resources, are required.
The cumulative size of the Special Programs data products is capped at ~10\% of the most recent DR because this matches the expected fractional survey area of Special Programs compared to the main survey"}.

In Section 2.7.1.6, ``Community Computing Services", LSR-REQ-0055\lsrreq{0055} requires that the LSST system {\it ``shall provide and maintain an amount of computing capacity equivalent to at least"} 10\% (\reqparam{userComputingFraction}) {\it ``of the total LSST data processing capacity (computing and storage) for the purpose of scientific analysis of LSST data and the production of"} User-Generated data products. 
The discussion clarifies that the scope of this service remains to be determined.
This level of computational resources includes user processing of Special Programs data.

In Section 3.1.3.1, ``Survey Time Allocation", LSR-REQ-0075\lsrreq{0075} requires that the {\it ``survey performance requirements shall be met utilizing approximately 90\% of the historically available observing time, leaving the remaining time available for yet to be defined special programs"}.

In Section 1.3.1.3, LSR-REQ-0124\lsrreq{0124}, the discussion specifies that an image quality parameter related to ellipticity applies only to main survey data.



\subsection{Observatory System Specifications (OSS)}

Version 19.1 (revision 2021-07-30), \citeds{LSE-30}.

Note that Version 13 (2018-06-26) was an update for LCR-1309, which added requirements, specifications, and discussions regarding the processing of Special Programs data based on earlier versions of this DMTN.

In Section 2.2.3.1, ``Standard Operating States", OSS-REQ-0044\ossreq{0044} specifies that {\it ``the LSST observatory system shall be designed and constructed to support ... manual observing - used for specific non-scheduler driven observing to support system verification and testing or specialized science programs"}.
Although most Special Programs will be executed via the survey scheduler as part of {\it ``fully automated observing"}, manual observing might be necessary for, e.g., target-of-opportunity Special Programs.

Section 3.1.5.1.2, ``Data Products Handling for Special Programs", OSS-REQ-0392\ossreq{0392} is essentially a flow-down of requirements from the LSR (0122, 0075, and 0121), and specifies that {\it ``the handling of data products from Special Programs shall be compliant with the approach defined in LSE-163"}.

In Section 3.2.5.3, OSS-REQ-0403\ossreq{0403} is a flow-down of LSR-REQ-0124\lsrreq{0124} related to the ellipticity correlation function distribution.

In Section 3.6.1.3, ``Continuous Exposures", OSS-REQ-0319\ossreq{0319} requires that {\it ``The Observatory shall be capable of continuous operation throughout a night with the interval between successive visits equal to the FPA readout time"}.
The discussion clarifies that {\it ``this mode of observing is needed to support observations when the telescope is not being re-pointed"}, such as deep drilling fields or other Special Programs.

In Section 3.6.1.4, ``Minimum Exposure Time", OSS-REQ-0291\ossreq{0291} specifies that {\it ``the camera shall be able to obtain a single exposure with an effective minimum exposure time of no more than"} 1 second (\reqparam{minExpTime}) {\it ``with a goal of an effective minimum exposure time of"} 0.1 seconds (\reqparam{minExpTimeGoal}). 
The discussion clarifies that {\it ``if the exposure is shortened from the 15 second nominal, the spacing between successive exposures should be extended to maintain the average readout rate consistent with a 15 second exposure"}, which may increase the overheads of Special Programs using short exposure times.
The discussion also clarifies that {\it ``if the exposure is lengthened from the 15 second nominal, the thermal stability may be affected, which may affect photometric accuracy"}.

In Section 3.6.1.5, ``Publish Visit Type", OSS-REQ-0384\ossreq{0384} specifies that {\it ``the OCS [Observatory Control System] shall configure the [Data Management System] DMS (in particular Prompt Processing) with the type of visits to be processed: Standard, Alternate, or a specific type of Non-Standard"}.
The discussion clarifies that this allows the Prompt processing pipeline to be reconfigured on-the-fly in order to incorporate non-standard visits from, e.g., Special Programs.
The time required for reconfiguration might introduce some latency or cause some images to not be processed by the Prompt pipeline.

In Section 3.6.2.1.2, ``Maximum time for operational filter change", OSS-REQ-0293\ossreq{0293} specifies that {\it ``the camera system shall provide the capability of changing the operational filter with any other internal filter in a time less than"} 120 seconds (\reqparam{tFilterChange}).
This would impose a large overhead on, e.g., a Special Program that changes filters often without slewing.
See also OSS-REQ-0295\ossreq{0295}, Appendix~\ref{sec:hardbounds} of this document, and/or the filter change memorandum (\url{ls.st/spt-494}), for more information about the total lifetime number of filter changes.

In Section 3.6.3.3, ``Rotator tracking Time, OSS-REQ-0301\ossreq{0301} specifies that {\it ``the LSST shall be able to maintain field rotation tracking over a period of at least"} 1 hour (\reqparam{rotTrackTime}).
The discussion clarifies that this {\it ``is driven by the need to conduct extended 'deep drilling' observations on a single field}.
There do not seem to be any constraints on the speed of the rotator or the minimum distance between successive visits.

In Section 3.6.3.10, ``Non-Sidereal Tracking", OSS-REQ-0380\ossreq{0380} specifies that {\it ``the LSST system shall be capable of tracking in an arbitrary direction on the sky along a parametric RA(t) and DEC(t) trajectory, at angular rates of up to"} 220 arcseconds per second (\reqparam{nonsiderealAngularRateEl} and \reqparam{nonsiderealAngularRateAZ}) {\it ``with a tracking error not to exceed"} 0.5 arcseconds per minute (\reqparam{nonsiderealTrackingError}).
The discussion notes that {\it ``this is standard capability for modern telescopes"}, but might be relevant to some Special Programs.

%%% MLG: this does not appear in OSS V13
% $\bullet$ OSS-REQ-0027 requires that the scheduling system be able to optimize over at least \texttt{nSciProp} = 6 "science proposals", where these "proposals" are observing targets/constraints such as the distribution of filters, the astronomical conditions, and relative priority (OSS 2.1.1.2, Multiple Science Programs). {\bf JIRA ticket DM-12579 confirmed that there is no maximum number, and so many Special Programs will be able to be included in the scheduler}.

%%% MLG: this does not appear in OSS V13
% $\bullet$ OSS-REQ-0381 requires that the schedule be able to handle targets of opportunity, which would be relevant for e.g., Special Programs for gravitational wave follow-up (OSS 2.1.1.7, Visit Optimization).

%%% MLG: these appear in OSS V13, but aren't as relevant to DMTN-065 anymore
% $\bullet$ OSS-REQ-0189 and OSS-REQ-0190 set the minimum number of raw exposures to be supported as \texttt{nRawExpNightWinterAvg} = 1960 per night on average (but up to \texttt{nRawExpNightMax} = 2800 per night if e.g., two hours of a short-exposure twilight mini-survey are included) and \texttt{nRawExpYear} = 5.5$\times10^5$ per year, respectively. These numbers are set by predicting the maximum number of exposures that would be acquired on the longest night of the year in WFD cadence with 2 second slews, assuming $\sim80\%$ completion, but adding a 10\% margin. These estimates appear adequate for Special Programs in general.

%%% MLG: these appear in OSS V13, but aren't as relevant to DMTN-065 anymore
% $\bullet$ OSS-REQ-0194 and OSS-REQ-0323 set the minimum number of calibration exposures to be supported as \texttt{nCalibExpDay} = 450 on average and \texttt{nCalExpYear} = 1.5$\times10^5$ per year, respectively. These are \textit{minimums}, and so if a Special Program requires additional exposures, this should be possible to accommodate.



\subsection{Data Management Subsystems Requirements (DMSR)}

Version 9 (revision 2021-02-12), \citeds{LSE-61}. 

Note that Version 8.3 (2020-05-04) was an update for LCR-2265, which updated requirements, specifications, and discussions regarding the processing of Special Programs data based on earlier versions of this DMTN.

In Section 1.2.3, ``Raw Science Image Metadata", DMS-REQ-0068\dmreq{0068} specifies that {\it ``for each raw science image, the DMS shall store image metadata"} including {\it ``Program metadata (identifier for main survey, deep drilling, etc.)"}.
The discussion clarifies that {\it ``the program metadata should be sufficient to associate an image with a specific Special Program so that DMS-REQ-0320 and DMS-REQ-0397 can be satisfied"}.

In Section 1.3.13, ``Alert Content", the discussion for DMS-REQ-0274\dmreq{0274} explains that the {\it ``program and/or scheduler metadata"} included in an alert packet {\it ``should be sufficient to identify whether the image is associated with a Special Program (such as an in-progress Deep Drilling Field)"}.

In Section 1.4.18.1, ``Produce All-Sky HiPS Map", the discussion for DMS-REQ-0379\dmreq{0379} raises the point that generating separate HiPS maps for Special Programs (e.g., DDFs) remains an open question.

In Section 1.4.18.5, ``Produce MOC Maps", DMS-REQ-0383\dmreq{0383} specifies that Data Release processing {\it ``shall include the production of Multi-Order Coverage maps for the survey data"}, and that {\it ``additional MOCs SHOULD be produced to represent special-programs datasets"}.
It is noted that a separate technical note would be created to define these MOCs.

The bulk of the DMS's requirements related to Special Programs are in Section 1.6 of the DMSR.

In Section 1.6.1, ``Processing of Data From Special Programs", DMS-REQ-0320\dmreq{0320} specifies that {\it ``it shall be possible for special programs to trigger their own data processing recipes, during the night instead of the nightly Alert Processing (but the recipes may still issue Alerts), or on alternative timescales"}.
The discussion clarifies that the {\it ``LSST will provide these recipes ... when
possible, which includes cases where DM can run original or reconfigured versions of existing pipelines, and excludes cases where the development of new algorithms, or the allocation of significant additional computational resources, are required. An example of an alternative timescale is a nightly trigger to coadd all the deep-drilling field images. Decisions about which recipes are applied to which Special Programs will be made by the Operations team, after consideration of the scientific goals, computational resources, and data rights policy"}.
This requirement is derived from OSS-REQ-0392, which is essentially a flow-down of requirements from the LSR (0122, 0075, and 0121).

In Section 1.6.2, ``Prompt/DR Processing of Data from Special Programs", DMS-REQ-0397\dmreq{0397} specifies that {\it ``it shall be possible for special programs data to be processed with the prompt and/or annual-release pipelines alongside data from the main survey"}.
The discussion further clarifies that {\it ``the data from Special Programs should only be included ... when it is (a) possible ... to do so without additional effort and (b) beneficial to the LSST's main science objectives. Decisions about which data are included ... will be made by the Operations team"}.
This requirement is also derived from OSS-REQ-0392, which is essentially a flow-down of requirements from the LSR (0122, 0075, and 0121).

In Section 1.6.3, ``Level 1 Processing of Special Programs Data", DMS-REQ-0321\dmreq{0321} specifies that {\it ``all [Prompt] processing from special programs shall be completed before data arrives from the following night's observations"}.
This is essentially adding a quantifier to DMS-REQ-0397, to specify that {\it ``when it is (a) possible ... to do so"} means when it is possible to complete the processing before the next night's observations.
This requirement is also derived from OSS-REQ-0392, which is essentially a flow-down of requirements from the LSR (0122, 0075, and 0121).

In Section 1.6.4, ``Constraints on Level 1 Special Program Products Generation", DMS-REQ-0344\dmreq{0344} specifies that {\it ``the publishing of [Prompt] data products from Special Programs shall be subject to the same performance requirements of"} 24 hours (\reqparam{L1PublicT}) for the release of Prompt data products and 1 minute (\reqparam{OTT1}) for the transmission of Alert packets.
This is essentially a more detailed version of DMS-REQ-0321 which includes the Alert production timescale.
This requirement is also derived from OSS-REQ-0392, which is essentially a flow-down of requirents from the LSR (0122, 0075, and 0121).

In Section 1.6.5, ``Special Programs Database", DMS-REQ-0322\dmreq{0322} specifies that {\it ``data products for special programs shall be stored in databases that are distinct from those used to store standard [Prompt] and [Data Release] data products"} and that {\it ``it shall be possible for these databases to be federated ... to allow cross-queries and joins"}.
This requirement is also derived from OSS-REQ-0392, which is essentially a flow-down of requirements from the LSR (0122, 0075, and 0121).

In Section 4.1.16, ``Level 2 and Reprocessed Level 1 Catalog Access", DMS-REQ-0313\dmreq{0313} specifies that {\it ``the DMS shall maintain ... versions
of the most recent catalogs generated from Special Programs data"}.
As with all LSST data, {\it ``there is no requirement for older data releases to be queryable"}.



\subsection{Data Management Applications Design (DMAD)}

Version 4.3 (revision 2020-11-10), \citeds{LDM-151}.

The DMAD is not a requirements document.
Instead, it describes the scientific design of the LSST Science Pipelines: the algorithms and software that will be implemented to meet the requirements for processing the LSST data. 

Special Programs are only mentioned a few times, either as a potential source of single-snap visits or as a potential source of reference images or catalogs (e.g., training sets).

As described above (e.g., LSR-REQ-0122), the LSST system shall deliver unique and separate data products for visits from Special Programs whenever this (1) enables the intended science of the Special Program and (2) can be accomplished using the original or reconfigured versions of the LSST Science Pipelines.
For cases in which the development of new algorithms or the allocation of significant additional computational resources are required to produce Special Programs data products, User-Generated pipelines and processing will be necessary.

The DMAD can be used as the reference document to decide whether a given Special Program will require User-Generated pipelines and processing.

%%% MLG: the following are all from <2018 and were not updated in 2022

% \subsubsection{Extreme-Depth CoAdds} The system has been sized to hold $\sim200$ exposures in memory at once, which defined by the current maximum number of visits per field in the WFD main survey in $10$ years (from a conversation with K.-T.~Lim). Note that the panchromatic CoAdds would be built from the individual filter CoAdds, so the algorithm does not need to handle $\sim800$ images. From a computational standpoint, $200$ is the maximum number of images that can be stacked with an algorithm that requires all images to be accessible in memory at once (i.e., loading all images and calculating the median for each pixel). Deeper stacks might be possible with algorithms that deal with images consequentially. It is conceivable that a Special Program which needs to stack $>200$ images is not possible to accomplish with reconfigured pipelines, and would have to be processed with external, user-contributed resources. However, the exact DM capabilities in this area are not yet well known because NCSA has not yet defined the machine capabilities. Furthermore, the planned commissioning data will go to a $\sim20$ year depth, and so it can reasonably be expected that DM will have to be able to accommodate at least a stack that deep.

% \subsubsection{Deblending} The deep deblender algorithm described in Section 5.3.3 will, out of necessity, be optimized for use in the bulk of the WFD main survey. It may or may not end up being appropriate for use in the Galactic Plane mini-survey area, depending on the science goal. Level 3 deblenders for specific Special Programs fields may require development by the user community.

% \subsubsection{Variability Characterization} The periodic and aperiodic variability characterizations described in Section 6.21 of the DMAD are placeholders, but are representative of what is likely to be implemented: algorithms that are applicable to a broad range of variability types. From DM's perspective, all that is needed is sufficient information to enable relatively useful filters, from which the downstream broker/user can do additional filtering, and these parameterizations might not be sufficient for all science goals. It is conceivable that the goals of a particular Special Program might require different algorithms; these could be provided by DM, or written as Level 3 and either made joinable to the DM reconfigured data products or perhaps incorporated directly.

% \subsubsection{Photometric Redshifts} As described in Section 5.6.5 the Level 2 DRP \texttt{Object} catalog will include a photometric redshift, but this algorithm will be produced by the science community and then adopted and run at scale by DM. It is conceivable that the photo-$z$ algorithm for a Special Programs data set, such as a deep drilling field, might be different from that used for the WFD main survey.



\subsection{Data Products Definitions Document (DPDD)}

Version 3.6 (revision 2021-12-17), \citeds{LSE-163}.

Section 6 describes the data products for Special Programs.
The DPDD is not a requirements document; Section 6 summarizes the requirements presented above and does not introduce any new constraints or new information about Special Programs. 

%%% MLG commented out the paragraphs below in 2022, they were out of date.

% $\bullet$ The database schema for \texttt{DIASource} does not appear to have an element that identifies which template image that was used. This will be needed for both Levels 1 and 3 differencing pipelines and products, for both Special Programs and WFD main survey data. DMS-REQ-0074 already does require that the identity of input exposures is stored for each difference image (LSE-61). In conversation with K.-T., we find that this isn't a problem, as it will be handled by provenance: the code configuration used and the time of processing are sufficient to identify and regenerate the template image. However, K.-T. has pointed out that the capability to regenerate the \textit{exact same} template -- the pixels that were subtracted -- is not a current deliverable. However, the stamp of the difference image will live on in the Alerts database, so we also do not foresee this as a problem.

% $\bullet$ The DMAD specifies that externally defined targets can be incorporated into the \texttt{Objects} catalog (Section 3.2.5), and this may be a particular interest to Special Programs. It is unclear how such targets will be identified or flagged as such in the database schema, and whether we need to add an element for this. Currently, the \texttt{Object} database contains an element \texttt{prv\_inputId} which is an \texttt{integer}, and is described as the \textit{"Pointer to prv\_InputType. Indicates which input was used to produce a given object."} Is that all we need? {\bf JIRA ticket DM-12580 clarified that  \texttt{Object.prv\_inputID} in the database schema is one possible way to identify whether an \texttt{Object} is an externally provided coordinate}.

% $\bullet$ The \texttt{Object} and \texttt{DIAObject} elements that have been reserved for variability characterization, as described in the DMAD and the DPDD, are as follows: \\
% \texttt{Object} and \texttt{DIAObject.lcPeriodic} = \texttt{float[6 x 32]} = Periodic features extracted from light-curves using generalized Lomb-Scargle periodogram \\
% \texttt{Object} and \texttt{DIAObject.lcNonPeriodic} = \texttt{float[6 x 32]} = Non-periodic features extracted from light-curves using generalized Lomb-Scargle periodogram \\
% Section 6.21 of the DMAD describes the nominal algorithms to define these parameters, but as we mentioned in Section \ref{ssec:docrev_dmad}, different kinds of variability might be measurable Special Programs cadences that are quite different from the WFD main survey. Are 32 floats in each of the 6 filters always going to be a large enough volume? The Transients and Variable Stars Science Collaboration currently has a Task Force assigned to address this, and {\bf it is the topic of JIRA ticket DM-12581}.



%%% MLG commented out this whole section in 2022, since LSE-180 is from 2013
%%%  LSE-180 is deprecated?

% \subsection{Photometric Calibration for the LSST Survey, \citeds{LSE-180}}

% LSE-180 is built on \texttt{OpSim} runs that do include some nominal DDF, but the photometric calibration investigated in this work does not much deal with potential issues induced by non-standard visit patterns or exposure times of Special Programs, as its scope is the WFD main survey. Potential issues with DM processing -- including calibrations -- of non-standard visit exposure times is raised in Section \ref{ssec:dmplans_NSV}. Regarding the reduction and calibration of non-standard visit images, LSE-180 makes two relevant points: \\
% $\bullet$ In LSE-180, it is assumed that all factors affecting the system transmission are stable on 15 second timescales (page 10), but not what the upper limit of that might be. \\
% $\bullet$ LSE-180 comments on the dither pattern for the WFD survey in that "dither patterns where the overlap is one quarter of the field of view or more produce results meeting the SRD requirements", but this is specific to photometric calibration of the WFD. The LSE-180 also mentions that an inappropriate dither pattern can make it hard to correct for the variation of system bandpass as a function of the focal plane position -- but so long as this is solved in the WFD, the corrections can be applied to the much smaller amount of data from the Special Programs.

% However, Lupton's recent work on calibrations has made much of LSE-180 obsolete, and it is not clear whether it is still needed as a separate document. At the time of writing, Lupton's most recent take on calibrations can be found in \url{https://github.com/lsst-dm/calibration/blob/master/calibration.pdf}. K.-T. referenced the Appendix B for the overview of calibration types, but it seems that there is not yet a final plan that can be assessed for its suitability for Special Programs. It is plausible that Special Programs could have their own calibration database, and obtaining additional calibration frames is not expected to cause a bottleneck (at the moment there is $\sim1.7$ hours per day for this, and exceeding this could eat into the engineering time). The most likely cause for trouble involving calibrations and Special Programs is the computational time needed to create the additional and/or different calibration files to be applied to non-standard Special Programs data, and/or the additional overhead required to load them in the Level 1 pipeline if the schedule interleaves SP-WFD-WFD-SP.

% {\bf JIRA ticket DM-12582 is currently open, and aims to define the potential additional calibration needs of Special Programs data.}


\section{Potential Hardware Boundaries on Data Diversity}\label{sec:hardbounds}

The potential boundaries on the diversity of data products that could be imposed by limitations from the Rubin Observatory hardware -- camera, telescope, and/or site -- are considered.

\subsection{Filter Changes}
The maximum time for filter change is 120 seconds: 30 seconds for the telescope to reorient the camera to its nominal zero angle position on the rotator, and 90 seconds to the camera subsystem for executing the change (OSS-REQ-0293; \citeds{LSE-30}).
Assuming that most Special Programs would be designed to keep overheads $<$100\% and would be using standard 30 second visits, the filter change time indicates that it is likely that at least 4 exposures in a given filter would be obtained between filter changes, but this is not actually a hardware boundary. 
The filter change mechanism is designed to undergo a total of 100000 changes over its lifetime, and each filter is designed to support up to 30000 changes over its lifetime, where lifetime is 15 years.
That is an average of $\sim$27 changes per day, some of which would occur in they day during calibrations (estimate, $\sim$10) and the rest at night.
As stated in the filter change memorandum (\url{ls.st/spt-494}), {\it ``the system could support as many changes involving the 5 filters loaded in the carousel as desired, without any practical limitation"}.

\subsection{Filter Carousel Loads}
The filter carousel can hold five of the six LSST filters at a time.
The system is designed to support $3000$ loads in $15$ years (\url{ls.st/spt-494}).
Filter loads are only done in the day, and there will never be data in more than five filters in a given night.

\subsection{Exposure Times}
The minimum exposure time is $1$ second, with a stretch goal of $0.1$ seconds (OSS-REQ-0291; \citeds{LSE-30}).
The maximum exposure time is not restricted.

\subsection{Readout Time}
The readout time is $2$ seconds, and would be significant overhead on short exposures.

\subsection{Inter-Image Time}
Images with exposure times $<15$ seconds {\it might} still have to be separated by $15$ seconds for thermal tolerance; i.e., the minimum readout rate might be one image every $15$ seconds, regardless of exposure time (OSS-REQ-0291; \citeds{LSE-30}).

As discussed in Jira ticket DM-12573, the main issue is thermal and is related to the shutter, both the motors and the brakes; an elevated Camera skin temperature would affect image quality.
As of 2022, early tests suggest that a sustained (30 minutes) sequence that increases the heat load by large factors would not work, but further functional testing of the system once the Camera was fully assembled are needed for full characterization of the issue.

This potential $15$ interval between images is also a potential hardware boundary on the potential diversity of data products.


\subsection{Telescope Slew}
As described in \citeds{Document-28382}, large slews would have considerable overheads, but there are no hardware boundaries on the size of a single slew or the accrued slew distance.

\subsection{Pointing and Sidereal Tracking}
The specifications for the telescope's pointing and tracking in \citeds{LSE-30} indicate that $<$0.2 arcsecond precision in field pointing (OSS-REQ-0302) and $<$1 arcsecond in open-loop tracking (OSS-REQ-0303) would not be possible, but guiding would improve the latter (OSS-REQ-0305).
Furthermore, obtaining the \emph{exact same} alignment of the pixel grid in RA-Dec {\it ``would put demands on the camera rotator that were not planned"}\footnote{As per C. Claver's comments in ticket DM-12573.}.

\subsection{Non-Sidereal Tracking}
The requirement that the LSST system be able to perform non-sidereal tracking is set by OSS-REQ-0380 in \citeds{LSE-30}.
This capability will include angular rates of up to 220 arcseconds per second in both azimuth and elevation. 

\subsection{Camera Rotation}
The requirements on the rotator's capabilities do not set any limits on the per-night or total lifetime rotation (OSS-REQ-0301, -0300; \citeds{LSE-30}) which might put boundaries on the distance between successive visits or the ability to jump between two widely separated fields.
Currently, there are no hardware boundaries imposed by camera rotation constraints on the potential diversity of data products.


\section{Potential Processing Boundaries on Data Diversity}\label{sec:procbounds}

The capability of the LSST Science Pipelines to process diverse data is explored below.

Note that processing boundaries might ultimately be defined not by what is technically possible, but by the resulting image quality parameters (e.g., the number of stars with sufficient flux for photometric calibration).
Furthermore, the processing boundaries might not be fully constrained until the final performance of the LSST Science Pipelines, as described in the Data Management Applications Design, \citeds{LDM-151}) document, is fully characterized.

{\bf Summary of the conclusions below:}\\
Very short ($<$2 sec) exposures could be difficult to reduce due to an incompletely-formed PSF, and very short or very long ($>$150 sec) exposures could be difficult to calibrate due to having too few (or too few unsaturated) stars.
It is currently unclear whether images with very bright sky backgrounds (twilight images) can be processed with the LSST Science Pipelines, or whether user generated pipelines will be needed.
The full reduction and calibration of images obtained with non-sidereal tracking, in which the stars are streaked, is currently beyond the scope of the LSST Science Pipelines, and will require a user generated pipeline.

\subsection{Exposure Times}\label{ssec:procbounds_expt}

Images which deviate significantly from the $15$ second duration for the WFD main survey may encounter issues in the instrument signature removal routine, in the correction for differential chromatic refraction, in the difference imaging analysis pipeline, and/or in the photometric and astrometric calibrations due to a differently sampled set of standard stars per CCD.

\subsubsection{Short Exposures (Non-Standard Visits of $<$30 sec)}
The LSST System Requirements document states that {\it ``The LSST shall be capable of obtaining and processing exposures not taken in a standard visit mode including those with a minimum exposure time of} {\tt minExpTime}", which is 1 second (stretch goal 0.1 seconds; LSR-REQ-0111 in \citeds{LSE-29}).

However, for exposure times there are other considerations, as changing the exposure time also affects the photometric and astrometric calibrations.
Assuming that 1 second exposure can be reduced and calibrated, its detected point sources will span a dynamic range of $r$$\approx$ 13--21 magnitudes.
A template image built on 15 second exposures will saturate at $r$$\approx$15.8 mag, but this still leaves stars between 15.8--21.0 magnitudes to be used in the PSF-matching (and all other filters have a similarly large overlap).

In order for an image to be successfully PSF-matched to the template, the PSF must be well formed (no speckle pattern), and have a spatial variance that the pipeline is capable of modeling (be smoothly varying on some minimal scale).
As a simple demonstration, Figure \ref{fig:expt} shows that perhaps exposure times shorter than $2$ seconds do not have a well-formed PSF (using the centroid of a 2D Gaussian fit as a proxy for "well-formed").

\begin{figure}
\begin{center}
\includegraphics[width=14cm,trim={0cm 0cm 0cm 0cm}, clip]{figures/exptime.png}
\caption{At left, Arroyo atmosphere-only simulated PSF for LSST (with oversampled pixels) with exposure times of 0.5, 2, and 15 seconds (top to bottom), courtesy of Bo Xin. At right, blue and purple lines show the location of the centroid derived from a 2D Gaussian fit to the PSF as a function of exposure time, with the red dashed line showing the true center. We can see that for exposure times greater than 2 seconds, the centroid converges near its true value. \label{fig:expt}}
\end{center}
\end{figure}

% In conversation with DM-AP team members (Reiss, Findeisen, Connolly, Bo) there has not yet been a study of the safe range of exposure times that will be allowed to contribute Alert Production.
% One possibly useful study is Chang et al. (2012), "Atmospheric point spread function interpolation for weak lensing in short exposure imaging data".
% They show that a 15 second exposure contains PSF variability on short spatial scales across a 1 square degree image which, for extragalactic fields with few stars (i.e., but good for weak lensing), is hard to characterize.
% They also present a new software package to do mitigate the effects.
% Software packages \texttt{PhoSim} (Peterson et al. 2015; \citep{2015ApJS..218...14P}) or \texttt{ARROYO} \citep{2004SPIE.5497..290B} could be used to characterize the PSF stability as a function of exposure time.

\subsubsection{Long Exposures (Non-Standard Visits of $>$30 sec)}

There is no maximum exposure time specified for an LSST image.
Given that the template image will be a stack of at least a year or two of data, processing a $5$--$10$ times deeper single image through the difference imaging pipeline should be fine.
However, a $2\times150$ second exposure would saturate at $r \approx 18.3$, perhaps leaving too few stars overlapping with e.g., templates or WFD images, for astrometric and photometric calibrations.
Furthermore, cosmic-ray rejection completeness might be reduced for longer exposures (unknown), which could impact the quality of a difference image and the detected sources.
Additionally, any system qualities that vary on short (but $>30$ second) timescales could inhibit photometric calibration (e.g., tracking).

\subsection{Twilight Images with a Bright Background}

Images obtained during twilight for scientific purposes are also likely to have shorter exposure times, and so the issues described in Section \ref{ssec:procbounds_expt} also apply here.
Whether or not bright-background images can (or shall) be fully processed -- reduced, calibrated, background-subtracted, and delivered with astrometric and photometric solutions -- or whether this will require a user generated pipeline, remains to be determined (see also the example in Section \ref{ssec:SPCS_Twilight}).
This may depend on the exposure time and the number of stars available in the image.

\subsection{Images Obtained with Non-Sidereal Tracking}

Non-sidereal tracking leads to images in which stars are streaked, but the moving object appears as a point source.
Full processing -- providing reduced, calibrated, background-subtracted images that are delivered with astrometric and photometric solutions -- of these images is beyond the scope of the DM pipelines as it would require the development of new algorithms, and will need to be done as a user generated pipeline.
The first steps of such a pipeline, such as Instrument Signature Removal, will probably be possible to achieve by reconfiguring the relevant DM software tasks.


%%% MLG removed the following in Feb 2022, it's no longer a concern.
% \subsection{Number of Exposures per Visit (Long Sequences of a Single Field)}

% There is no processing constraint on the number of consecutive exposures that could be obtained of a single field.
% From a DM perspective, it would be best if these exposures were packaged into visits of no more than 2 exposures per visit, to minimize the need to reconfigure of the pipelines, and because the camera only ``clears" between visits. 

% K.-T. Lim has pointed out that an odd number of exposures is a non-standard visit; two snaps is hardwired into the code. This is baked-in to a configuration so that the pipeline can have a definition of what kind of timing delay constitutes ``late".  Moving away from 2 exposures per visit requires a configuration change to the pipelines, which incurs an overhead (up to 1 minute) -- in fact, K.-T. things that between $10$ and $120$ seconds exposure times can easily be handled by the pipeline (i.e., can be run through ISR using scaled calibration frames), so long as they come in pairs. The real problem is knowing how long the processing should take, and not killing a process that is taking longer because there were 4 snaps in the visit instead of 2. To accommodate non-standard visits requires that the scheduler pass on the information of the number of snaps in the visit (\ref{DMSR-1}). Then the processing pipeline will know to, e.g., not attempt to difference the two snaps in the case were there is an odd number of snaps in a visit. \textit{MLG -- I've heard rumors of a CR regarding alternate standard visits of $1\times30$ seconds, but do not know the status or implications of this.}

% K.-T. has also pointed out that currently, a deep drilling field would be interpreted as a single visit of 50 exposures by the scheduler. One implication of this is that since the camera only ``clears" prior to a new visit, it would not do this for the entire 50-exposure sequence. The processing pipeline would need to know how to divide this sequence up into visits. As there is no current requirement for DM to receive the information that the scheduler is about to do a 50 exposure visit, we need \ref{DMSR-1} to add the proposal ID and the number of exposures per visit to the meta data, and then it should be OK for DM to parse this visit information in the reduction pipeline.


%%% MLG removed the following in Feb 2022, it's no longer a concern for Special Programs (crowded fields are in the WFD).
% \subsection{Images in Very Crowded Fields}

% The LSST pipelines' performance in crowded fields is documented in \citeds{DMTN-077}, which finds that, e.g., in Galactic Plane regions with a source density of $500000$ sources per square degree, the completeness drops to 50\% at $20.2$ magnitudes.
% The slide deck at \citeds{Document-27962} also describes DM's plans for processing crowded fields. These may or may not be appropriate for Special Programs data, depending on the science goals.

\section{Hypothetical Scenarios}\label{sec:spcs}

\textbf{\emph{Hypothetical}} examples of Special Programs
and the standard, Special, and User-Generated processing
to illustrate what \emph{might be done}.

The details of the data acquisition and processing mentioned below are 
\emph{just illustrative examples} of decisions that have yet to be made.

The steps used to describe the hypothetical processing for each case scenario are: \\
Step 1. Data acquisition. \\
Step 2. Standard Prompt processing and Alert Production. \\
Step 3. Special Processing with reconfigured pipelines. \\
Step 4. Standard processing for inclusion in Main Survey data products. \\
Step 5. User-Generated Processing. \\

\subsection{Outer solar system mini-survey}\label{ssec:SPCS_TNO}

This hypothetical Special Programs processing summary is based on the Becker et al. (2011) 
white paper to find outer solar system objects with shift-and stack (SAS) \citedsp{Document-11013}.

Step 1. Data acquisition. \\
In a single night, the 9 adjacent fields in a 3x3 grid are observed with 
$336$ $\times$ $15$ second $r$ or $g$-band exposures (168 standard visits). 
These observations are repeated 2-3 nights later, and then this 2-night sequence
is repeated 3 more times: 1.5 months, 3 months, and 13.5 months later. 
They are not all at the same RA, Dec, but at selected ecliptic coordinates.

Step 2. Standard Prompt processing and Alert Production. \\
Each $2\times15$ second standard visit is processed by the Prompt pipeline 
and alerts are released within 60 seconds.
Within 24 hours, the {\tt DiaSource} and {\tt DiaObject} catalogs are updated
to include the results of Prompt processing of these visits.
After 80 hours, the processed visit images and difference images become available.
All images and sources originating from this Special Program have 
region and observing mode labels, e.g., ``SP-OSSO".

The results of Prompt processing are not very relevant for this Special Program's primary science goal,
which requires a year of dispersed observations before the processing pipelines for shift-and-stack can be run.
However, including these data in Prompt processing means that
they can contribute to LSST's other time-domain and Solar System science goals.

Step 3. Special Processing with reconfigured pipelines. \\
None possible. 
Shift-and-stack processing is beyond the scope of existing algorithms in the LSST Science Pipelines.

Step 4. Standard processing for inclusion in Main Survey data products. \\
Every year, each $2\times15$ second standard visit is reprocessed by the DIA data release pipelines
and the results are included alongside Main Survey data in the relevant DIA data products
(e.g., processed visit images, difference images, associated source catalogs).
In the first year after the Special Program is executed,
Rubin Data Management finds that 10\% of the standard visits from this Special Program
had coordinates and image quality that help improve uniformity of the all-sky coadd,
and so they are included.
In later years, this fraction decreases (remember, this is \emph{hypothetical}).
In all data releases, any and all processed images and catalog sources that originate in visits from this Special Program
have the same region and observing mode labels, e.g., ``SP-OSSO".

Step 5. User-Generated Processing. \\
The User-Generated Processing pipeline running the shift-and-stack processing is be set up and submitted 
for batch processing by the user through the Science Platform or on an external system. 
The pipeline's inputs are the processed visit images (and/or difference images) from Prompt processing.
User-generated custom algorithms then shift-and-stack the images, and then the LSST Science Pipelines
tasks are used to do source detection and characterization and create catalogs.
User-generated custom code derives orbital parameters for the detections, and stores
them in a user-generated catalog with a similar format to {\tt SSObjects}.


\subsection{Deep Drilling Field}\label{ssec:SPCS_SNDDF}

Step 1. Data acquisition. \\
In the COSMOS DDF, the scheduler obtains 10 standard visits in a row in each of the $griz$ filters
with a small dither pattern between visits.
This happens every other night for a three month season for four years.

Step 2. Standard Prompt processing and Alert Production. \\
Same as above, but the observing mode label might be, e.g., ``SP-DDF-COSMOS".

Step 3. Special Processing with reconfigured pipelines. \\
First, a template image of appropriate depth for ``nightly" difference imaging is created.
At the end of each nightly sequence of observations, a pipeline based on recongfigured 
components of the LSST Science Pipelines could be automatically triggered.
This pipeline creates nightly coadds in each filter and runs DIA using the template.
Unique and separate catalogs with the same format
as {\tt DiaObject} and {\tt DiaSource} could be updated within, e.g., 24 hours (not a requirement),
and alert packets could be created and produced (with a flag set to identify them
as longer-latency alerts from nightly stacked DDF images).
Instead of running daily, this processing might run weekly instead (TBD).
At the end of each season, deeply coadded images that include all of the DDF's visits 
from all years are re-generated, along with a separate {\tt Object}-like catalog.
All images and catalogs are stored in separate butler collections and TAP tables from
the Main Survey data products.

Step 4. Standard processing for inclusion in Main Survey data products. \\
Every year, each standard visit is reprocessed by the DIA data release pipelines
and the results are included alongside Main Survey data in the relevant DIA data products
(e.g., processed visit images, difference images, associated source catalogs).
Due to their small dither and lack of rotation, not even a single DDF image 
is used to supplement the Main Survey's all-sky coadd.

Step 5. User-Generated Processing. \\
In order to achieve a secondary science goal of finding very high-$z$ faint supernovae,
a team of users reconfigure the LSST Science Pipelines to create weekly deep coadds
of the COSMOS field an appropriate-depth template image, and to run DIA at the
end of the season.
These data products are stored in separate catalogs with the same format and schema as
the {\tt DiaSource} and {\tt DiaObject} tables that are private to the team.


\subsection{Short-exposure twilight survey}\label{ssec:SPCS_Twilight}

Twilight observations obtained at, e.g., 60 degrees from the Sun, are particularly
well-suited for finding Near-Earth Objects (NEOs).

Step 1. Data acquisition. \\
At a specified time (or e.g., 6 degree twilight), the scheduler begins a dither pattern of 
$2$-second exposures. 
Coordinates and exposure times are set by the Sun distance, sky brightness, and desired saturation limits.

Step 2. Standard Prompt processing and Alert Production. \\
Pending studies of DIA and Alert Production pipeline capabilities to process 
short-exposure, high sky-background images (see Section~\ref{ssec:proc_bounds_processing}).

Step 3. Special Processing with reconfigured pipelines. \\
Pending studies of the LSST Science Pipelines capabilities to process 
short-exposure, high sky-background images (see Section~\ref{ssec:proc_bounds_processing}).

Step 4. Standard processing for inclusion in Main Survey data products. \\
These short-exposure, high sky background images would not contribute to the data products created for the Main Survey.

Step 5. User-Generated Processing. \\
If short-exposure images cannot be processed with the existing DM algorithms, 
user-generated processing would be needed to reduce the raw data, and to 
futher detect and characterize sources in the processed images.





\section{Previously Proposed Special Programs}\label{sec:prevpropsp}

{\bf This section has not been updated since 2018.}

In this section we compile information about the science goals and observational methods for Special Programs that have been previously proposed or discussed in the Science Community. We use these to infer the potential deviations from standard visit images, and to get a basic idea of the DM processing needs that would be required to enable the science. The main resources from which we have collected information about the Community's Special Program are: \citep{2008arXiv0805.2366I}; \citep{LPM-17}; the LSST Deep Drilling Field white papers from 2011\footnote{\url{https://project.lsst.org/content/whitepapers32012}}; presentations by Niel Brandt and Stephen Ridgway at the LSST Project and Community Workshop in August 2016\footnote{\url{https://project.lsst.org/meetings/lsst2016/sites/lsst.org.meetings.lsst2016/files/Brandt-DDF-MiniSurveys-01.pdf} and \url{https://project.lsst.org/meetings/lsst2016/sites/lsst.org.meetings.lsst2016/files/Ridgway-SimulationsMetrics_1.pdf}}; \citep{2013arXiv1304.3455G}; and Chapter 10 of \citep{2017arXiv170804058L}.

So far, only one aspect of the LSST Special Programs are set: the locations of the four chosen deep drilling fields\footnote{\url{https://www.lsst.org/scientists/survey-design/ddf}}. There are three mini-survey areas that have been discussed extensively by the Science Community: the North Ecliptic Spur (NES), the South Celestial Pole, and the Galactic Plane (see Figure 8 of \citep{2008arXiv0805.2366I}). In Table \ref{tab:ddfms} we list the four extragalactic deep drilling fields have already been specified, along with an \textit{incomplete} list of potential mini-surveys that have been openly discussed in the Science Community. In Section \ref{sec:SPCS}, we create detailed DM Processing Case Studies for several of these Special Programs in order to identify any potential issues with reconfiguring the DM pipelines to create specific data products for these programs.

\begin{table}[h]
\begin{center}
\begin{footnotesize}
\caption{Approved DDF and Incomplete List of Potential Special Programs.}
\label{tab:ddfms}
\begin{tabular}{lll}
\hline \hline
Name & Coordinates & Description  \\
\hline
DDF Elias S1    & 00:37:48, -44:00:00  & approved, cadence TBD \\
DDF XMM-LSS & 02:22:50, -04:45:00  & approved, cadence TBD  \\
DDF Extended Chandra Deep Field-South & 03:32:30, -28:06:00  & approved, cadence TBD  \\
DDF COSMOS  & 10:00:24, +02:10:55 & approved, cadence TBD  \\
\hline
North Ecliptic Spur      & & solar system objects (find and characterize) \\
Galactic Plane             & & more intensive stellar surveying \\
South Equatorial Cap  & & S/LMC and more Galactic science \\
Twilight                        & & short exposures (0.1s) for bright stars \\
Mini-Moons                     &  & finding mini-moons \\
Sweetspot                       & & 60 deg from Sun for NEOs on Earth-like orbits \\
Meter-Sized Impactors     & & detection a week before impact \\
GW Optical Counterparts & & search and recovery \\
Old Open Cluster M67      & dec +12 & compact survey above Galactic plane  \\
\hline
\end{tabular}
\end{footnotesize}
\end{center}
\end{table}

Here we consider a variety of scientific fields in turn, the Special Programs that have been discussed in that Science Community so far, and the implications of these Programs for the diversity of data and data products. Generally, the types of LSST Special Programs that are open for proposals include: (i) additional deep drilling fields; (ii) refined observing strategies for deep drilling fields; (iii) optimized survey areas for the NES, South Pole, and Galactic Plane; (iv) refined observing strategies for the NES, South Pole, and Galactic Plane; and (v) additional mini-surveys (areas and observing strategies).

\medskip
\noindent \textbf{A Nominal DDF Observing Strategy -- } Ivezi\'{c} et al. (2008, \citep{2008arXiv0805.2366I}; Section 3.1.2) describes a nominal DDF data set as $\sim50$ consecutive $15$ second exposures in each of four filters, repeated every two nights for four months. Each exposure would have a $5\sigma$ limit of $r\sim24$; the nightly stack would have a limit of $r\sim26.5$; and the final deep stack of all exposures would have a limit of $r\sim28$. This description does not comment on the processing mode, but, depending on the science goals the exposures could be done as either a series of 50 non-standard visits ($1\times15$ seconds) or 25 standard visits ($2\times15$ seconds). 

\medskip
\noindent \textbf{Solar System Objects (SSO) -- } Four of the mini-surveys in Table \ref{tab:ddfms} have science goals related to studies of SSO. Observations of the North Ecliptic Spur area could yield more $\geq140$ m near-earth objects (NEOs) for the final LSST sample (reference: Brandt's talk). The Mini-Moons Mini-Survey aims to find and study the temporarily captured satellites of the Earth (Section 10.2, \citep{2017arXiv170804058L}). The Sweetspot Survey would use twilight fields to find NEOs in Earth-like orbits (i.e., these objects are never in opposition fields, but overhead at sunrise/sunset; Section 10.2, \citep{2017arXiv170804058L}). The Meter-Sized Impactors program would find and track meter-sized impactors $<2$ weeks before impact (Section 10.2, \citep{2017arXiv170804058L}). {\bf Summary:} most of these science goals do not seem to require non-standard visits or exposure times, with the exception of the Sweetspot survey which occurs during twilight and thus may require shorter exposures. The cadence and patterns of these mini-surveys may differ from the WFD main survey, especially when very fast-moving objects are sought. From a processing perspective, it seems that many of these science goals will be achievable by using the products of Solar System Processing, which runs on the Prompt Pipeline's \texttt{DIASource} catalogs after they are updated each night. The exception is finding faint SSOs (e.g., Trans-Neptunian Objects Trojans, asteroids, long-period comets, dwarf planets) through shift-and-stack (SAS) processing \citedsp{Document-11013}, because SAS is not a capability being built within the DM system and cannot be done solely by reconfiguring DM pipelines. An example of user-generated pipeline for SAS is described in Section \ref{sec:SPCS}.

\medskip
\noindent \textbf{Stars in the Milky Way and Magellanic Clouds -- } As described in \citedsp{Publication-141}, mini-surveys of the Galactic Plane can better distinguish faints stars from faint red galaxies by including at least 3 filters of coverage (e.g., $izy$; similar to WFD), and could mitigate losses from proper motion and increase the detection rate of stellar flares by obtaining all the images in short time span (i.e., a more concentrated cadence than the WFD).  As described in \citedsp{Publication-145}, applying the nominal DDF observing strategy over the full area of the Large and Small Magellanic Clouds can characterize stellar variability to $M_V<6.5$ on timescales from 15 seconds to 3 days. For this, special co-adds may be required, e.g., \textit{"to reach variability levels of 0.1 to 0.005 mag will require co-adds depending on the timescale of the particular variables"} \citedsp{Publication-145}. The Twilight survey in Table \ref{tab:ddfms} proposes short exposures to enable bright stars to be put on the same photometric system as the deeper LSST WFD main survey catalog, and enable science that is based on their long monitoring baselines from historical observations. In Chapter 10.4 of \citep{2017arXiv170804058L}, a proposed short-exposure survey of M67 would use the camera's stretch goal of $0.1$ second exposures or, if that is not possible, \textit{"custom pixel masks to accurately perform photometry on stars as much as $6$ magnitudes brighter than the saturation level"}. {\bf Summary:} while some of these science goals can be accomplished with standard visits, MW \& L/SMC science goals are likely to request shorter exposure times, perhaps down to $0.1$ seconds. These science goals are also likely to propose cadence and filter distributions that are significantly different from the WFD main survey. From a processing perspective, the science goals depending on shorter exposures will only be able to be met by reconfiguring the DM pipelines if the short exposures can be shown to successfully be processed (with, e.g., instrument signature removal); the science goals can likely be met with data products in the same format as the Prompt or DR Pipeline (i.e., {\tt Source} and {\tt Object} catalogs, single visits and deep CoAdds). Although it is not mentioned in the above paragraph, the MW \& L/SMC science community is also most likely to require special processing to extract information from saturated stars, which is outside the scope of DM. See Section \ref{ssec:SPCS_GPVSEx} for more detailed DM processing case studies.

\noindent \textbf{Exoplanets -- } As described in Section 3.1.2 of \citep{2008arXiv0805.2366I}, transiting exoplanets could be detected with the nominal DDF plan, which would allow for $1\%$ variability to be detected over hour-long timescales; a DDF field at Galactic latitude $30$ degrees would yield $10^6$ stars at $r<21$ that would have $\mathrm{SNR}>100$ in each single exposure of the sequence. \citep{2013arXiv1304.3455G} describes how transits can be extract from a wider-area survey of the Galactic Plane, and how microlensing candidates can be found with $\sim22$ mag imaging over the Galactic Plane region every 3-4 days (since microlensing events are slower; these would then require follow-up with external facilities). Dealing with the more crowded fields would be mitigated by the shallower images, in this case. One of the main points of \citep{2013arXiv1304.3455G} is that the Galactic Plane can yield a lot of science despite the fact that its eventual deep co-adds would be uselessly confusion limited, and therefore should not be skipped. \textbf{Summary.} Some of these science goals appear possible with standard visit images, and some might request shorter exposures to avoid confusion in crowded fields when the science can be done with brighter stars. From a processing perspective, the science goals are likely to be achievable with reconfigured DM pipelines, but this depends heavily on performance in crowded fields. See Section \ref{ssec:SPCS_GPVSEx} for a more detailed DM processing case study for Galactic Plane regions.

\noindent \textbf{Supernovae -- } The nominal DDF plan described in \citep{2008arXiv0805.2366I}, which builds nightly stacks with a limit of $r\sim26.5$ out of standard visit images, would extend the SN sample to $z\sim1.2$ and provide more densely sampled light curves for cosmological analyses. The optimal exposure time distribution might be 6, 5, 10, 10, 9, 10 in $ugrizy$ \citedsp{Publication-144}. High-cadence observations of DDF would be the only way to detect fast transients, particularly extragalactic novae, some tidal disruption events, optical counterparts to gamma-ray bursts, and peculiar SNe \citep{2014ApJ...794...23D}. Generating the best-possible individual SN light curves for cosmological analyses requires building special, deep-as-possible, SN-free host galaxy images and using them as a template. This will also be necessary for studying SNe that appear in the template image; i.e., that last $>1000$ days. These are mostly Type IIn, probably explosions of massive stars into dense circumstellar material, which are not used for cosmology but rather to study late-stage stellar evolution and mass loss. SN-free images will also be needed to measure correlated properties for cosmology and to do host-galaxy science. The latter, specifically the "characterization of ultra-faint SN host galaxies", is also mentioned in the Galaxies DDF WP \citedsp{Publication-142}. Short-exposure observations of bright, nearby SNe may also be useful to include near-peak photometry in the LSST magnitude system, and enable full light-curve analyses. \textbf{Summary.} All of these science goals appear possible with standard visit images (with the exception of a target-of-opportunity short-exposure program to observe bright SNe). From a processing perspecitve, the science goals appear to be accessible with reconfigured DM pipelines to stack and difference the data. In particular, the DRP codes to create "transient-free CoAdds" will be suitable for generating the SN-free templates for DDF, as they will do for the Main Survey images. See also Section \ref{ssec:SPCS_SNDDF} for a DM processing case study to find SNe in a DDF.

\noindent \textbf{Galaxies -- } The additional depth of a DDF may provide access to a larger collection of low-$\mu$ objects. \citedsp{Publication-142} mentions "identification of nearby isolated low-redshift dwarf galaxies via surface-brightness fluctuations" and "characterization of low-surface-brightness extended features around both nearby and distant galaxies". The DDF stacks could also be used to characterize of high-$z$ clusters, although this ability might depend on deblending extended objects. Also, the DDF observations, when combined with the WFD, allow for AGN monitoring on a variety of timescales in well-characterized galaxies \citedsp{Publication-142,Publication-143}. \textbf{Summary.} As with the SN science goals, these use standard visit images and reconfigured DM pipelines to make deep CoAdds and extract sources. In addition, it seems likely that user-generated algorithms that are optimized to detect and characterize particular types of faint extended sources will be needed, and these are beyond the scope of DM.

\noindent \textbf{Weak Lensing -- } The deeper imaging from DDFs can help with shear systematics and the effects of magnification in the analysis of WFD data (community forum, Jim Bosch). \textbf{Summary.} As with the SN and Galaxies DDF-related science goals, these use standard visit images and reconfigured DM pipelines can be used to make deep CoAdds and extract sources, as Jim notes.
%$\bullet$ \textit{Jim Bosch -- "Will need to process at least some deep drilling fields (high-latitude ones) in the same way we process a full data release production before running the full data release production, so we can use the results to build priors and/or calibrate shear estimates on the wide survey"} (\texttt{\Large{Community}} forum) \\
%$\bullet$ \textit{Jim Bosch -- "Will need to process various wide-depth subsets of some deep drilling fields (again, high-latitude ones) using the regular DRP pipeline. We'll definitely want best-seeing, worst-seeing, and probably a couple of independent typical-seeing subsets, but there may be other ways we'd want to subdivide as well."} (\texttt{\Large{Community}} forum)  \\
%$\bullet$ \textit{MLG side note -- Photo-$z$ are very important to weak lensing \citedsp{Document-10963} and so perhaps the implemented method should be chosen with weak lensing science prioritized.} \\

\end{document}
