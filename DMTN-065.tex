\documentclass[DM,lsstdoc,toc]{lsstdoc}
\usepackage{graphicx}
\usepackage{url}
\usepackage{latexsym}
% \usepackage{color}
% black, blue, brown, cyan, darkgray, gray, green, lightgray, lime, magenta, blue, orange, pink, purple, red, teal, violet, white, yellow.
\usepackage{enumitem}

\title[LSST Special Programs]{Data Management \\ and LSST Special Programs}

\author{M.~L.~Graham, M.~Juri\'{c}, K.-T.~Lim, E.~Bellm, G.~Dubois-Felsmann, L.~Guy, and the Data Management System Science Team}

\setDocRef{DMTN-065}
\date{\today}
\setDocUpstreamLocation{\url{https://github.com/lsst-dm/dmtn-065}}

\setDocAbstract{
The core LSST science goals will be met by the Wide-Fast-Deep (WFD) Main Survey, which is expected to be accomplished with 85--90\% of the observing time.
The remaining 10--15\% of the time will be spent on Special Programs: additional survey areas and/or observing strategies that are driven by specific science goals which build on, or are beyond, the core science pillars of the LSST.

\medskip
This document provides a summary of the Special Programs being considered, the potential diversity of data they might produce (e.g., shorter or longer exposure times), and the role of Rubin Observatory in reducing, processing, and serving data products from Special Programs.
The latter includes a discussion about the relevant system requirements, the needed capabilities of the LSST Science Pipelines and the Rubin Science Platform, and the cases in which Special Programs data products would be user-generated.

\medskip
The specialized, science-specific aspects of Special Programs processing and analysis that are best left to the science community are also described, and illustrated with case-study examples.

\medskip
The main target audience of this document is Rubin Observatory staff -- the construction-era Data Management (DM) team and the operations-era Data Production (RDP) team -- but members of the science community who are planning to use Special Programs to reach their science goals may also find this document useful.
}

\setDocChangeRecord{%
\addtohist{0}{2017-11-14}{Status: internal working document.}{Melissa Graham}
\addtohist{1}{2018-06-17}{Updated to finalize and issue.}{Melissa Graham}
\addtohist{2}{2021-12-01}{Updates per DM-20375.}{Melissa Graham}
}

\begin{document}

\maketitle

% CITATION EXAMPLES
% \verb|\citellp|: \citellp{LPM-17, LSE-30} \\
% \verb|\citell|: (SRD; \citell{LPM-17,LSE-29}) \\
% \verb|\citep[][]|: \citep[e.g.,][are interesting]{LPM-17,LSE-29} \\
% \verb|\cite|: \cite{LPM-17,LSE-29}

% % % % % % % % % % % % % % % % % % 
\section{Terms and Definitions} \label{sec:terms}

\subsection{Wide-Fast-Deep (WFD) Main Survey}\label{ssec:terms_wfd}
The WFD Main Survey is the core science program of the LSST, designed to achieve the science goals defined by the Science Requirements Document (SRD; \citeds{LPM-17}).
The WFD will cover at least $\sim$18000 deg$^2$ and is expected to be accomplished with 85--90\% of the observing time.
As of the Phase 1 SCOC recommendations in \citeds{PSTN-053}, the WFD Main Survey is thought of as several contiguous areas.
The largest is the low-dust-extinction survey area, which would also receive the most ($>$800) visits over the 10-year survey and be completed to uniform depth, likely with a rolling cadence.
Smaller regions cover the higher-extinction high stellar density Galactic Bulge and Magellanic Clouds, the North Ecliptic Spur (NES), and the remainder of the Galactic Plane and the South Celestial Pole.
Boundaries, re-visit cadences, and depths for all WFD areas remain to be determined \citedsp{PSTN-053}.

\subsubsection{Mini-Surveys}\label{sssec:terms_wfd_mini}
As of the Phase 1 SCOC recommendations in \citeds{PSTN-053}, the term ``mini-survey" was being used to refer to the areas of the WFD / Main Survey outside of the large, low-dust-extinction survey area.
This term used to refer to non-WFD regions (Special Programs).

\subsection{Special Programs}\label{ssec:terms_sp}
Special Programs are additional survey areas and/or observing strategies that are driven by specific science goals which either build on, or are outside of, the core science pillars of the LSST.
Special Programs will fill the remaining 10--15\% of the available survey time, and include both ``deep drilling" fields and ``micro-surveys".

\subsubsection{Deep Drilling Field (DDF)}\label{sssec:terms_sp_ddf}
A DDF is a single pointing for which many exposures are obtained during the night.
The five DDFs (below) are expected to take $\sim$5\% of the total available time.
Generally, the LSST observing strategy for a DDF is to obtain some or all of these exposures consecutively, and to maintain a high inter-night cadence over a short season (e.g., returning every two nights over four months, \citealt{2019ApJ...873..111I}) -- but the exact strategies and total time for the LSST DDFs remain to be determined \citedsp{PSTN-053}.

\begin{itemize}
\item Elias S1 (00:37:48, -44:00:00)
\item XMM-LSS (02:22:50, -04:45:00)
\item Extended Chandra Deep Field-South (03:32:30, -28:06:00)
\item COSMOS (10:00:24, +02:10:55)
\item Euclid Deep Field South  (04:04:58, -48:25:23)\footnote{\url{https://www.cosmos.esa.int/web/euclid/euclid-survey}}
\end{itemize}

\subsubsection{Micro-Surveys}\label{sssec:terms_sp_micro}
As of the Phase 1 SCOC recommendations in \citeds{PSTN-053}, the term ``micro-survey" refers to either new sky areas observed with a WFD-like survey, or sky areas within the WFD but observed with a specialized strategy.
Micro-surveys are expected to take up the remaining $\sim$10\% of the total available survey time over LSST's 10 years, and those which would use $>$0.3\% of the time and are currently being considered by the SCOC are listed below (see also \citeds{PSTN-053}).

\begin{itemize}
\item short-exposure twilight visits (for near-Sun and near-Earth objects; NEOs)
\item a static short-exposure (5 sec) map of the sky in $ugrizy$ in the first year (for calibration)
\item an extended short-exposure survey of the sky in $ugrizy$ (for transient detection and static-sky calibration)
\item target-of-opportunity (ToO) follow-up (to identify optical counterparts to gravitational wave sources)
\item coverage of the Roman microlensing bulge field (potentially as a DDF)
\item deeper $g$-band imaging of 10 local volume galaxies
\item a high-cadence survey of 2 fields in the SMC (for microlensing)
\item annual week-long surveys of the Carina nebula and surrounding star-forming regions
\item a limited-visit (i.e., shallow) ``northern stripe" survey to declination $<$+30 degrees
\item a survey of the Virgo cluster to WFD depth
\end{itemize}

Note that there remains some grey area between what is a mini-survey within the WFD, and what is a WFD-like micro-survey.
For example, the last two on the list above (the northern stripe and the Virgo cluster) could also be (or might be in the future) called WFD mini-surveys, and not Special Programs micro-surveys.

\subsection{Visit Types}\label{ssec:terms_visits}
A visit is an observation of a single pointing at a given time, of which there are three types as listed below.
As of the Phase 1 SCOC recommendations in \citeds{PSTN-053}, most WFD and DDF visits would be standard or alternative standard visits.
However, non-standard visits with longer exposure times are being considered, especially for $u$-bands, and several of the micro-surveys are considering shorter exposure times.
The full potential for diversity in Special Programs data is reviewed in Section~\ref{ssec:proc_datadiv}.

\begin{itemize}
\item Standard Visit -- Composed of $2\times15$ second exposures (commonly referred to as ``snaps").
\item Alternative Standard Visit -- Composed of a single $30$ second exposure.
\item Non-Standard Visit -- Any other exposure time(s) or number of snaps.
\end{itemize}



\clearpage
% % % % % % % % % % % % % % % % % % 
\section{Reduction and Processing for Special Programs Data Products}\label{sec:proc}


\subsection{Rubin Requirements Related to Special Programs}
% \lsrreq, \ossreq, \dmreq
% \reqparam

A detailed list of the requirements related to Special Programs is provided in Appendix~\ref{sec:docrev}.
The most important requirements are summarized below.

\subsubsection{Data Products}

Rubin Observatory and the LSST system (the observatory and the data management systems) are required to process Special Programs data to produce unique and separate data products ``whenever possible" (LSR-REQ-0121).

The term ``whenever possible" includes cases where the original or reconfigured versions of the LSST Science Pipelines can be run, and excludes cases where the development of new algorithms or the allocation of significant additional computational resources are required (LSR-REQ-0121).

The statement ``to produce unique and separate data products" typically refers to producing the same kinds of data products as will be generated by the Prompt and Data Release pipelines (processed visit images, coadded images, difference images, and catalogs of sources and objects for those images).

It is a requirement that the cumulative size of the Special Programs data products generated by Rubin Observatory be no more than $\sim$10\% the size of the Data Release data products (i.e., proportional to the fraction of survey time spent; LSR-REQ-0121).

It is a requirement that these Special Programs data products be distinct, and joinable with (in other words, they can be federated or cross-matched with) the Prompt and Data Release data products (DMS-REQ-0322).

The derivation of value-added data products, such as HiPS or MOC maps, for Special Programs remains an open question (DMS-REQ-0379, 0383).

\subsubsection{Metadata}

In order to support Special Programs processing, the LSST system is required to store metadata that includes program information for every raw image, such as identifiers for images obtained as part of the Main Survey or a Special Program (DMS-REQ-0068).

It is required that this metadata be sufficient for Special Programs to trigger their own real-time data processing recipes ``whenever possible" (DMS-REQ-0320), and be included in alert packets (DMS-REQ-0274).

\subsubsection{Processing}

It is a requirement that Special Programs processing with the Prompt pipeline (or a reconfigured version of it) is subject to the same timescales and latency constraints of 24 hours for the release of Prompt data products and 1 minute for the transmission of Alert packets (DMS-REQ-0344).

It is also a requirement that Special Programs processing be done on timescales intermediate to the Prompt and Data Release processing, ``whenever possible" and whenever necessary to enable the intended science goals of the Special Program (LSR-REQ-0032).

It is a requirement that the LSST system be able to process non-standard visits with short exposure times as low as 1 second, with a discussion note that such short exposures might have degraded image quality (LSR-REQ-0111).

Processing for Special Programs by Rubin Observatory is expected to use no more than $\sim$10\% of computational and storage capacity of the Rubin data processing cluster (i.e., proportional to the fraction of survey time spent and the size of the Rubin-processed Special Programs data products; Section 6 of the DPDD).

\subsubsection{User Processing}

In cases where the science goals of a Special Program require that new algorithms or software be developed, User-Generated pipelines and data products will be needed.

The 10\% of the total data processing capacity that Rubin Observatory is required to reserve for all User-Generated processing includes that applied by users to Special Programs data -- in other words, there is no additional capacity {\it for users} that will be reserved only for Special Programs data (LSR-REQ-0041).





\subsection{The Potential Diversity of Special Programs Data} \label{ssec:proc_datadiv}

As discussed in Section~\ref{sec:terms}, most of the Special Programs that are currently under consideration will use standard or alternative standard visits.
However, some are likely to require non-standard visits with shorter exposures (5 sec).
Furthermore, some are likely to acquire images with a significantly brighter sky background (e.g., twilight images) than most nighttime survey images.

The cadence and patterns of Special Programs might also differ from the WFD, such as long series of exposures obtained of the same field (e.g., DDFs), or a strategy optimized to find very fast-moving objects (e.g., NEOs).

It does not appear that any of the currently-proposed Special Programs are likely to violate boundaries imposed by the Rubin Observatory hardware, but there remain a few open questions about boundaries on data processing imposed by the LSST Science Pipelines.

\subsubsection{Hardware Boundaries}

Appendix~\ref{sec:hardbounds} lists all of the hardware boundaries that might constrain the potential diversity of Special Programs data.

The minimum exposure time is 1 second (stretch goal: 0.1 seconds), and there is currently a hardware boundary that limits the readout rate to 1 every 15 seconds.
This would affect the image acquisition rate and increase the overheads of the proposed short-exposure micro-surveys.

Hardware imposes no other boundaries on how data can be obtained, but Special Programs that request a high number of filter changes and/or long slews could be inefficient due to large overheads.

\subsubsection{Processing Boundaries}

Appendix~\ref{sec:procbounds} describes the boundaries on what types of visits can be processed and calibrated by the LSST Science Pipelines, which are designed to process standard (or alternative standard) visits.

Very short ($<$2 sec) exposures could be difficult to reduce due to an incompletely-formed PSF, and very short or very long ($>$150 sec) exposures could be difficult to calibrate due to having too few (or too few unsaturated) stars.

It is currently unclear whether images with very bright sky backgrounds (twilight images) can be processed with the LSST Science Pipelines, or whether user generated pipelines will be needed.

The full reduction and calibration of images obtained with non-sidereal tracking, in which the stars are streaked, is currently beyond the scope of the LSST Science Pipelines, and will require a user generated pipeline.


\subsection{Anticipated Rubin-Processed Special Programs Data Products}

{\bf The LSST Project's Role in Processing Special Programs Data -- } The formal requirements regarding LSST's role in processing Special Programs data are in \citeds{LSE-61}, and the following statements have been derived from those requirements. The LSST Project will not take formal responsibility for specialized data reduction algorithms needed to process data, including images taken in non-standard modes. The term "specialized algorithms" refers to software that is not already within scope of the LSST Data Management (DM) science pipelines, and may include, for example: difference imaging for short exposures in which the PSF is not well-formed, shift-and-stack for faint moving objects, or any software with computational needs that significantly surpass the processing budget per image (compared to the processing of a WFD image). The Project will incorporate Special Programs data into the Prompt and/or Data Release processing pipelines and data products of the WFD Main Survey, such as {\tt Alerts}, {\tt CoAdds}, or {\tt Source} and {\tt Object} catalogs (with appropriate flags; LSST data products are described in \citeds{LSE-163}), whenever this (1) can be accomplished with existing software, and (2) is scientifically beneficial to that data product. The Project will also reconfigure its pipelines to generate separate imaging and catalog data products for Special Programs, whenever this can be accomplished with existing software. Finally, the Project will enable user-generated processing via the Science Platform (\citeds{LSE-319}), which will provide software tools and computational resources for (re)processing LSST data.

Whether or not Special Programs images are incorporated into the WFD main survey's data products, it is anticipated that most of the Special Program's science goals will require (or benefit from) separate data products (i.e., CoAdds and/or catalogs). For this reason, LSST intends to reconfigure the DM's pipelines in order to generate unique and separate -- but joinable -- imaging and catalog products for Special Programs data, whenever possible. In this context, ``possible" means that no new algorithms need to be written and that an intensive amount of additional computational resources is not required for the processing.

In a ``possible" scenario, DM would assemble a pipeline from existing DM codes in order to process data associated with a given Special Program and build image and catalog products that meet the science needs of that particular program. For example, for a DDF SN survey (see Section \ref{ssec:SPCS_SNDDF}), existing DM codes would be used to: (1) make a deep template image from a certain time window, (2) process standard single visit images, (3) create a nightly CoAdd, (4) run difference imaging analysis, (5) run source detection on the difference images, and (6) create \texttt{DIASource} and \texttt{DIAObject} catalog equivalents (this example is also given in Section 6 of the \DPDD, \citedsp{LSE-163}). This type of reconfiguration would also be possible to create as a user-generated pipeline (Section \ref{ssec:dmplans_user}), but having these products provided by the Project ensures a consistent and verified level of quality. %, as well as access for all users to the processed Special Programs data products, which would only increase the scientific value of Special Programs data. --> MLG: data rights doc will probably state that the products for SP data that is only processed by user-generated pipelines will have to be served to all users

The above statements of intent are derived from the Data Management Subsystem Requirements document, \citeds{LSE-61}, which contains several requirements related to the processing of data from Special Programs (DMS-REQ-0069, 0320, 0321, 0322, and 0344). 
Ensuring that the work-hours needed to reconfigure and test the pipelines, to run them and to verify and validate the data products for public release (which may potentially be needed on intermediate timescales that do not coincide with the Prompt/Yearly timescales, e.g., monthly stacks of deep drilling fields), is the role of Operations.

DMS-REQ-0320 states that "it shall be possible for special programs to trigger their own data processing recipes".
A header keyword identifying an image as related to a Special Program will be included (e.g., the survey name), and that will be sufficient to send it to a dedicated processing pipeline. 


\subsection{Anticipated User-Generated Special Programs Data Products}\label{ssec:dmplans_user}

In cases where the science goals of a Special Program require specialized algorithms and cannot be achieved by reconfiguring DM's software, then user-generated pipelines will be needed.
Towards this end, LSST DM is making all of its software open-source, and preparing the Science Platform (\citeds{LSE-319}), through which users can access the tools and computational resources to assemble data processing pipelines to achieve their science goals (whether related to Special Programs data or not).
During Operations, there will be a method for the system to allocation processing resources in the case of over-subscription.

If the user-generated processing pipeline for Special Programs data requires requires significantly more computational resources than have been allocated -- where that allocation has been sized approximately, based on image processing for WFD main survey data (i.e., difference imaging, source detection, and/or stacking) -- then external computational resources may be necessary.
To support such external processing DM intends to make the data and its code base accessible to and exportable by users in the science community.
No user-generated pipeline may contribute Alerts to the Alert Stream. %, although a separate stream should be possible if the packet and transport formats are adopted (see also Section \ref{ssec:dmplans_user}).

It is furthermore expected that, over time, some user-designed pipelines might become ``adopted", installed and operated (and change controlled) by the LSST Operations team.
For both adopted and user-run code, whether they are for Special Programs or WFD survey data, the LSST DM team will encourage and facilitate data product databases that are built with the same schema as -- and can easily be joined with -- the tables of the Prompt and DRP data products.
An alternative option to ``adopted" code is ``adopted" data products: situations in which user-generated code is run externally, a data catalog is returned to LSST to be ingested, verified, and made public.



\subsection{Including Special Programs Data in the WFD Main Survey's Data Products}\label{ssec:dmplans_WFD}

The Project may incorporate Special Programs data into the WFD main survey's pipelines and data products whenever this is (1) possible and (2) scientifically beneficial. This will likely be at the discretion of the data quality assessment team during Operations. In the following three sections we project when and how Special Programs data might be incorporated into the pipelines and data products of the Prompt pipeline and the Alert stream (Section \ref{ssec:dmplans_prompt}) and the annual Data Release pipeline (Section \ref{ssec:dmplans_drp}).

\subsubsection{The Prompt Pipeline and Alert Generation}\label{ssec:dmplans_prompt}

It would be beneficial to transient science to include as many LSST images into the Difference Imaging Analysis (DIA) pipeline and the Alert Stream as possible. Only images that can be processed with the Prompt DIA pipeline with the $60$-second timeline can contribute to the Alert Stream. As discussed in Section \ref{ssec:proc_datadiv}, this might prohibit exposures shorter than $<15$ seconds and/or visit cadences shorter than $1$ per $\sim30$ seconds. This might also prohibit the inclusion of very crowded fields that require more computational resources. A field must have an LSST template image to be processed by the DIA pipeline, which would prohibit immediate Alerts from new survey fields. It should be possible to load and use an alternative template image (than what would be used for that field if and when it is covered by the WFD main survey) in the Prompt pipeline for fields from a Special Program. 

There may be a couple of issues encountered with Alerts from many consecutive visits of Deep Drilling fields. One is that, since the Alert contains the full record of all associated {\tt DIASources} from the past 12 months (\citeds{LSE-163}), for a Deep Drilling Field with significantly more visits over the year, the size of the Alert might become prohibitively large (TBD). Another is that the self-consistency of the \texttt{DIAObjects} catalog may suffer during consecutive visits of a single field. For example, the processing for image $2$ of a sequence would begin when the processing for image $1$ is only halfway complete. Any new {\tt DIASource} in image $1$ that cannot be associated (by coordinate) with an existing {\tt DIAObject} becomes a new {\tt DIAObject}. When this source is again detected in image $2$, another new {\tt DIAObject} would be created if the catalog has not yet been updated.

% \textbf{Templates:} The template images that are used in the Level 1 difference imaging pipeline will be built from the Level 2 DRP, and so the first factor affecting a Special Programs image's suitability for Level 1 is to be in a region of sky with an existing template. So long as there is a template, when the exposure time is equivalent to a WFD visit image, $\sim 30$ seconds, treating the image as Level 1 is going to be fine. When processing Special Programs data with the Level 1 pipeline, certain science cases might call for the capability to load and use a certain template that is different from the WFD template (i.e., built over a different timescale). K.-T. confirms that there is not enough memory allotted to store more than one template over the whole sky, but for sub-regions, storing and using an alternative template should be possible. This is not an issue limited to Special Programs, since during commissioning it is conceivable that multiple template versions. K.-T. also confirms that the capability for the processing pipeline to choose a given template based on the programID in the raw image metadata will exist.

% \textbf{Processing backup:} Are there options for short-term increases in parallel processing power at NCSA? (Such options might be needed anyway to process crowded fields with $>$10k \texttt{DIASources}.) The bandwidth needed to load templates at a faster rate is also a concern, since templates have twice as many pixels, if the data acquistion rate is surveying at twice the WFD main survey rate, that's a $4\times$ additional bandwidth load. However, at NCSA there should be ways to elastically change the amount of processing power available. This is also an issue for crowded fields (below).

% Special Programs are more likely to include crowded fields than the WFD main survey area. Due to the increased number of sources, the number of \texttt{DIASources} -- and therefore the number of Alerts -- increases as well. In turn, this increases the processing time and in some cases, may exceed the 60 second limit for Alert Production. A policy is needed on whether Alerts from crowded fields should be allowed a delay, or allowed to be incomplete. K.-T. reports that the control system can easily kill processes that are running over time and move forward with existing outputs -- but that perhaps it will be just as easy to let it keep running and elastically grab additional NCSA resources as needed. This appears possible because the batch system is larger than the Level 1 allocation, although we might not know that this is possible until Level 1 integration happens at NCSA (see Document LDM-230, the operations concepts).

{\bf Solar System Processing -- } Since Solar System Processing takes \texttt{DIASources} as input, any Special Programs images that can be run through the Alert Pipeline can also be incorporated into Solar System Processing.
As discussed under "Solar System Objects (SSO)" in Appendix \ref{sec:prevpropsp}, most of the Special Programs data associated with SS science will obtain standard visit images anyway.
%There was some concern that a large number of small-separation sources might overwhelm the processing system (i.e., from a deep drilling field with many exposures in a sequence), but upon further consideration this worry was rejected.

\subsubsection{The Data Release Pipeline (DRP)}\label{ssec:dmplans_drp}

This document is not the place for a full consideration of whether or not it would be ``scientifically beneficial" to include any Special Programs data in the DRP data products -- namely, the deep image CoAdds and their corresponding {\tt Source} and {\tt Object} catalogs (\citeds{LSE-163}) -- and we leave that decision for the data quality assessment team in LSST Operations. One example might be when Special Programs data brings additional area up to the same level of depth and cadence as the rest of the WFD main survey. Another may be if including some or all of the shallower Galactic Plane coverage suppresses edge effects or low-order modes in the all-sky photometric solutions.


\clearpage
% % % % % % % % % % % % % % % % % % 
\section{Enabling the Discovery and Analysis of Special Programs Data Products}\label{sec:analysis}

\subsection{Anticipated Rubin Science Platform Capabilities for Special Programs }

\begin{itemize}
\item discoverability of SP data when browsing an all-sky map
\item to query data by tag of WFD, WFD mini-survey, or Special Program (e.g., DDF field, micro-survey identifier)
\end{itemize}





\clearpage
% % % % % % % % % % % % % % % % % %
\section{Special Programs Processing Examples}\label{sec:SPCS}

For further insight to the DM-related needs of potential Special Programs, we can write out all of the data acquisition and processing steps, in order, that some of the proposed Special Programs might use.
Note that we are not including any analysis in these descriptions, only processing and products. These are not necessarily complete and may even be incorrect in some places, as we are not experts in the science needs of these potential Special Programs; they could use some more thought and input.

Basic steps that we use to describe a processing case study: \\
Step 1. Data Acquisition. \\
Step 2. Inclusion in the Prompt Pipeline and Alert Generation. \\
Step 3. Delivery of LSST Processed Images. \\
Step 4. Reconfigured Processing Pipelines and Separate Data Products. \\
Step 5. Inclusion in the DRP Data Products for the WFD Main Survey. \\
Step 6. User-Generated Pipelines and Products. \\

\subsection{Searching for TNOs with Shift-and-Stack}\label{ssec:SPCS_TNO}

This Special Programs processing summary is based on Becker et al. (2011) white paper to find TNOs with shift-and stack (SAS) \citedsp{Document-11013}.

Step 1. Data Acquisition. \\
The observational sequence is triggered. In a single night, the 9 adjacent fields in a 3x3 grid are observed with $336$ $\times$ $15$ second $r$-band exposures. This sequence is always repeated 2-3 nights later. This re-visit sequence is repeated 3 more times: 1.5 months, 3 months, and 13.5 months later. Data obtained in the $g$-band filter is also acceptable. \citedsp{Document-11013}

Step 2. Inclusion in the Prompt Pipeline and Alert Generation. \\
Each $2\times15$ second visit is processed in the Prompt pipeline and Alerts are released within 60 seconds.

Step 3. Delivery of LSST Processed Images. \\
The raw, reduced, and calibrated exposures and difference images from the Prompt pipeline are made available within \texttt{L1PublicT} (currently 24 hours; LSR-REQ-0104), but this is not very relevant for this program, which requires a year of dispersed observations before the processing pipelines for SAS can be run.

Step 4. Reconfigured Processing Pipelines and Separate Data Products. \\
Shift-and-stack processing is beyond the scope of DM's algorithms.

Step 5. Inclusion in the DRP Data Products for the WFD Main Survey. \\
As with all Special Programs data, they might be included in the products of the WFD main survey if DM decides it is beneficial. However, since these images are much deeper than stacks made from the WFD survey, and the strict timing of the observations might lead to their acquisition in sub-optimal conditions, it is unlikely that they would \textit{all} be incorporated.

Step 6. User-Generated Pipelines and Products. \\
The user-generated pipeline running the shift-and-stack processing will be set up and submitted for batch processing by the user through the Science Platform or on an external processor. Pipeline inputs will be the 336 processed exposures per field per re-visit sequence. The DRP difference imaging routine will be used with the same template tract/patch for all. Custom, user-generated algorithms will shift the exposures and create difference images, then DRP routines can stack and do source detection and characterization and generate an object database. Custom code will derive orbital parameters for the detections and add them to a {\tt SSObjects}-like database.


%% % % % % % %
\subsection{A More General Level 3 Shift-and-Stack Case Study, by Mario}\label{ssec:SPCS_SAS}

A Level 3 processing case study for shift-and-stack on a large number images. By Mario.

\#1. The scheduler is configured to repeatedly (e.g., 10 times) observe a field during the same night with longer exposure than usual (e.g., 120 sec). [ and we should take the actual numbers from the TNO-DDF whitepaper; don't have the internet right now or I would].

\#2. The images are processed as regular "Level 1" products within 60 seconds, and transmitted as alerts, with results stored into the regular L1 database. This will happen automatically for all images (perhaps within some range of exposure times?).

\#3. The raw images (and all necessary calibrations), calexps, and standard L1 diffims are made available within 10 minutes to the batch system for processing with special programs-specific codes. This is the same batch system we make available to the users for running Level 3 codes [Q for us: is it? or is is the same one that's used to process calibrations? have these systems been sized?].

\#3 a). The code running the shift-and-stack processing will be externally developed and delivered, but will be installed and operated (and change controlled!) by the LSST Operations team. That is, we don't expect someone external to the ops team to babysit the code on a nightly basis. In fact, it's the opposite: once the codes are delivered, any changes will go through LSST's software change control process.

\#4. There will be a facility to trigger program-specific processing on the batch system upon the arrival of a new image (above); this processing will then be queued up for execution. We assume that the policy for processing of special programs data may give it preferential treatment relative to general-purpose L3.

\#5. Once the processing finishes, the results of will be stored to a program-specific database. No alerts (in VOEvent sense) will be issued. We will provide a generic notification facility (perhaps something as simple as an RSS feed) that new data has been made available in a certain database/data store. [This is an example where I'd want to make sure somebody within DM is planning to provide such a facility.].

\#6. The outputs stored can be special-program specific (i.e., tables with nearly arbitrary schemas -- some columns -- like ra/dec for spatial joins -- should be present in main tables). The outputs can also contain images (stored in also special-program specific repository), or custom products (treated like opaque files). The visualizations available for these (catalogs, images, arbitrary files) through the Portal will be limited (e.g., generic table visualizations or x-y plots).

\#7. When the images are made available to the batch system (step \#3), they also become available to *everyone*. I.e., someone else could also run a custom L3 pipeline on these data, feeding their custom L3 database. (This isn't in the requirements right now -- we say that images will become available in 24hrs -- and is addressed in Section \ref{ssec:dmplans_user}).

\subsection{Searching for Supernovae in Deep Drilling Fields}\label{ssec:SPCS_SNDDF}

Step 1. Data Acquisition. \\
On a single deep drilling field, the scheduler obtains e.g., 5, 10, 10, 9, and 10 visits with $2\times15$ second exposures in $grizy$ (or similar for the night's filter set) and a small dither pattern between visits.

Step 2. Inclusion in the Prompt Pipeline and Alert Generation. \\
Each $2\times15$ second visit is processed by the Prompt pipeline's DIA, and Alerts are released within 60 seconds. They are flagged to denote the image source is a DDF and that source association might be compromised.

Step 3. Delivery of LSST Processed Images. \\
The raw, reduced, and calibrated exposures and difference images from the Prompt pipeline are made available within \texttt{L1PublicT} (currently 24 hours; LSR-REQ-0104).

Step 4. Reconfigured Processing Pipelines and Separate Data Products. \\
The required data products for this science goal can be met by reconfiguring the DM pipelines. First, a template image for the field will be made using DM stacking algorithms. On nights when this DDF is observed, at the end of the sequence of observations, DM algorithms are used to create a nightly deep stack, PSF-match it with the template, create a deep difference image, run source detection on the differences, and create separate databases of \texttt{DIAObject}, \texttt{DIASource}, and \texttt{Object} that are unique to this DDF. The LSST codes for alert packet and transport could be used to distribute the detected objects e.g., to the same brokers that receive the Alert Stream, or alternative destinations. However, these packets would not be distributed via the LSST {\tt Alert Stream}, and would need to be identified as, e.g., DDF Alerts.

Step 5. Inclusion in the DRP Data Products for the WFD Main Survey. \\
As with all Special Programs data, they might be included in the products of the WFD main survey if DM decides it is beneficial.

Step 6. User-Generated Pipelines and Products. \\
For the science goal of searching for supernovae in nightly stacked DDF images, no separate user-generated software appears necessary.


\subsection{A Twilight Survey with Short Exposures}\label{ssec:SPCS_Twilight}

Several kinds of twilight surveys with short exposures have been or might be proposed: to put brighter stars (or transients such as supernovae) that saturate in a $15$ second image onto the LSST photometric system and/or to observe the Sweetspot, 60 degrees from the sun, for near-Earth objects. The processing case study for these is currently limited by unknowns about the first step: the reduction of processed single visit images.

Step 1. Data Acquisition. \\
At a specified time (or e.g., 6 degree twilight), the scheduler begins dither pattern of short exposures. Location and exposure times are set by the sky brightness and desired saturation limits.

Step 2. Inclusion in the Prompt Pipeline and Alert Generation. \\
Pending studies of short-exposure suitability for DIA (see Section \ref{sec:procbounds}) and scalable processing capabilities to incorporate a faster image-input rate than $1$ every $30$ seconds, these data could {\it potentially} be incorporated and spawn Alerts.

Step 3. Delivery of LSST Processed Images. \\
Pending the issues mentioned above, the raw, reduced, and calibrated exposures and difference images from the Prompt pipeline are made available within  \texttt{L1PublicT} (currently 24 hours; LSR-REQ-0104).

Step 4. Reconfigured Processing Pipelines and Separate Data Products. \\
This is officially not determined, but so long as the short-exposure images can be processed and have enough stars for photometric and astrometric calibration, reconfigured DM pipelines will probably be sufficient for creating image and catalog products from this kind of data.

Step 5. Inclusion in the DRP Data Products for the WFD Main Survey. \\
These short-exposure, high sky background images would not contribute to the DRP data products created for the WFD survey.

Step 6. User-Generated Pipelines and Products. \\
If short-exposure images cannot be processed with the existing DM algorithms, a user-generated processing pipeline might be needed to reduce the raw data. 

Side note: A short-exposure survey of the bright stars of M67, described in Chapter 10.4 of the Observing Strategy White Paper \citep{2017arXiv170804058L}, suggests using the stretch goal of 0.1 second exposures or, if that is not possible, \textit{"custom pixel masks to accurately perform photometry on stars as much as 6 magnitudes brighter than the saturation level"}. This would be considered a user-generated algorithm.

\subsection{The Galactic Plane Survey for Variable Stars and/or Exoplanets}\label{ssec:SPCS_GPVSEx}

Step 1. Data Acquisition. \\
The schedule incorporates fields in the Galactic Plane, and executes $2\times15$ second visits in these fields (or shorter, for a shallower depth than the WFD main survey).

Step 2. Inclusion in the Prompt Pipeline and Alert Generation. \\
Each $2\times15$ second visit is processed in the Prompt pipeline and Alerts are released within 60 seconds. Extremely crowded fields might have to be skipped if they take longer to process and violate the $60$ second latency for Alerts. 

Step 3. Delivery of LSST Processed Images. \\
The raw, reduced, and calibrated exposures and difference images from the Prompt pipeline are made available within  \texttt{L1PublicT} (currently 24 hours; LSR-REQ-0104).

Step 4. Reconfigured Processing Pipelines and Separate Data Products. \\
The image and catalog products needed for science with the Galactic Plane are very similar to the products of the Prompt and DRP pipelines, so it seems that not much reconfiguration would be needed. The biggest difference might be the incorporation of a user-supplied deblender algorithm optimized for very crowded fields.

Step 5. Inclusion in the DRP Data Products for the WFD Main Survey. \\
It is quite likely that images from the Galactic Plane will be included into the products of the WFD main survey, as they could e.g., reduce edge effects and help with global photometric classification, but this will depend on deblender performance, and left to the discretion of DM. 

Step 6. User-Generated Pipelines and Products. \\
It seems likely that science users will want to deploy their alternative deblending algorithms on this data set and create their own catalogs.

\subsection{Gravitational Wave Event Follow-Up}\label{ssec:SPCS_GW}

For a description of how target of opportunity data to search for the optical counterparts of gravitational wave events would be processed, see \citeds{rtn-008}.


% % % % % % % % % % % % % % % % % %
\clearpage
\bibliography{local,lsst,refs,books,refs_ads}


% % % % % % % % % % % % % % % % % %
\clearpage
\appendix

\section{Documentation Review for Requirements Related to Special Programs}\label{sec:docrev}

This review is limited to requirements related to Special Programs processing or data products,
or requirements that constrain the diversity of images (e.g., exposure time limits).

Updates to Rubin documents related to Special Programs that were motivated by past versions of this DMTN
were made via LCR-1309 and LCR-2265. 

% LCR 1309: https://project.lsst.org/groups/ccb/node/2383
% LCR 2265: https://project.lsst.org/groups/ccb/node/4036


\subsection{Science Requirements Document (SRD)}

Version 5.2.4 (revision 2018-01-30), \citedsp{LPM-17}.

Section 3.4 ``The Full Survey Specifications" states the SRD's assumption that 90\% of the total 
available survey time would be spent on the main survey, and that the remaining 10\% would be spent 
{\it ``to obtain improved coverage of parameter space ... [or to] observe special regions"}.



\subsection{LSST System Requirements (LSR)}

Version 7.1 (revision 2020-03-05), \citeds{LSE-29}.

Note that Version 5 (2018-06-26) was an update for LCR-1309, which added requirements, specifications, 
and discussions regarding the processing of Special Programs data based on earlier versions of this DMTN.

In Section 1.5.1.3, ``Processing Data from Special Programs", LSR-REQ-0122\lsrreq{0122} is a requirement 
that the LSST system {\it ``shall deliver unique and separate data products for visits from Special Programs"} 
whenever possible, and that they {\it ``shall be delivered on timescales intermediate"} to the Prompt and 
Data Release timescales {\it ``when this enables the intended science of the Special Program"}.
The discussion clarifies that {\it ``the term 'whenever possible' includes cases where the Data Management System 
can run original or reconfigured versions of existing pipelines, and excludes cases where the development of new 
algorithms, or the allocation of significant additional computational resources, are required"}.

In Section 2.4.1.1.2, ``Non-Standard Visit", LSR-REQ-0111\lsrreq{0111} requires that the LSST system 
{\it ``be capable of obtaining and processing exposures not taken in a standard visit mode including those with a 
minimum exposure time of"} 1 second ({\tt minExpTime}\reqparam{minExpTime}).
The discussion notes that {\it ``non-standard visit exposures may possibly be degraded in some aspects of performance 
(e.g. cosmic ray rejection on visits consisting of a single exposure), and might be incompatible with difference 
imaging and alert production (e.g., short exposures in which the PSF is not fully formed)"}.

The requirement in Section 1.5.1.3 is echoed in Section 2.6.1.1, ``Organization of Data Products", in which 
LSR-REQ-0032\lsrreq{0032} is a requirement that the data processing system provide the means for three 'classes' of 
data products on different timescales (Prompt, Data Release, and User-Generated), and also to provide a means for 
processing Special Programs data because the {\it ``science goals of Special Programs may require that their processed 
data products be made available in an additional fourth class, and possibly with intermediate timescales"}.

In Section 2.6.1.1.3, ``Level 3 Data Products", LSR-REQ-0041\lsrreq{0041} specifies that the LSST system 
{\it ``shall support"} User-Generated data products.
The discussion clarifies that ``{\it there will be technical limits on DM's ability to meet this requirement, such as 
cases where an intensive amount of additional computational resources is required, because only ~10\% of the total 
computational system is allocated for user processing"}.
This level of support applies also to user processing of Special Programs data.
See also the reference to LSR-REQ-0055 below.

Section 2.6.1.1.4, ``Data Products for Special Programs", LSR-REQ-0121\lsrreq{0121} specifies that the LSST system 
{\it ``shall produce unique and separate Data Products as the result of processing data from Special Programs whenever 
possible, on a timescale that enables the intended science goals of the Special Program.
The cumulative size of the online Special Programs data products shall be no more than ~10\% of the size of the DRP 
data products from the most recent data release"}.
The discussion clarifies that {\it ``the term 'whenever possible' includes cases where the Data Management System can 
run original or reconfigured versions of existing pipelines, and excludes cases where the development of new 
algorithms, or the allocation of significant additional computational resources, are required.
The cumulative size of the Special Programs data products is capped at ~10\% of the most recent DR because this 
matches the expected fractional survey area of Special Programs compared to the main survey"}.

In Section 2.7.1.6, ``Community Computing Services", LSR-REQ-0055\lsrreq{0055} requires that the LSST system 
{\it ``shall provide and maintain an amount of computing capacity equivalent to at least"} 10\% 
({\tt userComputingFraction}\reqparam{userComputingFraction}) {\it ``of the total LSST data processing capacity 
(computing and storage) for the purpose of scientific analysis of LSST data and the production of"} User-Generated 
data products. 
The discussion clarifies that the scope of this service remains to be determined.
This level of computational resources includes user processing of Special Programs data.

In Section 3.1.3.1, ``Survey Time Allocation", LSR-REQ-0075\lsrreq{0075} requires that the {\it ``survey performance 
requirements shall be met utilizing approximately 90\% of the historically available observing time, leaving the 
remaining time available for yet to be defined special programs"}.


\subsection{Observatory System Specifications (OSS)}

Version 19.3 (revision 2022-12-12), \citeds{LSE-30}.

Note that Version 13 (2018-06-26) was an update for LCR-1309, which added requirements, specifications, and discussions regarding the processing of Special Programs data based on earlier versions of this DMTN.

In Section 2.2.3.1, ``Standard Operating States", OSS-REQ-0044\ossreq{0044} specifies that {\it ``the LSST 
observatory system shall be designed and constructed to support ... manual observing - used for specific 
non-scheduler driven observing to support system verification and testing or specialized science programs"}. 
Although most Special Programs will be executed via the survey scheduler as part of {\it ``fully automated 
observing"}, manual observing might be necessary for, e.g., target-of-opportunity Special Programs.

Section 3.1.5.1.2, ``Data Products Handling for Special Programs", OSS-REQ-0392\ossreq{0392} is a flow-down of 
requirements from the LSR (0122, 0075, and 0121; LSE-29), and specifies that {\it ``the handling of data products 
from Special Programs shall be compliant with the approach defined in LSE-163"}.

In Section 3.6.1.3, ``Continuous Exposures", OSS-REQ-0319\ossreq{0319} requires that {\it ``The Observatory shall be 
capable of continuous operation throughout a night with the interval between successive visits equal to the FPA 
readout time"}.
The discussion clarifies that {\it ``this mode of observing is needed to support observations when the telescope is 
not being re-pointed. For example observing ``deep drilling" fields..."}.

In Section 3.6.1.4, ``Minimum Exposure Time", OSS-REQ-0291\ossreq{0291} specifies that {\it ``the camera shall be 
able to obtain a single exposure with an effective minimum exposure time of no more than"} 1 second 
({\tt minExpTime}\reqparam{minExpTime}) {\it ``with a goal of an effective minimum exposure time of"} 0.1 seconds 
({\tt minExpTimeGoal}\reqparam{minExpTimeGoal}). 
The discussion clarifies that {\it ``if the exposure is shortened from the 15 second nominal, the spacing between 
successive exposures should be extended to maintain the average readout rate consistent with a 15 second exposure"}, 
which may increase the overheads of Special Programs using short exposure times.
The discussion also clarifies that {\it ``if the exposure is lengthened from the 15 second nominal, the thermal 
stability may be affected, which may affect photometric accuracy"}.

In Section 3.6.1.5, ``Publish Visit Type", OSS-REQ-0384\ossreq{0384} specifies that {\it ``the OCS [Observatory 
Control System] shall configure the [Data Management System] DMS (in particular Prompt Processing) with the type 
of visits to be processed: Standard, Alternate, or a specific type of Non-Standard"}.
The discussion clarifies that this allows the Prompt processing pipeline to be reconfigured on-the-fly in order to 
incorporate non-standard visits from, e.g., Special Programs.
The time required for reconfiguration might introduce some latency or cause some images to not be processed by the 
Prompt pipeline.

In Section 3.6.2.1.2, ``Maximum time for operational filter change", OSS-REQ-0293\ossreq{0293} specifies that 
{\it ``the camera system shall provide the capability of changing the operational filter with any other internal 
filter in a time less than"} 120 seconds ({\tt tFilterChange}\reqparam{tFilterChange}).
This would impose a large overhead on, e.g., a Special Program that changes filters often without slewing.
See also OSS-REQ-0295\ossreq{0295}, Appendix~\ref{sec:hardbounds} of this document, and/or the filter change 
memorandum (\url{ls.st/spt-494}), for more information about the total lifetime number of filter changes.

In Section 3.6.3.1, ``Absolute Pointing", OSS-REQ-0298\ossreq{0298} specifies that {\it ``the LSST shall point to 
a defined set of sky coordinates with an RMS accuracy of"} 2 arcseconds ({\tt absPointErr}\reqparam{absPointErr}).

In Section 3.6.3.3, ``Rotator tracking Time, OSS-REQ-0301\ossreq{0301} specifies that {\it ``the LSST shall be able 
to maintain field rotation tracking over a period of at least"} 1 hour ({\tt rotTrackTime}\reqparam{rotTrackTime}).
The discussion clarifies that this {\it ``is driven by the need to conduct extended 'deep drilling' observations on 
a single field"}.

In Section 3.6.3.5, ``Offset Pointing", OSS-REQ-0302\ossreq{0302} specifies that {\it ``the LSST shall be capable of 
offset pointing within a single field-of-view with a precision of no more than"} 0.2 arcseconds 
({\tt offsetPointingErr}\reqparam{offsetPointingErr}).

In Section 3.6.3.6, ``Open Loop Tracking", OSS-REQ-0303\ossreq{0303} specifies that {\it ``The LSST shall be 
capable of open loop tracking without the assistance of real time optical feedback to an accuracy of"} 1.0 arcseconds 
({\tt openTrackErr}\reqparam{openTrackErr}) {\it ``over any 10 minute duration during normal night time operations"}.
Note that the open loop tracking requirement is \emph{without guiding}.

In Section 3.6.3.10, ``Non-Sidereal Tracking", OSS-REQ-0380\ossreq{0380} specifies that {\it ``the LSST system shall 
be capable of tracking in an arbitrary direction on the sky along a parametric RA(t) and DEC(t) trajectory, at 
angular rates of up to"} 220 arcseconds per second ({\tt nonsiderealAngularRateEl}\reqparam{nonsiderealAngularRateEl} 
and {\tt nonsiderealAngularRateAZ}\reqparam{nonsiderealAngularRateAZ}) {\it ``with a tracking error not to exceed"} 
0.5 arcseconds per minute ({\tt nonsiderealTrackingError}\reqparam{nonsiderealTrackingError}).
The discussion notes that {\it ``this is standard capability for modern telescopes"}, but might be relevant to some 
Special Programs.


\subsection{Data Management Subsystems Requirements (DMSR)}

Version 9 (revision 2021-02-12), \citeds{LSE-61}. 

Note that Version 8.3 (2020-05-04) was an update for LCR-2265, which updated requirements, specifications, and 
discussions regarding the processing of Special Programs data based on earlier versions of this DMTN.

In Section 1.2.3, ``Raw Science Image Metadata", DMS-REQ-0068\dmreq{0068} specifies that {\it ``for each raw 
science image, the DMS shall store image metadata"} including {\it ``Program metadata (identifier for main survey, 
deep drilling, etc.)"}.
The discussion clarifies that {\it ``the program metadata should be sufficient to associate an image with a specific 
Special Program so that DMS-REQ-0320 and DMS-REQ-0397 can be satisfied"}.

In Section 1.3.13, ``Alert Content", the discussion for DMS-REQ-0274\dmreq{0274} explains that the {\it ``program 
and/or scheduler metadata"} included in an alert packet {\it ``should be sufficient to identify whether the image is 
associated with a Special Program (such as an in-progress Deep Drilling Field)"}.

In Section 1.4.18.1, ``Produce All-Sky HiPS Map", the discussion for DMS-REQ-0379\dmreq{0379} raises the point that 
generating separate HiPS maps for Special Programs (e.g., DDFs) remains an open question.

In Section 1.4.18.5, ``Produce MOC Maps", DMS-REQ-0383\dmreq{0383} specifies that Data Release processing 
{\it ``shall include the production of Multi-Order Coverage maps for the survey data"}, and that {\it ``additional 
MOCs SHOULD be produced to represent special-programs datasets"}.
It is noted that a separate technical note would be created to define these MOCs.

The bulk of the DMS's requirements related to Special Programs are in Section 1.6 of the DMSR.

In Section 1.6.1, ``Processing of Data From Special Programs", DMS-REQ-0320\dmreq{0320} specifies that {\it ``it 
shall be possible for special programs to trigger their own data processing recipes, during the night instead of 
the nightly Alert Processing (but the recipes may still issue Alerts), or on alternative timescales"}.
The discussion clarifies that the {\it ``LSST will provide these recipes ... when possible, which includes cases 
where DM can run original or reconfigured versions of existing pipelines, and excludes cases where the development 
of new algorithms, or the allocation of significant additional computational resources, are required. An example of 
an alternative timescale is a nightly trigger to coadd all the deep-drilling field images. Decisions about which 
recipes are applied to which Special Programs will be made by the Operations team, after consideration of the 
scientific goals, computational resources, and data rights policy"}.
This requirement is derived from OSS-REQ-0392, which is essentially a flow-down of requirements from the LSR (0122, 
0075, and 0121).

In Section 1.6.2, ``Prompt/DR Processing of Data from Special Programs", DMS-REQ-0397\dmreq{0397} specifies that 
{\it ``it shall be possible for special programs data to be processed with the prompt and/or annual-release pipelines 
alongside data from the main survey"}.
The discussion further clarifies that {\it ``the data from Special Programs should only be included ... when it is 
(a) possible ... to do so without additional effort and (b) beneficial to the LSST's main science objectives. 
Decisions about which data are included ... will be made by the Operations team"}.
This requirement is also derived from OSS-REQ-0392, which is essentially a flow-down of requirements from the LSR 
(0122, 0075, and 0121).

In Section 1.6.3, ``Level 1 Processing of Special Programs Data", DMS-REQ-0321\dmreq{0321} specifies that {\it ``all 
[Prompt] processing from special programs shall be completed before data arrives from the following night's 
observations"}.
This is essentially adding a quantifier to DMS-REQ-0397, to specify that {\it ``when it is (a) possible ... to do so"} 
means when it is possible to complete the processing before the next night's observations.
This requirement is also derived from OSS-REQ-0392, which is essentially a flow-down of requirements from the LSR 
(0122, 0075, and 0121).

In Section 1.6.4, ``Constraints on Level 1 Special Program Products Generation", DMS-REQ-0344\dmreq{0344} specifies 
that {\it ``the publishing of [Prompt] data products from Special Programs shall be subject to the same performance 
requirements of"} 24 hours ({\tt L1PublicT}\reqparam{L1PublicT}) for the release of Prompt data products and 1 minute 
({\tt OTT1}\reqparam{OTT1}) for the transmission of Alert packets.
This is essentially a more detailed version of DMS-REQ-0321 which includes the Alert production timescale.
This requirement is also derived from OSS-REQ-0392, which is essentially a flow-down of requirents from the LSR 
(0122, 0075, and 0121).

In Section 1.6.5, ``Special Programs Database", DMS-REQ-0322\dmreq{0322} specifies that {\it ``data products for 
special programs shall be stored in databases that are distinct from those used to store standard [Prompt] and 
[Data Release] data products"} and that {\it ``it shall be possible for these databases to be federated ... to allow 
cross-queries and joins"}.
This requirement is also derived from OSS-REQ-0392, which is essentially a flow-down of requirements from the LSR 
(0122, 0075, and 0121).

In Section 4.1.16, ``Level 2 and Reprocessed Level 1 Catalog Access", DMS-REQ-0313\dmreq{0313} specifies that 
{\it ``the DMS shall maintain ... versions of the most recent catalogs generated from Special Programs data"}. 
As with all LSST data, {\it ``there is no requirement for older data releases to be queryable"}.


\subsection{Data Management Applications Design (DMAD)}

Version 4.3 (revision 2020-11-10), \citeds{LDM-151}.

The DMAD is not a requirements document.
Instead, it describes the scientific design of the LSST Science Pipelines: the algorithms and software that will 
be implemented to meet the requirements for processing the LSST data. 

Special Programs are only mentioned a few times, either as a potential source of single-snap visits or as a 
potential source of reference images or catalogs (e.g., training sets).

As described above (e.g., LSR-REQ-0122), the LSST system shall deliver unique and separate data products for visits 
from Special Programs whenever this (1) enables the intended science of the Special Program and (2) can be 
accomplished using the original or reconfigured versions of the LSST Science Pipelines.
For cases in which the development of new algorithms or the allocation of significant additional computational 
resources are required to produce Special Programs data products, User-Generated pipelines and processing will be 
necessary.

The DMAD can be used as the reference document to decide whether a given Special Program will require User-Generated pipelines and processing.


\subsection{Data Products Definitions Document (DPDD)}

Version 3.6 (revision 2021-12-17), \citeds{LSE-163}.

Section 6 describes the data products for Special Programs.
The DPDD is not a requirements document; Section 6 summarizes the requirements presented above and does not introduce 
any new constraints or new information about Special Programs. 


\section{Potential Hardware Boundaries on Data Diversity}\label{sec:hardbounds}

The potential boundaries on the diversity of data products that could be imposed by 
limitations from the Rubin Observatory hardware (camera, telescope, and/or site) 
are considered.

\subsection{Filter Changes}

The maximum time for filter change is 120 seconds: 30 seconds for the telescope to 
reorient the camera to its nominal zero angle position on the rotator, and 90 seconds to 
the camera subsystem for executing the change (OSS-REQ-0293; \citeds{LSE-30}).

Assuming that most Special Programs would be designed to keep overheads $<$100\% and would 
be using standard 30 second visits, the filter change time indicates that it is likely 
that at least 4 exposures in a given filter would be obtained between filter changes, but 
this is not actually a hardware boundary.

The filter change mechanism is designed to undergo a total of 100000 changes over its 
lifetime, and each filter is designed to support up to 30000 changes over its lifetime, 
where lifetime is 15 years.

That is an average of $\sim$27 changes per day, some of which would occur in they day 
during calibrations (estimate, $\sim$10) and the rest at night.

As stated in the filter change memorandum (\url{ls.st/spt-494}), {\it ``the system could 
support as many changes involving the 5 filters loaded in the carousel as desired, without 
any practical limitation"}.

\subsection{Filter Carousel Loads}

The filter carousel can hold five of the six LSST filters at a time. 
The system is designed to support $3000$ loads in $15$ years (\url{ls.st/spt-494}). 
Filter loads are only done in the day, and there will never be data in more than five 
filters in a given night.

\subsection{Exposure Times}

The minimum exposure time is $1$ second, with a stretch goal of $0.1$ seconds 
(OSS-REQ-0291; \citeds{LSE-30}). 
The maximum exposure time is not restricted.

\subsection{Readout Time}

The readout time is $2$ seconds, and would be significant overhead on short exposures.

\subsection{Inter-Image Time}

Images with exposure times $<15$ seconds {\it might} still have to be separated by $15$ 
seconds for thermal tolerance; i.e., the minimum readout rate might be one image every 
$15$ seconds, regardless of exposure time (OSS-REQ-0291; \citeds{LSE-30}).

As discussed in Jira ticket DM-12573, the main issue is thermal and is related to the 
shutter, both the motors and the brakes; an elevated Camera skin temperature would affect 
image quality.

As of 2022, early tests suggest that a sustained (30 minutes) sequence that increases the 
heat load by large factors would not work, but further functional testing of the system 
once the Camera was fully assembled are needed for full characterization of the issue.

This potential $15$ interval between images is also a potential hardware boundary on the 
potential diversity of data products.

\subsection{Telescope Slew}

As described in \citeds{Document-28382}, large slews would have considerable overheads, 
but there are no hardware boundaries on the size of a single slew or the accrued slew 
distance.

\subsection{Pointing and Sidereal Tracking}

The specifications for the telescope's pointing and tracking in \citeds{LSE-30} indicate 
that $<$0.2 arcsecond precision in field pointing (OSS-REQ-0302) and $<$1 arcsecond in 
open-loop tracking (OSS-REQ-0303) would not be possible, but guiding would improve the 
latter (OSS-REQ-0305).

Furthermore, obtaining the \emph{exact same} alignment of the pixel grid in RA-Dec 
{\it ``would put demands on the camera rotator that were not planned"}\footnote{As per 
C. Claver's comments in ticket DM-12573.}.

\subsection{Non-Sidereal Tracking}

The requirement that the LSST system be able to perform non-sidereal tracking is set by 
OSS-REQ-0380 in \citeds{LSE-30}. 
This capability will include angular rates of up to 220 arcseconds per second in both 
azimuth and elevation. 

\subsection{Camera Rotation}

The requirements on the rotator's capabilities do not set any limits on the per-night or 
total lifetime rotation (OSS-REQ-0301, -0300; \citeds{LSE-30}) which might put boundaries 
on the distance between successive visits or the ability to jump between two widely 
separated fields.

Currently, there are no hardware boundaries imposed by camera rotation constraints on the potential diversity of data products.


\section{Potential Processing Boundaries}\label{sec:procbounds}

The capability of the LSST Science Pipelines to process diverse data is explored below.

Note that processing boundaries might ultimately be defined not by what is technically 
possible, but by the resulting image quality parameters, e.g., the number of stars with 
sufficient flux for photometric calibration.

Furthermore, the processing boundaries might not be fully constrained until the final 
performance of the LSST Science Pipelines, as described in the Data Management 
Applications Design, \citeds{LDM-151}) document, is fully characterized.

{\bf Summary.}\\

\begin{itemize}
\item \bf{Short or long exposures.} Very short ($<$2 sec) exposures could be difficult to 
process due to an incompletely-formed PSF. Short (or long, $>$150 sec) exposures 
could be difficult to calibrate due to having too few (or too few unsaturated) stars.
\item \bf{Bright backgrounds (twilight).} It is currently unclear whether images with very 
bright sky backgrounds (twilight images) can be processed with the LSST Science Pipelines, 
or whether user generated pipelines will be needed.
\item \bf{Non-sidereal tracking.} The full reduction and calibration of images obtained 
with non-sidereal tracking, in which the stars are streaked, is currently beyond the scope 
of the LSST Science Pipelines, and will require a user generated pipeline.
\end{itemize}


\subsection{Exposure Times}\label{ssec:procbounds_expt}

Images which deviate significantly from the $15$ second duration for the WFD main survey 
may encounter issues in the instrument signature removal routine, in the correction for 
differential chromatic refraction, in the difference imaging analysis pipeline, and/or in 
the photometric and astrometric calibrations due to a differently sampled set of standard 
stars per CCD.

\subsubsection{Short Exposures (Non-Standard Visits of $<$30 sec)}

The LSST System Requirements document states that {\it ``The LSST shall be capable of 
obtaining and processing exposures not taken in a standard visit mode including those 
with a minimum exposure time of} {\tt minExpTime}", which is 1 second (stretch goal 0.1 
seconds; LSR-REQ-0111 in \citeds{LSE-29}).

However, for exposure times there are other considerations, as changing the exposure time 
also affects the photometric and astrometric calibrations.
Assuming that 1 second exposure can be reduced and calibrated, its detected point sources 
will span a dynamic range of $r$$\approx$ 13 to 21 magnitudes.
A template image built on 15 second exposures will saturate at $r$$\approx$15.8 mag, but 
this still leaves stars between 15.8 and 21.0 magnitudes to be used in the PSF-matching 
(and all other filters have a similarly large overlap).

In order for an image to be successfully PSF-matched to the template, the PSF must be 
well formed (no speckle pattern), and have a spatial variance that the pipeline is 
capable of modeling (be smoothly varying on some minimal scale).
As a simple demonstration, Figure \ref{fig:expt} shows that perhaps exposure times 
shorter than $2$ seconds do not have a well-formed PSF (using the centroid of a 2D 
Gaussian fit as a proxy for "well-formed").

\begin{figure}
\begin{center}
\includegraphics[width=14cm,trim={0cm 0cm 0cm 0cm}, clip]{figures/exptime.png}
\caption{At left, Arroyo atmosphere-only simulated PSF for LSST (with oversampled pixels) 
with exposure times of 0.5, 2, and 15 seconds (top to bottom), courtesy of Bo Xin. At 
right, blue and purple lines show the location of the centroid derived from a 2D Gaussian 
fit to the PSF as a function of exposure time, with the red dashed line showing the true 
center. We can see that for exposure times greater than 2 seconds, the centroid converges 
near its true value. \label{fig:expt}}
\end{center}
\end{figure}

\subsubsection{Long Exposures (Non-Standard Visits of $>$30 sec)}

There is no maximum exposure time specified for an LSST image.
Given that the template image will be a stack of at least a year or two of data, 
processing a $5$--$10$ times deeper single image through the difference imaging pipeline 
should be fine.

However, a $2\times150$ second exposure would saturate at $r \approx 18.3$, perhaps 
leaving too few stars overlapping with e.g., templates or WFD images, for astrometric and 
photometric calibrations.

Furthermore, cosmic-ray rejection completeness might be reduced for longer exposures 
(unknown), which could impact the quality of a difference image and the detected sources. 
Additionally, any system qualities that vary on short (but $>30$ second) timescales could 
inhibit photometric calibration (e.g., tracking).

\subsection{Twilight Images with a Bright Background}

Images obtained during twilight for scientific purposes are also likely to have shorter 
exposure times, and so the issues described in Section \ref{ssec:procbounds_expt} also 
apply here.

Whether or not bright-background images can (or shall) be fully processed -- reduced, 
calibrated, background-subtracted, and delivered with astrometric and photometric 
solutions -- or whether this will require a user generated pipeline, remains to be 
determined (see also the example in Section \ref{ssec:SPCS_Twilight}). 
This may depend on the exposure time and the number of stars available in the image.

\subsection{Images Obtained with Non-Sidereal Tracking}

Non-sidereal tracking leads to images in which stars are streaked, but the moving object 
appears as a point source.

Full processing -- providing reduced, calibrated, background-subtracted images that are 
delivered with astrometric and photometric solutions -- of these images is beyond the 
scope of the DM pipelines as it would require the development of new algorithms, and will 
need to be done as a user generated pipeline. 
The first steps of such a pipeline, such as Instrument Signature Removal, will probably 
be possible to achieve by reconfiguring the relevant DM software tasks.


\section{Previously Proposed Special Programs}\label{sec:prevpropsp}

{\bf This section has not been updated since 2018.}

In this section we compile information about the science goals and observational methods for Special Programs that have been previously proposed or discussed in the Science Community. We use these to infer the potential deviations from standard visit images, and to get a basic idea of the DM processing needs that would be required to enable the science. The main resources from which we have collected information about the Community's Special Program are: \citep{2008arXiv0805.2366I}; \citep{LPM-17}; the LSST Deep Drilling Field white papers from 2011\footnote{\url{https://project.lsst.org/content/whitepapers32012}}; presentations by Niel Brandt and Stephen Ridgway at the LSST Project and Community Workshop in August 2016\footnote{\url{https://project.lsst.org/meetings/lsst2016/sites/lsst.org.meetings.lsst2016/files/Brandt-DDF-MiniSurveys-01.pdf} and \url{https://project.lsst.org/meetings/lsst2016/sites/lsst.org.meetings.lsst2016/files/Ridgway-SimulationsMetrics_1.pdf}}; \citep{2013arXiv1304.3455G}; and Chapter 10 of \citep{2017arXiv170804058L}.

So far, only one aspect of the LSST Special Programs are set: the locations of the four chosen deep drilling fields\footnote{\url{https://www.lsst.org/scientists/survey-design/ddf}}. There are three mini-survey areas that have been discussed extensively by the Science Community: the North Ecliptic Spur (NES), the South Celestial Pole, and the Galactic Plane (see Figure 8 of \citep{2008arXiv0805.2366I}). In Table \ref{tab:ddfms} we list the four extragalactic deep drilling fields have already been specified, along with an \textit{incomplete} list of potential mini-surveys that have been openly discussed in the Science Community. In Section \ref{sec:SPCS}, we create detailed DM Processing Case Studies for several of these Special Programs in order to identify any potential issues with reconfiguring the DM pipelines to create specific data products for these programs.

\begin{table}[h]
\begin{center}
\begin{footnotesize}
\caption{Approved DDF and Incomplete List of Potential Special Programs.}
\label{tab:ddfms}
\begin{tabular}{lll}
\hline \hline
Name & Coordinates & Description  \\
\hline
DDF Elias S1    & 00:37:48, -44:00:00  & approved, cadence TBD \\
DDF XMM-LSS & 02:22:50, -04:45:00  & approved, cadence TBD  \\
DDF Extended Chandra Deep Field-South & 03:32:30, -28:06:00  & approved, cadence TBD  \\
DDF COSMOS  & 10:00:24, +02:10:55 & approved, cadence TBD  \\
\hline
North Ecliptic Spur      & & solar system objects (find and characterize) \\
Galactic Plane             & & more intensive stellar surveying \\
South Equatorial Cap  & & S/LMC and more Galactic science \\
Twilight                        & & short exposures (0.1s) for bright stars \\
Mini-Moons                     &  & finding mini-moons \\
Sweetspot                       & & 60 deg from Sun for NEOs on Earth-like orbits \\
Meter-Sized Impactors     & & detection a week before impact \\
GW Optical Counterparts & & search and recovery \\
Old Open Cluster M67      & dec +12 & compact survey above Galactic plane  \\
\hline
\end{tabular}
\end{footnotesize}
\end{center}
\end{table}

Here we consider a variety of scientific fields in turn, the Special Programs that have been discussed in that Science Community so far, and the implications of these Programs for the diversity of data and data products. Generally, the types of LSST Special Programs that are open for proposals include: (i) additional deep drilling fields; (ii) refined observing strategies for deep drilling fields; (iii) optimized survey areas for the NES, South Pole, and Galactic Plane; (iv) refined observing strategies for the NES, South Pole, and Galactic Plane; and (v) additional mini-surveys (areas and observing strategies).

\medskip
\noindent \textbf{A Nominal DDF Observing Strategy -- } Ivezi\'{c} et al. (2008, \citep{2008arXiv0805.2366I}; Section 3.1.2) describes a nominal DDF data set as $\sim50$ consecutive $15$ second exposures in each of four filters, repeated every two nights for four months. Each exposure would have a $5\sigma$ limit of $r\sim24$; the nightly stack would have a limit of $r\sim26.5$; and the final deep stack of all exposures would have a limit of $r\sim28$. This description does not comment on the processing mode, but, depending on the science goals the exposures could be done as either a series of 50 non-standard visits ($1\times15$ seconds) or 25 standard visits ($2\times15$ seconds). 

\medskip
\noindent \textbf{Solar System Objects (SSO) -- } Four of the mini-surveys in Table \ref{tab:ddfms} have science goals related to studies of SSO. Observations of the North Ecliptic Spur area could yield more $\geq140$ m near-earth objects (NEOs) for the final LSST sample (reference: Brandt's talk). The Mini-Moons Mini-Survey aims to find and study the temporarily captured satellites of the Earth (Section 10.2, \citep{2017arXiv170804058L}). The Sweetspot Survey would use twilight fields to find NEOs in Earth-like orbits (i.e., these objects are never in opposition fields, but overhead at sunrise/sunset; Section 10.2, \citep{2017arXiv170804058L}). The Meter-Sized Impactors program would find and track meter-sized impactors $<2$ weeks before impact (Section 10.2, \citep{2017arXiv170804058L}). {\bf Summary:} most of these science goals do not seem to require non-standard visits or exposure times, with the exception of the Sweetspot survey which occurs during twilight and thus may require shorter exposures. The cadence and patterns of these mini-surveys may differ from the WFD main survey, especially when very fast-moving objects are sought. From a processing perspective, it seems that many of these science goals will be achievable by using the products of Solar System Processing, which runs on the Prompt Pipeline's \texttt{DIASource} catalogs after they are updated each night. The exception is finding faint SSOs (e.g., Trans-Neptunian Objects Trojans, asteroids, long-period comets, dwarf planets) through shift-and-stack (SAS) processing \citedsp{Document-11013}, because SAS is not a capability being built within the DM system and cannot be done solely by reconfiguring DM pipelines. An example of user-generated pipeline for SAS is described in Section \ref{sec:SPCS}.

\medskip
\noindent \textbf{Stars in the Milky Way and Magellanic Clouds -- } As described in \citedsp{Publication-141}, mini-surveys of the Galactic Plane can better distinguish faints stars from faint red galaxies by including at least 3 filters of coverage (e.g., $izy$; similar to WFD), and could mitigate losses from proper motion and increase the detection rate of stellar flares by obtaining all the images in short time span (i.e., a more concentrated cadence than the WFD).  As described in \citedsp{Publication-145}, applying the nominal DDF observing strategy over the full area of the Large and Small Magellanic Clouds can characterize stellar variability to $M_V<6.5$ on timescales from 15 seconds to 3 days. For this, special co-adds may be required, e.g., \textit{"to reach variability levels of 0.1 to 0.005 mag will require co-adds depending on the timescale of the particular variables"} \citedsp{Publication-145}. The Twilight survey in Table \ref{tab:ddfms} proposes short exposures to enable bright stars to be put on the same photometric system as the deeper LSST WFD main survey catalog, and enable science that is based on their long monitoring baselines from historical observations. In Chapter 10.4 of \citep{2017arXiv170804058L}, a proposed short-exposure survey of M67 would use the camera's stretch goal of $0.1$ second exposures or, if that is not possible, \textit{"custom pixel masks to accurately perform photometry on stars as much as $6$ magnitudes brighter than the saturation level"}. {\bf Summary:} while some of these science goals can be accomplished with standard visits, MW \& L/SMC science goals are likely to request shorter exposure times, perhaps down to $0.1$ seconds. These science goals are also likely to propose cadence and filter distributions that are significantly different from the WFD main survey. From a processing perspective, the science goals depending on shorter exposures will only be able to be met by reconfiguring the DM pipelines if the short exposures can be shown to successfully be processed (with, e.g., instrument signature removal); the science goals can likely be met with data products in the same format as the Prompt or DR Pipeline (i.e., {\tt Source} and {\tt Object} catalogs, single visits and deep CoAdds). Although it is not mentioned in the above paragraph, the MW \& L/SMC science community is also most likely to require special processing to extract information from saturated stars, which is outside the scope of DM. See Section \ref{ssec:SPCS_GPVSEx} for more detailed DM processing case studies.

\noindent \textbf{Exoplanets -- } As described in Section 3.1.2 of \citep{2008arXiv0805.2366I}, transiting exoplanets could be detected with the nominal DDF plan, which would allow for $1\%$ variability to be detected over hour-long timescales; a DDF field at Galactic latitude $30$ degrees would yield $10^6$ stars at $r<21$ that would have $\mathrm{SNR}>100$ in each single exposure of the sequence. \citep{2013arXiv1304.3455G} describes how transits can be extract from a wider-area survey of the Galactic Plane, and how microlensing candidates can be found with $\sim22$ mag imaging over the Galactic Plane region every 3-4 days (since microlensing events are slower; these would then require follow-up with external facilities). Dealing with the more crowded fields would be mitigated by the shallower images, in this case. One of the main points of \citep{2013arXiv1304.3455G} is that the Galactic Plane can yield a lot of science despite the fact that its eventual deep co-adds would be uselessly confusion limited, and therefore should not be skipped. \textbf{Summary.} Some of these science goals appear possible with standard visit images, and some might request shorter exposures to avoid confusion in crowded fields when the science can be done with brighter stars. From a processing perspective, the science goals are likely to be achievable with reconfigured DM pipelines, but this depends heavily on performance in crowded fields. See Section \ref{ssec:SPCS_GPVSEx} for a more detailed DM processing case study for Galactic Plane regions.

\noindent \textbf{Supernovae -- } The nominal DDF plan described in \citep{2008arXiv0805.2366I}, which builds nightly stacks with a limit of $r\sim26.5$ out of standard visit images, would extend the SN sample to $z\sim1.2$ and provide more densely sampled light curves for cosmological analyses. The optimal exposure time distribution might be 6, 5, 10, 10, 9, 10 in $ugrizy$ \citedsp{Publication-144}. High-cadence observations of DDF would be the only way to detect fast transients, particularly extragalactic novae, some tidal disruption events, optical counterparts to gamma-ray bursts, and peculiar SNe \citep{2014ApJ...794...23D}. Generating the best-possible individual SN light curves for cosmological analyses requires building special, deep-as-possible, SN-free host galaxy images and using them as a template. This will also be necessary for studying SNe that appear in the template image; i.e., that last $>1000$ days. These are mostly Type IIn, probably explosions of massive stars into dense circumstellar material, which are not used for cosmology but rather to study late-stage stellar evolution and mass loss. SN-free images will also be needed to measure correlated properties for cosmology and to do host-galaxy science. The latter, specifically the "characterization of ultra-faint SN host galaxies", is also mentioned in the Galaxies DDF WP \citedsp{Publication-142}. Short-exposure observations of bright, nearby SNe may also be useful to include near-peak photometry in the LSST magnitude system, and enable full light-curve analyses. \textbf{Summary.} All of these science goals appear possible with standard visit images (with the exception of a target-of-opportunity short-exposure program to observe bright SNe). From a processing perspecitve, the science goals appear to be accessible with reconfigured DM pipelines to stack and difference the data. In particular, the DRP codes to create "transient-free CoAdds" will be suitable for generating the SN-free templates for DDF, as they will do for the Main Survey images. See also Section \ref{ssec:SPCS_SNDDF} for a DM processing case study to find SNe in a DDF.

\noindent \textbf{Galaxies -- } The additional depth of a DDF may provide access to a larger collection of low-$\mu$ objects. \citedsp{Publication-142} mentions "identification of nearby isolated low-redshift dwarf galaxies via surface-brightness fluctuations" and "characterization of low-surface-brightness extended features around both nearby and distant galaxies". The DDF stacks could also be used to characterize of high-$z$ clusters, although this ability might depend on deblending extended objects. Also, the DDF observations, when combined with the WFD, allow for AGN monitoring on a variety of timescales in well-characterized galaxies \citedsp{Publication-142,Publication-143}. \textbf{Summary.} As with the SN science goals, these use standard visit images and reconfigured DM pipelines to make deep CoAdds and extract sources. In addition, it seems likely that user-generated algorithms that are optimized to detect and characterize particular types of faint extended sources will be needed, and these are beyond the scope of DM.

\noindent \textbf{Weak Lensing -- } The deeper imaging from DDFs can help with shear systematics and the effects of magnification in the analysis of WFD data (community forum, Jim Bosch). \textbf{Summary.} As with the SN and Galaxies DDF-related science goals, these use standard visit images and reconfigured DM pipelines can be used to make deep CoAdds and extract sources, as Jim notes.
%$\bullet$ \textit{Jim Bosch -- "Will need to process at least some deep drilling fields (high-latitude ones) in the same way we process a full data release production before running the full data release production, so we can use the results to build priors and/or calibrate shear estimates on the wide survey"} (\texttt{\Large{Community}} forum) \\
%$\bullet$ \textit{Jim Bosch -- "Will need to process various wide-depth subsets of some deep drilling fields (again, high-latitude ones) using the regular DRP pipeline. We'll definitely want best-seeing, worst-seeing, and probably a couple of independent typical-seeing subsets, but there may be other ways we'd want to subdivide as well."} (\texttt{\Large{Community}} forum)  \\
%$\bullet$ \textit{MLG side note -- Photo-$z$ are very important to weak lensing \citedsp{Document-10963} and so perhaps the implemented method should be chosen with weak lensing science prioritized.} \\

\end{document}
