\section{Executive Summary} \label{sec:execsum}

The three memoranda for Rubin Data Management regarding Special Programs are:

\begin{enumerate}

\item All visits should have both a region and a program label.
These labels should be propagated to all derived data products, such as
processed images, catalog data, and alerts. (Section~\ref{ssec:sci_labels})

\item All visits that \emph{can} be processed by the Prompt pipelines and generate 
alerts \emph{should} be, in support of time domain and Solar System science goals. (Section~\ref{ssec:proc_wfd})

\item Special Processing to create ``unique and separate" data products for Special Programs
should be done when it is possible and necessary, as defined in Section~\ref{ssec:proc_rdm}.

\end{enumerate}

\section{Introduction}\label{sec:intro}

The following is an introduction to the key terms related to Special Programs.

\subsection{Visit Types}

A visit is an observation of a single pointing at a given time, of which 
there are three types:

\begin{itemize}
\item Standard Visit -- Composed of $2\times15$ second exposures (commonly referred to as ``snaps").
\item Alternative Standard Visit -- Composed of a single $30$ second exposure.
\item Non-Standard Visit -- Any other exposure time(s) or number of snaps.
\end{itemize}

Non-standard visits with shorter or longer exposure times are being 
considered for some Special Programs.

\subsection{Wide Fast Deep (WFD) Program}

The WFD is the core science program of the LSST, designed to achieve the science 
goals defined by the Science Requirements Document (SRD; \citeds{LPM-17}).

The WFD program is defined from a scheduling perspective as a certain observing strategy
(i.e., a set of cadences and filter balances) applied to a low-dust region 
(extragalactic areas) and several special regions:
the Galactic bulge and plane; the South Celestial Pole (SCP); the North Ecliptic Spur (NES);
and the Virgo Cluster \citedsp{PSTN-055}.

The WFD is expected to use alternative standard visits in the $u$-band and 
standard visits in all other bands \citedsp{PSTN-055}.
Different cadences, filter balances, rotational dithers, etc., might
be applied in the different regions composing WFD.
For example, rolling cadence is expected to be implemented in the low-dust region only.

WFD program data will be processed with regular (non-special) processing.
This processing will produce, among other data products (LSE-163), 
a contiguous sky footprint that covers at 
least $\sim$18000 deg$^2$ with $\gtrsim$800 visits per field, 
and is expected to be accomplished with 85--90\% of the observing time.
This contiguous all-sky "deep coadd" may have region-dependent processing parameters and inputs
(e.g., different calibration parameters or deblending algorithms for crowded fields)
and will be of variable depth, as most special regions recieve fewer visits.

\subsection{Regular (non-special) processing}

This is a term used only in this document to refer to the image processing 
described in the Data Products Definitions Document
(LSE-163) that are designed for, and will be applied to, the WFD program's observations.

In some cases, regular processing is also appropriate for Special Programs.

\subsection{Special Programs} 

This is a Rubin Data Management term used to refer to sky regions within
or beyond the WFD footprint which have a distinct observing strategy from
the WFD program's plans for that region (e.g., different visit type, cadence, filter distribution).

Special Programs are typically driven by specific science goals that build on or 
add to the core science pillars of the LSST.

About 10--15\% of the total 10-year LSST time-span will be spent obtaining 
observations associated with Special Programs.
Special Programs includes non-WFD LSST components such as the Deep Drilling 
Fields (DDFs) and the mini-, micro-, and nano-surveys \citedsp{PSTN-055}.

Data products from Special Programs is subject to the Rubin Data Policy \citedsp{RDO-013}
in the same way as data products from the WFD program.

Some science goals for Special Programs can be met with regular processing,
but some will require Special Processing by Rubin Data Management or user-generated processing.

\subsection{Special Processing}

This is a Rubin Data Management term to describe processing that 
uses components of the LSST Science Pipelines and is applied by Rubin
Data Management to images from Special Programs.
Special Processing creates data products that are unique and separate from those produced
by regular processing for the WFD program.

Special Processing is likely to use different inputs or configurations for the
LSST Science Pipelines, or to run on different timescales, than regular processing - 
as appropriate for the Special Programs' data and science goals.
However, the development and application of specialized \emph{algorithms} or new software
is beyond the scope of Special Processing.

\subsection{User-Generated Processing}

Any processing of Rubin data done by users in order to reach specific science goals, including
processing for Special Programs data, is referred to as User-Generated Processing.

User-Generated Processing for Special Programs data would be necessary in cases where
the science goals require custom algorithms, software, or very large computational
capacities which are beyond the scope of Special Processing or the Rubin-provided
computational resources (Section~\ref{ssec:sci_comp}).

Guidelines for User-Generated Processing, and for user-generated data products
that can be federated with the Rubin-product data products (i.e., joinable tables),
is forthcoming.

\subsection{Deep Drilling Field (DDF)}

A single pointing for which many (e.g., a hundred) visits are obtained 
(usually sequentially) during a single night, and repeated every few 
nights.

As of the Phase 2 SCOC recommendations in \citeds{PSTN-055}, the five 
confirmed DDFs were:

\begin{itemize}
\item Elias S1 (00:37:48, -44:00:00)
\item XMM-LSS (02:22:50, -04:45:00)
\item Extended Chandra Deep Field-South (03:32:30, -28:06:00)
\item COSMOS (10:00:24, +02:10:55)
\item Euclid Deep Field South  (04:04:58, -48:25:23)\footnote{\url{https://www.cosmos.esa.int/web/euclid/euclid-survey}}
\end{itemize}

The DDFs will require Special Processing (Section~\ref{ssec:proc_rdm})
and may also benefit from User-Generated Processing.

\subsection{Mini-, Micro- and Nano-Surveys}

Specific sky areas covered by a few hundred, a hundred, or tens of visits (respectively).
This document will refer to them collectively as mini-surveys.

The sky areas of mini-surveys can be within, adjacent to, or detached from the WFD footprint.
Mini-surveys can have non-standard visits.

For a list of the mini-surveys under consideration,
see \citeds{PSTN-055}.

Most mini-surveys are likely to require Special Processing (to be considered on a case-by-case basis),
and some are also likely to require User-Generated Processing.
