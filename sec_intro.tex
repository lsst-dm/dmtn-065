\section{Memoranda} \label{sec:mem}

The three memoranda for Rubin Data Management regarding Special Programs are:

\begin{enumerate}

\item All visits should have a region label \textit{and} an observing mode
label, and these labels should be propagated to processed images and catalog data.
(Section~\ref{ssec:sci_labels})

\item Special Processing should be done by Rubin Data Management to 
produce unique and separate data products
for Special Programs when it is both possible and necessary.
(Section~\ref{ssec:sci_sproc}).

\item All visits that \emph{can} be processed by the Prompt pipelines and generate 
alerts \emph{should} be, in support of time domain and Solar System science goals.
(Section~\ref{ssec:sci_pproc})

\end{enumerate}

\section{Terms and Definitions}\label{sec:intro}

The following is an introduction to the key terms related to Special Programs.

\subsection{Visit Types}

A visit is an observation of a single pointing at a given time, of which 
there are three types:

\begin{itemize}
\item Standard Visit -- Composed of $2\times15$ second exposures (commonly referred to as ``snaps").
\item Alternative Standard Visit -- Composed of a single $30$ second exposure.
\item Non-Standard Visit -- Any other exposure time(s) or number of snaps.
\end{itemize}

Non-standard visits with shorter or longer exposure times are being 
considered for some Special Programs.

\subsection{Region Label}

Based on Right Acension and Declination.

To be applied to data by Rubin Data Management, e.g., using a look-up table.

Region labels would include, e.g., WFD low-dust, Galactic Plane, Galactic 
Bulge, SCP, NES, or Virgo; DDF fieldname; mini-survey region name.

\subsection{Observing Mode Label}

Based on the scheduler mode at the time of the observation
(i.e., survey cadence or strategy).

To be applied by the Survey Scheduler team via the observations scheduler.

Observing mode labels might include, e.g., WFD, DDF, TOO, mini-survey name, or other 
options like engineering, commissioning, and director's discretionary.

\subsection{Main Survey including Wide-Fast-Deep (WFD)}

The LSST Main Survey includes the low-dust Wide-Fast-Deep (WFD) area 
(which includes low-dust extragalactic and Galactic regions) 
and several special regions:
the dusty areas of the Galactic bulge and plane; the South Celestial Pole (SCP); 
the North Ecliptic Spur (NES); and the Virgo Cluster \citedsp{PSTN-055}.
The WFD component is the core component of the LSST Main Survey, designed to achieve the science 
goals defined by the Science Requirements Document (SRD; \citeds{LPM-17}).

The Main Survey will be executed with a set of observing modes, and 
is expected to use alternative standard visits in the $u$-band and 
standard visits in all other bands \citedsp{PSTN-055}.
Different cadences, filter balances, rotational dithers, etc., 
might be applied in the different regions composing Main Survey.
For example, rolling cadence is expected to be implemented in the low-dust 
WFD region only.

Data from the Main Survey will be processed with standard (non-special) processing (Section~\ref{ssec:intro_stdproc}).
This processing will produce, among other data products (LSE-163), 
a contiguous sky footprint that covers at 
least $\sim$18000 deg$^2$ (with $\gtrsim$800 visits per field in WFD regions), 
and is expected to be accomplished with 85--90\% of the observing time.
This contiguous all-sky "deep coadd" may have region-dependent processing parameters and inputs
(e.g., different calibration parameters or deblending algorithms for crowded fields)
and will be of variable depth, as most special regions recieve fewer visits.

\subsection{Special Programs} 

This is a Rubin Data Management term used to refer to sky regions within
or beyond the Main Survey footprint that have a distinct observing strategy from
the Main Survey's plans for that region (e.g., different visit type, cadence, filter distribution).

Special Programs are typically driven by specific science goals that build on or 
add to the core science pillars of the LSST.
They include LSST components such as the Deep Drilling 
Fields (DDFs) and the mini-, micro-, and nano-surveys \citedsp{PSTN-055}.
About 10--15\% of the total 10-year LSST will be spent obtaining 
observations associated with Special Programs.

Data products from Special Programs is subject to the Rubin Data Policy \citedsp{RDO-013}
in the same way as data products from the Main Survey.

Some science goals for Special Programs can be met with standard processing,
but some will require Special Processing by Rubin Data Management or user-generated processing.

\subsection{Special Processing}

This is a Rubin Data Management term to describe processing that 
uses components of the LSST Science Pipelines and is applied by Rubin
Data Management to images from Special Programs.
Special Processing creates data products that are unique and separate from those produced
by standard processing for the Main Survey.

Special Processing is likely to use different inputs or configurations for the
LSST Science Pipelines, or to run on different timescales, than standard processing - 
as appropriate for the Special Programs' data and science goals.
However, the development and application of specialized \emph{algorithms} or new software
is beyond the scope of Special Processing.

Examples of Special Processing include the processing
of images with non-standard exposure times, images obtained
during twilight, or nightly stacking and differencing
of images in deep drilling fields.
See Section~\ref{ssec:proc_special} for more detailed examples of Special Processing.

\subsection{Standard (non-special) processing}\label{ssec:intro_stdproc}

This is a term used only in this document to refer to the image processing 
described in the Data Management Science Pipelines Design \citeds{LDM-151} that produces the data products described in the Data Products Definitions Document
(\citeds{LSE-163}) that are designed for, and will be applied to, the Main Survey's observations.

In some cases, standard processing is also appropriate for Special Programs.


\subsection{User-Generated Processing}

Any processing of Rubin data done by users in order to reach specific science goals, including
processing for Special Programs data, is referred to as User-Generated Processing.

User-Generated Processing for Special Programs data would be necessary in cases where
the science goals require custom algorithms, software, or very large computational
capacities which are beyond the scope of Special Processing or the Rubin-provided
computational resources (Section~\ref{ssec:sci_comp}).

Guidelines for User-Generated Processing, and for user-generated data products
that can be federated with the Rubin-product data products (i.e., joinable tables),
is forthcoming.

\subsection{Deep Drilling Field (DDF)}

A single pointing for which many (e.g., a hundred) visits are obtained 
(usually sequentially) during a single night, and repeated every few 
nights.

As of the Phase 2 SCOC recommendations in \citeds{PSTN-055}, the five 
confirmed DDFs were:

\begin{itemize}
\item Elias S1 (00:37:48, -44:00:00)
\item XMM-LSS (02:22:50, -04:45:00)
\item Extended Chandra Deep Field-South (03:32:30, -28:06:00)
\item COSMOS (10:00:24, +02:10:55)
\item Euclid Deep Field South  (04:04:58, -48:25:23)\footnote{\url{https://www.cosmos.esa.int/web/euclid/euclid-survey}}
\end{itemize}

Processing for the DDF images is likely to be a combination of standard (Prompt) processing,
Special Processing, and User-Generated Processing.

\subsection{Mini-, Micro- and Nano-Surveys}

Specific sky areas covered by a few hundred, a hundred, or tens of visits (respectively).
This document will refer to them collectively as mini-surveys.

The sky areas of mini-surveys can be within, adjacent to, or detached from the Main Survey footprint.
Mini-surveys can have non-standard visits.
Target-of-opportunity (TOO) observations for, e.g., the discovery of optical counterparts to 
multi-messenger astrophysical phenomena, are considered a type of mini-survey in this document.
For a list of the mini-surveys under consideration,
see \citeds{PSTN-055}.

Processing for the mini-surveys is likely to be a combination of standard (Prompt) processing,
Special Processing, and User-Generated Processing.
