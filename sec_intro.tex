\section{Introduction} \label{sec:intro}

Special Programs is a Rubin Data Management (RDM) term used to refer to survey 
components with specific sky regions within or beyond the 
Wide-Fast-Deep (WFD) main survey, and/or distinct cadences or 
strategies from the WFD main survey.

There are several requirements on the Rubin Data Management System (DMS) 
related to Special Programs, summarized in Section~\ref{sec:req}.
This document recommends implementation aspects for these requirements in order to 
maximize science from Special Programs (Section~\ref{sec:sci}).

Special Programs are driven by specific science goals that build on or 
add to the core science pillars of the LSST.
About 10--15\% of the total 10-year LSST time-span will be spent obtaining 
observations associated with Special Programs (and 85--90\% for WFD).

Special Programs includes non-WFD LSST components such as the Deep Drilling 
Fields (DDFs) and the mini-, micro-, and nano-surveys, some of which 
might use non-standard visits with different exposure times
(Section~\ref{ssec:intro_terms}).


\subsection{Additional Terms and Definitions}\label{ssec:intro_terms}

\textbf{Wide-Fast-Deep (WFD) Main Survey -- }
The core science program of the LSST, designed to achieve the science 
goals defined by the Science Requirements Document (SRD; \citeds{LPM-17}).
It is a contiguous area that covers at least $\sim$18000 deg$^2$ with 
$\gtrsim$800 visits per field, expected to be accomplished with 85--90\% 
of the observing time.

As of the Phase 2 SCOC recommendations in \citeds{PSTN-055}, the WFD area 
is considered to include the low dust extinction extragalactic regions; 
the higher extinction, high stellar density Galactic Plane (GP) and Bulge; the 
Virgo Cluster; the South Celestial Pole (SCP); and the North Ecliptic Spur 
(NES). 

In the past the Galactic, SCP, and NES regions have been referred to as 
``mini-surveys" (see below) but this does not make them "Special Programs". 
From a survey-scheduling perspective, they are now considered part of the 
WFD survey (whether or not the scheduler software treats them as such).

From a science and data-processing perspective, these components of the 
WFD do not require special processing or separate data products 
(i.e., do not need to be treated by RDM as ``Special Programs").
This is because they will use standard (or alternative) visits and can
be incorporated into WFD Prompt processing (Section~\ref{ssec:proc_wfd}), 
and because their distinct areas do not overlap with the other WFD components 
and so do not require, e.g., separate deep coadds. 


\textbf{Deep Drilling Field (DDF) -- }
A single pointing for which many (e.g., a hundred) visits are obtained 
(usually sequentially) during a single night, and repeated every few 
nights.

As of the Phase 2 SCOC recommendations in \citeds{PSTN-055}, the five 
confirmed DDFs were:

\begin{itemize}
\item Elias S1 (00:37:48, -44:00:00)
\item XMM-LSS (02:22:50, -04:45:00)
\item Extended Chandra Deep Field-South (03:32:30, -28:06:00)
\item COSMOS (10:00:24, +02:10:55)
\item Euclid Deep Field South  (04:04:58, -48:25:23)\footnote{\url{https://www.cosmos.esa.int/web/euclid/euclid-survey}}
\end{itemize}

From a science and data-processing perspective, the DDFs do require 
special processing and separate data products (Section~\ref{ssec:proc_rdm}).


\textbf{Mini-, Micro- and Nano-Surveys -- }
Specific sky areas covered by a few hundred, a hundred, or tens of visits 
(respectively; as in Phase 2 SCOC recommendations, \citeds{PSTN-055}).
For a list of the current mini-surveys under consideration, see 
\citeds{PSTN-055}.
This document refers to them all together as mini-surveys.

These sky areas are all (so far) either within or adjacent to the WFD area.
In some cases these visits are non-standard (e.g., short exposures).

From a data-processing and science perspective, most mini-surveys are 
likely to require special processing or separate data products, and should 
be considered on a case-by-base basis.


\textbf{Visit Types -- }
A visit is an observation of a single pointing at a given time, of which 
there are three types:

\begin{itemize}
\item Standard Visit -- Composed of $2\times15$ second exposures (commonly referred to as ``snaps").
\item Alternative Standard Visit -- Composed of a single $30$ second exposure.
\item Non-Standard Visit -- Any other exposure time(s) or number of snaps.
\end{itemize}

The Phase 2 SCOC recommendations for the WFD (\citeds{PSTN-055}) are to 
use alternative standard visits in the $u$-band and standard visits in all 
other bands; variable exposure times were not recommended for the WFD. 
Non-standard visits with shorter or longer exposure times are being 
considered for some mini-surveys.
