\section{Potential Hardware Boundaries on Data Diversity}\label{sec:hardbounds}

The potential boundaries on the diversity of data products that could be imposed by 
limitations from the Rubin Observatory hardware (camera, telescope, and/or site) 
are considered.

\subsection{Filter Changes}

The maximum time for filter change is 120 seconds: 30 seconds for the telescope to 
reorient the camera to its nominal zero angle position on the rotator, and 90 seconds to 
the camera subsystem for executing the change (OSS-REQ-0293; \citeds{LSE-30}).

Assuming that most Special Programs would be designed to keep overheads $<$100\% and would 
be using standard 30 second visits, the filter change time indicates that it is likely 
that at least 4 exposures in a given filter would be obtained between filter changes, but 
this is not actually a hardware boundary.

The filter change mechanism is designed to undergo a total of 100000 changes over its 
lifetime, and each filter is designed to support up to 30000 changes over its lifetime, 
where lifetime is 15 years.

That is an average of $\sim$27 changes per day, some of which would occur in they day 
during calibrations (estimate, $\sim$10) and the rest at night.

As stated in the filter change memorandum (\url{ls.st/spt-494}), {\it ``the system could 
support as many changes involving the 5 filters loaded in the carousel as desired, without 
any practical limitation"}.

\subsection{Filter Carousel Loads}

The filter carousel can hold five of the six LSST filters at a time. 
The system is designed to support $3000$ loads in $15$ years (\url{ls.st/spt-494}). 
Filter loads are only done in the day, and there will never be data in more than five 
filters in a given night.

\subsection{Exposure Times}

The minimum exposure time is $1$ second, with a stretch goal of $0.1$ seconds 
(OSS-REQ-0291; \citeds{LSE-30}). 
The maximum exposure time is not restricted.

\subsection{Readout Time}

The readout time is $2$ seconds, and would be significant overhead on short exposures.

\subsection{Inter-Image Time}

Images with exposure times $<15$ seconds {\it might} still have to be separated by $15$ 
seconds for thermal tolerance; i.e., the minimum readout rate might be one image every 
$15$ seconds, regardless of exposure time (OSS-REQ-0291; \citeds{LSE-30}).

As discussed in Jira ticket DM-12573, the main issue is thermal and is related to the 
shutter, both the motors and the brakes; an elevated Camera skin temperature would affect 
image quality.

As of 2022, early tests suggest that a sustained (30 minutes) sequence that increases the 
heat load by large factors would not work, but further functional testing of the system 
once the Camera was fully assembled are needed for full characterization of the issue.

This potential $15$ interval between images is also a potential hardware boundary on the 
potential diversity of data products.

\subsection{Telescope Slew}

As described in \citeds{Document-28382}, large slews would have considerable overheads, 
but there are no hardware boundaries on the size of a single slew or the accrued slew 
distance.

\subsection{Pointing and Sidereal Tracking}

The specifications for the telescope's pointing and tracking in \citeds{LSE-30} indicate 
that $<$0.2 arcsecond precision in field pointing (OSS-REQ-0302) and $<$1 arcsecond in 
open-loop tracking (OSS-REQ-0303) would not be possible, but guiding would improve the 
latter (OSS-REQ-0305).

Furthermore, obtaining the \emph{exact same} alignment of the pixel grid in RA-Dec 
{\it ``would put demands on the camera rotator that were not planned"}\footnote{As per 
C. Claver's comments in ticket DM-12573.}.

\subsection{Non-Sidereal Tracking}

The requirement that the LSST system be able to perform non-sidereal tracking is set by 
OSS-REQ-0380 in \citeds{LSE-30}. 
This capability will include angular rates of up to 220 arcseconds per second in both 
azimuth and elevation. 

\subsection{Camera Rotation}

The requirements on the rotator's capabilities do not set any limits on the per-night or 
total lifetime rotation (OSS-REQ-0301, -0300; \citeds{LSE-30}) which might put boundaries 
on the distance between successive visits or the ability to jump between two widely 
separated fields.

Currently, there are no hardware boundaries imposed by camera rotation constraints on the potential diversity of data products.
