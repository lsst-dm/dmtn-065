\section{Potential Processing Boundaries}\label{sec:procbounds}

The capability of the LSST Science Pipelines to process diverse data is explored below.

Note that processing boundaries might ultimately be defined not by what is technically 
possible, but by the resulting image quality parameters, e.g., the number of stars with 
sufficient flux for photometric calibration.

Furthermore, the processing boundaries might not be fully constrained until the final 
performance of the LSST Science Pipelines, as described in the Data Management 
Applications Design, \citeds{LDM-151}) document, is fully characterized.

{\bf Summary.}\\

\begin{itemize}
\item \bf{Short or long exposures.} Very short ($<$2 sec) exposures could be difficult to 
process due to an incompletely-formed PSF. Short (or long, $>$150 sec) exposures 
could be difficult to calibrate due to having too few (or too few unsaturated) stars.
\item \bf{Bright backgrounds (twilight).} It is currently unclear whether images with very 
bright sky backgrounds (twilight images) can be processed with the LSST Science Pipelines, 
or whether user generated pipelines will be needed.
\item \bf{Non-sidereal tracking.} The full reduction and calibration of images obtained 
with non-sidereal tracking, in which the stars are streaked, is currently beyond the scope 
of the LSST Science Pipelines, and will require a user generated pipeline.
\end{itemize}


\subsection{Exposure Times}\label{ssec:procbounds_expt}

Images which deviate significantly from the $15$ second duration for the WFD main survey 
may encounter issues in the instrument signature removal routine, in the correction for 
differential chromatic refraction, in the difference imaging analysis pipeline, and/or in 
the photometric and astrometric calibrations due to a differently sampled set of standard 
stars per CCD.

\subsubsection{Short Exposures (Non-Standard Visits of $<$30 sec)}

The LSST System Requirements document states that {\it ``The LSST shall be capable of 
obtaining and processing exposures not taken in a standard visit mode including those 
with a minimum exposure time of} {\tt minExpTime}", which is 1 second (stretch goal 0.1 
seconds; LSR-REQ-0111 in \citeds{LSE-29}).

However, for exposure times there are other considerations, as changing the exposure time 
also affects the photometric and astrometric calibrations.
Assuming that 1 second exposure can be reduced and calibrated, its detected point sources 
will span a dynamic range of $r$$\approx$ 13 to 21 magnitudes.
A template image built on 15 second exposures will saturate at $r$$\approx$15.8 mag, but 
this still leaves stars between 15.8 and 21.0 magnitudes to be used in the PSF-matching 
(and all other filters have a similarly large overlap).

In order for an image to be successfully PSF-matched to the template, the PSF must be 
well formed (no speckle pattern), and have a spatial variance that the pipeline is 
capable of modeling (be smoothly varying on some minimal scale).
As a simple demonstration, Figure \ref{fig:expt} shows that perhaps exposure times 
shorter than $2$ seconds do not have a well-formed PSF (using the centroid of a 2D 
Gaussian fit as a proxy for "well-formed").

\begin{figure}
\begin{center}
\includegraphics[width=14cm,trim={0cm 0cm 0cm 0cm}, clip]{figures/exptime.png}
\caption{At left, Arroyo atmosphere-only simulated PSF for LSST (with oversampled pixels) 
with exposure times of 0.5, 2, and 15 seconds (top to bottom), courtesy of Bo Xin. At 
right, blue and purple lines show the location of the centroid derived from a 2D Gaussian 
fit to the PSF as a function of exposure time, with the red dashed line showing the true 
center. We can see that for exposure times greater than 2 seconds, the centroid converges 
near its true value. \label{fig:expt}}
\end{center}
\end{figure}

\subsubsection{Long Exposures (Non-Standard Visits of $>$30 sec)}

There is no maximum exposure time specified for an LSST image.
Given that the template image will be a stack of at least a year or two of data, 
processing a $5$--$10$ times deeper single image through the difference imaging pipeline 
should be fine.

However, a $2\times150$ second exposure would saturate at $r \approx 18.3$, perhaps 
leaving too few stars overlapping with e.g., templates or WFD images, for astrometric and 
photometric calibrations.

Furthermore, cosmic-ray rejection completeness might be reduced for longer exposures 
(unknown), which could impact the quality of a difference image and the detected sources. 
Additionally, any system qualities that vary on short (but $>30$ second) timescales could 
inhibit photometric calibration (e.g., tracking).

\subsection{Twilight Images with a Bright Background}

Images obtained during twilight for scientific purposes are also likely to have shorter 
exposure times, and so the issues described in Section \ref{ssec:procbounds_expt} also 
apply here.

Whether or not bright-background images can (or shall) be fully processed -- reduced, 
calibrated, background-subtracted, and delivered with astrometric and photometric 
solutions -- or whether this will require a user generated pipeline, remains to be 
determined (see also the example in Section \ref{ssec:SPCS_Twilight}). 
This may depend on the exposure time and the number of stars available in the image.

\subsection{Images Obtained with Non-Sidereal Tracking}

Non-sidereal tracking leads to images in which stars are streaked, but the moving object 
appears as a point source.

Full processing -- providing reduced, calibrated, background-subtracted images that are 
delivered with astrometric and photometric solutions -- of these images is beyond the 
scope of the DM pipelines as it would require the development of new algorithms, and will 
need to be done as a user generated pipeline. 
The first steps of such a pipeline, such as Instrument Signature Removal, will probably 
be possible to achieve by reconfiguring the relevant DM software tasks.
