\section{Requirements Related to Special Programs}\label{sec:req}
% \lsrreq, \ossreq, \dmreq
% \reqparam

Appendix~\ref{sec:docrev} provides a full description of the requirements 
on the Data Management System related to Special Programs, which are summarized here.

\subsection{Metadata}\label{ssec:req_meta}

\textbf{Program metadata that is sufficient to trigger Special Processing should be stored. -- }
In order to support Special Programs processing, the LSST system is 
required to store metadata that includes program information for every raw 
image, such as identifiers for images obtained as part of the Main Survey 
or a Special Program (DMS-REQ-0068).
It is required that this metadata be sufficient for Special Programs to 
trigger their own real-time data processing recipes ``whenever possible" 
(DMS-REQ-0320), and be included in alert packets (DMS-REQ-0274).

\subsection{Data Products}\label{ssec:req_dp}

\textbf{Produce unique and special (and joinable) data products when possible. --}
Rubin Observatory and the LSST system (the observatory and the data 
management systems) are required to process Special Programs data to 
produce unique and separate data products ``whenever possible" 
(LSR-REQ-0121).
It is a requirement that these Special Programs data products be distinct 
and joinable with the Prompt and/or Data Release data products (DMS-REQ-0322).

The term ``whenever possible" includes cases where the original or 
reconfigured versions of the LSST Science Pipelines can be run, and 
excludes cases where the development of new algorithms or the allocation 
of significant additional computational resources are required 
(LSR-REQ-0121).

The statement ``to produce unique and separate data products" typically 
refers to producing the same kinds of data products as will be generated 
by the Prompt and Data Release pipelines (processed visit images, coadded 
images, difference images, and catalogs of sources and objects for those 
images).

The term ``joinable" means the Special Programs data products can be federated 
or cross-matched with the relevant Prompt and Data Release data products, and that
a column of cross-matched object identifiers is provided to enable table joins.

\textbf{The size of Special Programs data products should be about 10\% of the total. --}
It is a requirement that the cumulative size of the Special Programs data 
products generated by Rubin Observatory be no more than $\sim$10\% the 
size of the Data Release data products (LSR-REQ-0121).

The spirit of this requirement on data volume is that the size be proportional to the fraction 
of survey time spent on Special Programs.

The derivation of value-added data products, such as HiPS or MOC maps, for 
Special Programs remains an open question (DMS-REQ-0379, 0383), and is not required.
However, they are suggested to be produced with Special Processing where
appropriate (Section~\ref{ssec:proc_special}).

\subsection{Processing}\label{ssec:req_proc}

\textbf{Latency requirements apply to alerts from Special Programs images. --}
It is a requirement that any Special Programs processing done with the 
Prompt pipeline (or a reconfigured version of it) is subject to the same 
timescales and latency constraints of 24 hours for the release of Prompt 
data products and 1 minute for the transmission of Alert packets 
(DMS-REQ-0344).

\textbf{Intermediate timescales should be used to enable science when needed. --}
It is also a requirement that Special Programs processing be done on 
timescales intermediate to the Prompt and Data Release processing, 
``whenever possible" and whenever necessary to enable the intended science 
goals of the Special Program (LSR-REQ-0032).

\textbf{Exposure times of 1 second should be processable. -- }
It is a requirement that the LSST system be able to process non-standard 
visits with short exposure times as low as 1 second, with a discussion 
note that such short exposures might have degraded image quality 
(LSR-REQ-0111).

It is not a requirement, but processing for Special Programs by Rubin Observatory is expected to use no 
more than $\sim$10\% of computational and storage capacity of the Rubin data processing cluster 
(i.e., proportional to the fraction of survey time spent; Section 6 of LSE-163).

The 10\% of the total data processing capacity that Rubin Observatory is 
required to reserve for \emph{all} User-Generated Processing includes that 
applied by users to Special Programs data.
There is no additional capacity {\it for users} that will be reserved only for Special 
Programs data (LSR-REQ-0041).
