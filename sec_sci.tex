\section{Recommendations for Rubin DM to Enable Science from Special Programs}\label{sec:sci}

Recommended implementations for Rubin Data Management to meet the 
requirements related to Special Programs (Section~\ref{sec:req})
and enable science with data from Special Programs.

\subsection{Apply Region and Observing Mode Labels}\label{ssec:sci_labels}

\textbf{All visits should have at least one region label \textit{and} one observing mode 
label, and these labels should be available to users as query constraints for
all images and tables.}

\textbf{Region labels -- } 
applied to a visit based on its Right Acension and Declination.

To be applied to data by Rubin Data Management using, e.g., a look-up table.
Region labels would include, e.g., WFD low-dust, Galactic Plane, Galactic 
Bulge, SCP, NES, or Virgo; DDF fieldname; mini-survey region name.
A given visit is likely to have multiple region labels because, for example,
the DDFs are within the WFD low-dust region.
These could be stored in, e.g., a single string as a CSV.

The region labels should be available to users as query constraints for
all images and tables, via both the butler and the TAP service, but otherwise
the details of region label implementation are left to Rubin Data Management.

For example, for images, the region labels could be propagated to processed images
(e.g., visit, difference, and deep coadds), so that users may query the ObsTAP 
service for all images with a given region label.
For catalogs, the region labels could be stored for individual sources and
objects as a table column, or for TAP-accessible tables stored in a joinable
table such as the Visit, ccdVisit, or coaddPatches tables.
Another option may be to map regions into spatial divisions (HTM or HEALPix),
and enable users to query image and catalog products this way.

The spatial resolution of region labels is left to be determined as it depends
somewhat on the implementation.
In this document the region label is considered to be applied to a visit
if the visit overlaps the region, as a very simple implementation.
However, the field of view is so large that a finer resolution by ccd,
or by a smaller-mesh HTM map if the regions are determined for sources and objects,
would be possible (and more useful to users).

\textbf{Observing mode labels -- } 
applied to a visit based on the scheduler mode at the time of the observation
(i.e., survey cadence or strategy).

To be applied by the Survey Scheduler team via the observations scheduler.
Observing mode labels might include, e.g., WFD, DDF, TOO, mini-survey name, or other 
options like engineering, commissioning, and director's discretionary.
A given image would probably only have one observing mode label, but
it is left to the discretion of the Survey Scheduler team whether more than one
observing mode labels might be applied at a time and, e.g., held in a single string as a CSV.

The observing mode labels should be available to users as query constraints for
all single-visit processed and difference images and catalog sources,
via both the butler and the TAP service.
It would not make sense to apply observing mode labels to deep coadds or 
object tables, as these data products could be a mix of modes.


There are three main motivations for this recommendation.

\begin{enumerate}

\item To meet the requirement that metadata for observations associated 
with Special Programs is stored, and is sufficient for triggering 
real-time data processing recipes (Section~\ref{ssec:req_meta}).

\item To enable users to query and retrieve processed image and catalog
data associated with a specific Special Program, and meet the science goals
of that Special Program, when standard processing as been applied 
(e.g., Prompt Processing, Section~\ref{ssec:sci_pproc}).

\item To enable provenance when Special Programs data in included in standard
processing and the Main Survey's data products (e.g., if used to improve the
all-sky coadd).

\end{enumerate}

\subsection{Run Special Processing When Possible and Necessary}\label{ssec:sci_sproc}

\textbf{Special Processing should be done by Rubin Data Management to 
produce unique and separate (and joinable) data products
for Special Programs when it is both possible and necessary.}

\textbf{Possible -- } When original or reconfigured versions of the LSST
Science Pipelines can be used, and no new algorithmic or software development,
or significant additional computational resources, are needed.

\textbf{Necessary -- } When the \emph{primary} science goal for a Special Program 
cannot be met by including the data in standard processing (e.g., Prompt processing),
or where doing so would compromise the Main Survey data products (e.g., introduce additional
non-uniformity beyond what would be expected based on the survey strategy).

There are two main motivations for this recommendation.

\begin{enumerate}

\item To meet the requirement that Rubin Observatory produce
unique, separate, and joinable data products whenever this is possible 
with the original or reconfigured versions of the LSST Science Pipelines
(Section~\ref{ssec:req_dp}).

\item To enable science with Special Programs by all users, not just those
with the time and effort to process the data, and to reduce computational
load (and potential redundancy) in User-Generated Processing.

\end{enumerate}

The scope of Special Processing is at the discretion of
Rubin Data Management, including the timelines for data releases,
and there are two caveats to add to the above definitions.

\textbf{Caveat I. The definition of possible} comes from the requirement referenced
for the first motivation in the list above (i.e., is not defined by this document).
However, this definition failed to recognize cost-benefit tradeoffs to implementing
and operating Special Processing, and does not account for the costs of operating,
documenting, and maintaining additional processing campaigns and their data products.
These costs are especially challenging to estimate for non-standard data processing
(e.g., very high sky backgrounds, very short exposures; Section~\ref{ssec:proc_bounds_processing}).
Even in situations that require no algorithmic or software development,
the Rubin DM team might not have the staffing to handle Special Processing
for all cases that are technically "possible".
Current ballpark estimates are that the DM effort for Special Processing
should be commensurate with the 10\% level estimated for survey time,
storage space, and computational resources allocated for Special Programs.

\textbf{Caveat II. The definition of necessary} is defined here in this document only.
We emphasize that the term applies only to the \emph{primary} science goals, and does
not cover secondary or "bespoke" science goals for a Special Program.
Users should anticipate that User-Generated Processing will be needed for these cases.

The definitions of possible and necessary are further illustrated with examples of when
Special Processing vs. User-Generate Processing will be needed, in Sections
\ref{ssec:proc_special} and \ref{ssec:proc_user}, respectively.


\subsection{Include Special Programs Data in Prompt Processing}\label{ssec:sci_pproc}

\textbf{All visits that \emph{can} be processed by the Prompt pipelines and generate 
alerts \emph{should} be, in support of time domain and Solar System science goals.}

The condition ``can be processed" is ultimately left to the discretion of
Rubin Data Management in Rubin Operations, but it is expected to include
all standard and alternative visits in sky regions for which a template image exists.

The main motivation for this recommendation is that all time-domain and 
moving-object science goals (two of the four science pillars for the LSST) are
enhanced by any and all additional observations, even if they are not
optimized for the science (as the WFD area's cadence is being optimized).

Further discussion of standard Prompt processing of Special Programs data, with
examples and a discussion of a few challenges,
is provided in Section~\ref{sssec:proc_reg_prompt}.

\subsection{Provide Computational Resources for User-Generated Processing}\label{ssec:sci_comp}

Rubin Observatory will reserve 10\% of its total data processing capacity for users.
This component would include {\it all} user processing and re-processing of any and 
all LSST data, including Special Programs. 
This is already a requirement, as mentioned in Section~\ref{ssec:req_proc}.

As described in Section~\ref{ssec:proc_user} this processing capacity will be 
accessed via Rubin Science Platform, with a supported software environment and 
infrastructure for batch processing \citedsp{dmtn-202}.

Very computationally intense processing (e.g., shift-and-stack for faint moving 
objects) will likely require external resources\footnote{For more details about 
the boundary between what Rubin Observatory will provide (in terms of data products 
and processing resources) and what will be left to the expertise of the science community, 
see \url{https://www.lsst.org/about/dm/data-products}}.

\subsubsection{Rubin Science Platform Capabilities}\label{sssec:sci_rsp}

Users will need to be able to query for data that restricts by sky region and 
observing mode label. 
This can be accomplished by including those labels in all image and catalog 
metadata as described in Section~\ref{ssec:sci_labels}, as the TAP service 
and butler already provide the mechanism for user-specified queries.

Users will need to be able to discover Special Programs data products when 
browsing data.
This could be accomplished with, e.g., a toggle to overlay region boundaries
for Special Programs when browsing all-sky coadds built from Main Survey data
in the Portal Aspect.


