\section{Recommendations to Enable Science from Special Programs}\label{sec:sci}

Recommended implementations for Rubin Data Management to meet the 
requirements related to Special Programs (Section~\ref{sec:req})
and enable science with data from Special Programs.

\subsection{Region and Program Labels}\label{ssec:sci_labels}

\textbf{All visits should have a region label \textit{and} a program 
label, and these labels should be propagated to processed images and catalog data.}

\textbf{Region Labels -- } based on Right Acension and Declination.
Region labels should include, e.g., WFD low-dust, Galactic Plane, Galactic 
Bulge, SCP, NES, or Virgo; DDF fieldname; mini-survey region name.
Region labels should be propagated to processed images (visit, 
difference, and deep coadds), catalog sources, and catalog static-sky objects. 

\textbf{Program Labels -- } 
Based on survey cadeance or strategy.
Program labels might include, e.g., WFD, DDF, TOO, mini-survey name, or other 
options like engineering, commissioning, and director's discretionary.
Program labels should be propagated to single-visit processed and 
difference images and catalog sources.
It would not make sense to propagate program labels to deep coadds or 
object tables, as these data products could be a mix of programs.

There are three main motivations for this recommendation.

\begin{enumerate}

\item To meet the requirement that metadata for observations associated 
with Special Programs is stored, and is sufficient for triggering 
real-time data processing recipes (Section~\ref{ssec:req_meta}).

\item To enable users to query and retrieve processed image and catalog
data associated with a specific Special Program, and meet the science goals
of that Special Program, when regular processing as been applied 
(e.g., Prompt Processing, Section~\ref{ssec:sci_pproc}).

\item To enable provenance when Special Programs data in included in regular
processing and the WFD program data products (e.g., if used to improve the
all-sky coadd).

\end{enumerate}

\subsection{Special Processing}\label{ssec:sci_sproc}

\textbf{Special Processing should be done by Rubin Data Management to 
produce unique and separate data products
for Special Programs when it is both possible and necessary.}

\textbf{Possible -- } When original or reconfigured versions of the LSST
Science Pipelines can be used, and no new algorithmic or software development,
or significant additional computational resources, are needed.

\textbf{Necessary -- } When the primary science goal for a Special Program 
cannot be met by including the data in regular processing (e.g., Prompt processing),
or where doing so would compromise the WFD data products (e.g., introduce non-uniformity).

The definitions of possible and necessary, and the scope of Special Processing,
are ultimately left to the discretion of Rubin Data Management in Operations.
Note that secondary science goals may be considered as ``not necessary".

There are two main motivations for this recommendation.

\begin{enumerate}

\item To meet the requirement that Rubin Observatory produce
unique, separate, and joinable data products whenever this is possible 
with the original or reconfigured versions of the LSST Science Pipelines
(Section~\ref{sec:req}).

\item To enable science with Special Programs by all users, not just those
with the time and effort to process the data, and to reduce computational
load (and potential redundancy) in User-Generated Processing.

\end{enumerate}

Further discussion on Special Processing, with examples, is provided in Section~\ref{ssec:proc_special}.

\subsection{Prompt Processing}\label{ssec:sci_pproc}

\textbf{All visits that \emph{can} be processed by the Prompt pipelines and generate 
alerts \emph{should} be, in support of time domain and Solar System science goals.}

The condition ``can be processed" is ultimately left to the discretion of
Rubin Data Management in Rubin Operations, but it is expected to include
all standard and alternative visits in sky regions for which a template image exists.

The main motivation for this recommendation is that all time-domain and 
moving-object science goals (two of the four science pillars for the LSST) are
enhanced by any and all additional observations, even if they are not
optimized for the science (as the WFD program's cadence is being optimized).

Further discussion of regular Prompt processing of Special Programs data, with
examples and a discussion of a few challenges,
is provided in Section~\ref{sssec:proc_reg_prompt}.

\subsection{Computational Resources for User-Generated Processing}\label{ssec:sci_comp}

Rubin Observatory will reserve 10\% of its total data processing capacity for users.
This is already a requirement, as mentioned in Section~\ref{ssec:req_ug}.
This component would include {\it all} user processing and re-processing of any and 
all LSST data, including Special Programs. 

As described in Section~\ref{ssec:proc_user} this processing capacity will be 
accessed via Rubin Science Platform, with a supported software environment and 
infrastructure for batch processing \citedsp{dmtn-202}.

Very computationally intense processing (e.g., shift-and-stack for faint moving 
objects) will likely require external resources\footnote{For more details about 
the boundary between what Rubin Observatory will provide (in terms of data products 
and processing resources) and what will be left to the expertise of the science community, 
see \url{https://www.lsst.org/about/dm/data-products}}.

\subsection{Rubin Science Platform Capabilities}\label{ssec:sci_rsp}

Users will need to be able to query for data that restricts by sky region and 
program label. 
This can be accomplished by including those labels in all image and catalog 
metadata as described in Section~\ref{ssec:sci_labels}, as the TAP service 
and butler already provide the mechanism for user-specified queries.

Users will need to be able to discover Special Programs data products when 
browsing data.
This could be accomplished with, e.g., a toggle to overlay region boundaries
for Special Programs when browsing all-sky coadds built from WFD program data
in the Portal Aspect.


