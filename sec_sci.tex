\section{Recommended Implementations to Enable Science from Special Programs}\label{sec:sci}

Recommendations regarding implementations by Rubin Data Management to meet the 
requirements related to Special Programs summarized in Section~\ref{sec:req},
which will enable science with data from Special Programs.

\subsection{Region and Program Labels}\label{ssec:sci_labels}

\textbf{All visits should have a region label \textit{and} a program 
label, and these labels should be propagated to processed images and catalog data.}

\textbf{Region Labels -- } based on Right Acension and Declination.
Region labels should include, e.g., WFD low-dust, Galactic Plane, Galactic 
Bulge, SCP, NES, or Virgo; DDF fieldname; mini-survey region name.
Region labels should be propagated to processed images (visit, 
difference, and deep coadds), catalog sources, and catalog static-sky objects. 

\textbf{Program Labels -- } 
Based on survey cadeance or strategy.
Program labels might include, e.g., WFD, DDF, mini-survey name, or other 
options like engineering, commissioning, and director's discretionary.
Program labels should be propagated to single-visit processed and 
difference images and catalog sources.

It would not make sense to propagate program labels to deep coadds or 
object tables, as these data products could be a mix of programs.

The existence and propagation of these labels is not itself a requirement.
It is recommended for implementation for three reasons.

\begin{enumerate}

\item To meet the requirement that metadata for observations associated 
with Special Programs is stored, and is sufficient for triggering 
real-time data processing recipes (Section~\ref{ssec:req_meta}).

\item To enable users to query and retrieve processed image and catalog
data associated with a specific Special Program, and meet the science goals
of that Special Program, when regular processing as been applied 
(e.g., Prompt Processing, Section~\ref{ssec:sci_pproc}).

\item To enable provenance when Special Programs data in included in regular
processing and the WFD program data products (e.g., if used to improve the
all-sky coadd).

\end{enumerate}

\subsection{Special Processing}\label{ssec:sci_sproc}

\textbf{Special Processing should be done by Rubin Data Management to 
produce unique and separate (but joinable) data products
for Special Programs when it is both possible and necessary.}

\textbf{Possible -- } When original or reconfigured versions of the LSST
Science Pipelines can be used, and no new algorithmic or software development,
or significant additional computational resources, are needed.

\textbf{Necessary -- } When the science goals for a Special Program cannot
be met by only including the data in regular processing (e.g., Prompt processing).

\begin{enumerate}

\item To meet the requirement that Rubin Observatory produce
unique, separate, and joinable data products whenever this is possible 
with the original or reconfigured versions of the LSST Science Pipelines
(Section~\ref{sec:req}).

\item To enable science with Special Programs by all users, not just those
with the time and effort to process the data, and to reduce computational
load (and potential redundancy) in User-Generated Processing.

\end{enumerate}

To further illustrate this recommendation, a few examples of Special Processing
cases are provided below.

\begin{itemize}

\item \textbf{Possible and necessary:}

\item \textbf{Possible but not necessary:}
a time-domain mini-survey that uses standard visits \emph{could}
have separate difference-image analysis object and source catalogs
generated, but this is not necessary as the science goals for the
mini-survey can be met by processing its data with regular Prompt
Processing (Section~\ref{ssec:sci_pproc}), and ensuring the
data is properly labeled (Section~\ref{ssec:sci_labels}).

\item \textbf{Necessary but not possible:}

\end{itemize}


RESTART EDITING PROCESS HERE



\textbf{Case B (possible and necessary for RDM to create separate data products) -- }
For some science goals, users will need distinct data products composed solely from visits associated with certain Special Programs (Section~\ref{ssec:proc_rdm}).
These cases are usually associated with image coaddition and the static-sky catalogs 
of sources detected and measured in the deeply coadded images. 

\textbf{Case C (not possible, so user-generated data products will be needed) -- }
For some science goals, users might require unique and separate data products, 
new algorithms or processing routines, and/or significant additional 
computational resources for Special Programs.
In these situations, user processing and user-generated data products will be 
needed (Section~\ref{ssec:proc_user}).
These cases are usually related to time domain science goals that utilize 
intermediate-timescale ``custom" coadds (e.g., weekly, monthly stacks), 
science goals that require special processing like shift-and-stack techniques, 
or non-standard visits that are outside the boundaries of what the LSST
Science Pipelines can process (Section~\ref{ssec:proc_bounds}).


\subsection{Prompt Processing}\label{ssec:sci_pproc}

\textbf{Special Processing should produce unique and separate data products
for Special Programs when it is both possible and necessary.}



\subsection{Computational Resources for User-Generated Processing}\label{ssec:sci_comp}

As mentioned in Section~\ref{ssec:req_ug}, Rubin Observatory will reserve 10\% of its 
total data processing capacity for users.
This component would include {\it all} user processing and re-processing of any and 
all LSST data, including Special Programs. 

As described in Section~\ref{ssec:proc_user} this processing capacity will be 
accessed via Rubin Science Platform, with a supported software environment and 
infrastructure for batch processing \citedsp{dmtn-202}.

Very computationally intense processing (e.g., shift-and-stack for faint moving 
objects) will likely require external resources\footnote{For more details about 
the boundary between what Rubin Observatory will provide (in terms of data products 
and processing resources) and what will be left to the expertise of the science community, 
see \url{https://www.lsst.org/about/dm/data-products}}.


\subsection{Rubin Science Platform Capabilities}\label{ssec:sci_rsp}

Users will need to be able to query for data that restricts by sky region and 
program label. 
This can be accomplished by including those labels in all image and catalog 
metadata as described in Section~\ref{ssec:sci_labels}, as the TAP service 
and butler already provide the mechanism for user-specified queries.

Users will need to be able to discover Special Programs data products when 
browsing data, such as all-sky maps. 


